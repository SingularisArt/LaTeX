% basics
\usepackage[utf8]{inputenc} %
\usepackage[T1]{fontenc} %
\usepackage{braket} % Needed to use the bra ket notation
\usepackage{sidenotes} %
\usepackage{textcomp} %
\usepackage{url} %
\usepackage{hyperref} % Change color of links
\hypersetup{
  colorlinks,
  linkcolor={black},
  citecolor={black},
  urlcolor={blue!80!black}
}
\usepackage{colortbl} % Change color of the tables
\usepackage{graphicx}
\usepackage{lipsum} % Generate random text
\usepackage{float} % 
\usepackage{booktabs} %
\usepackage[shortlabels]{enumitem} %
\usepackage{emptypage} %
\usepackage{caption} %
\usepackage{subcaption} %
\usepackage{multicol} %
\usepackage[usenames,dvipsnames]{xcolor} %
\usepackage{amsmath, amsfonts, mathtools, amsthm, amssymb} %
\usepackage{mathrsfs} %
\usepackage{cancel} %
\usepackage{bm} %
\usepackage{systeme} %
\usepackage{stmaryrd} %
\usepackage{lmodern}
\usepackage{microtype}
\usepackage{upgreek}
\usepackage[misc]{ifsym}
\usepackage{anyfontsize}


\DeclareMathOperator{\sgn}{sgn}
\DeclareMathOperator{\GL}{GL}
\DeclareMathOperator{\im}{Im}
\DeclareMathOperator{\Ker}{Ker}
\DeclareMathOperator{\Hom}{Hom}
\DeclareMathOperator{\Tr}{Tr}
\DeclareMathOperator{\Hol}{Hol}
\DeclareMathOperator{\Aut}{Aut}
\DeclareMathOperator{\Fit}{Fitt}
\DeclareMathOperator{\coker}{coker}
\DeclareMathOperator{\Ext}{Ext}
\DeclareMathOperator{\Tor}{Tor}
\DeclareMathOperator{\Der}{Der}
\DeclareMathOperator{\PDer}{PDer}
\DeclareMathOperator{\rad}{rad}

\let\svlim\lim\def\lim{\svlim\limits}
\let\implies\Rightarrow
\let\impliedby\Leftarrow
\let\iff\Leftrightarrow
\let\epsilon\varepsilon
\newcommand\contra{\scalebox{1.1}{$\lightning$}}

% tables
\setlength{\tabcolsep}{5pt}
\renewcommand{\arraystretch}{1.5}

% horizontal rule
\newcommand\hr{
  \noindent\rule[0.5ex]{\linewidth}{0.5pt}
}

% hide parts
\newcommand\hide[1]{}

% si unitx
\usepackage{siunitx}
\sisetup{locale = FR}
\renewcommand\vec[1]{\mathbf{#1}}
\newcommand\mat[1]{\mathbf{#1}}

% tikz
\usepackage{tikz}
\usepackage{tikz-cd}
\usepackage{tkz-euclide}
\usetikzlibrary{shapes.geometric}
\usetikzlibrary{intersections, angles, quotes, calc, positioning}
\usetikzlibrary{tikzmark}
\usetikzlibrary{arrows.meta}
\usetikzlibrary{
	calc,
	patterns,
	positioning
}

\usepackage{pgfplots}
\pgfplotsset{compat=1.13}

% pgfplots
\pgfplotsset{
  compat=1.16,
  samples=200,
  clip=false,
  my axis style/.style={
    axis x line=middle,
    axis y line=middle,
    legend pos=outer north east,
    axis line style={
      ->,
    },
    legend style={
      font=\footnotesize
    },
    label style={
      font=\footnotesize
    },
    tick label style={
      font=\footnotesize
    },
    xlabel style={
      at={
        (ticklabel* cs:1)
      },
      anchor=west,
      font=\footnotesize,
    },
    ylabel style={
      at={
        (ticklabel* cs:1)
      },
      anchor=west,
      font=\footnotesize,
    },
    xlabel=$x$,
    ylabel=$y$
  },
}

% theorems
\usepackage{thmtools}
\usepackage{tcolorbox}
\usepackage[framemethod=TikZ]{mdframed}
\mdfsetup{skipabove=1em,skipbelow=0em}

\theoremstyle{definition}

\declaretheoremstyle[
  headfont=\bfseries\sffamily\color{ForestGreen!70!black}, bodyfont=\normalfont,
  mdframed={
    linewidth=2pt,
    rightline=false, topline=false, bottomline=false,
    linecolor=ForestGreen, backgroundcolor=ForestGreen!5,
  }
]{thmgreenbox}

\declaretheoremstyle[
  headfont=\bfseries\sffamily\color{NavyBlue!70!black}, bodyfont=\normalfont,
  mdframed={
    linewidth=2pt,
    rightline=false, topline=false, bottomline=false,
    linecolor=NavyBlue, backgroundcolor=NavyBlue!5,
  }
]{thmbluebox}

\declaretheoremstyle[
  headfont=\bfseries\sffamily\color{NavyBlue!70!black}, bodyfont=\normalfont,
  mdframed={
    linewidth=2pt,
    rightline=false, topline=false, bottomline=false,
    linecolor=NavyBlue
  }
]{thmblueline}

\declaretheoremstyle[
  headfont=\bfseries\sffamily\color{RawSienna!70!black}, bodyfont=\normalfont,
  mdframed={
    linewidth=2pt,
    rightline=false, topline=false, bottomline=false,
    linecolor=RawSienna, backgroundcolor=RawSienna!5,
  }
]{thmredbox}

\declaretheoremstyle[
headfont=\bfseries\sffamily\color{RawSienna!70!black}, bodyfont=\normalfont,
headpunct=,
mdframed={
  linewidth=2pt,
  rightline=false, leftline=false, topline=true, bottomline=true,
  linecolor=RawSienna,
}
]{thmexcbox}

\declaretheoremstyle[
headfont=\bfseries\sffamily\color{RawSienna!70!black}, bodyfont=\normalfont,
mdframed={
  linewidth=2pt,
  rightline=false, leftline=false, topline=false, bottomline=false,
}
]{thmredname}

\declaretheoremstyle[
  headfont=\bfseries\sffamily\color{RawSienna!70!black}, bodyfont=\normalfont,
  numbered=no,
  mdframed={
    linewidth=2pt,
    rightline=false, topline=false, bottomline=false,
    linecolor=RawSienna, backgroundcolor=RawSienna!1,
  },
  qed=\qedsymbol
]{thmproofbox}

\declaretheoremstyle[
  headfont=\bfseries\sffamily\color{NavyBlue!70!black}, bodyfont=\normalfont,
  numbered=no,
  mdframed={
    linewidth=2pt,
    rightline=false, topline=false, bottomline=false,
    linecolor=NavyBlue, backgroundcolor=NavyBlue!1,
  },
]{thmexplanationbox}

\declaretheorem[style=thmgreenbox, name=Definition]{definition}

\declaretheorem[style=thmredbox, name=Proposition]{prop}
\declaretheorem[style=thmredbox, name=Theorem]{theorem}
\declaretheorem[style=thmredbox, name=Lemma]{lemma}
\declaretheorem[style=thmredbox, numbered=no, name=Corollary]{corollary}
\declaretheorem[style=thmredbox, numbered=no, name=Identity]{identity}
\declaretheorem[style=thmexcbox, numbered=no, name=]{exc}

\renewcommand\qedsymbol{\color{RawSienna!70!black}Q.E.D.}
\declaretheorem[style=thmproofbox, name=Proof]{replacementproof}
\renewenvironment{proof}[1][\proofname]{\vspace{-9pt}\begin{replacementproof}}{\end{replacementproof}}

\declaretheorem[style=thmbluebox, numbered=no, name=Example]{example}
\declaretheorem[style=thmblueline, numbered=no, name=Remark]{remark}

\newtheorem*{notation}{Notation}
\newtheorem*{note}{Note}
\newtheorem*{previouslyseen}{As previously seen}
\newtheorem*{problem}{Problem}
\newtheorem*{observe}{Observe}
\newtheorem*{property}{Property}
\newtheorem*{intuition}{Intuition}

\def\testdateparts#1{\dateparts#1\relax}
\def\dateparts#1 #2 #3 #4 #5\relax{
  \marginpar{\small\textsf{\mbox{#1 #2 #3 #5}}}
}

\makeatletter

\def\@lesson{}%
\newcommand{\lesson}[3]{
  \ifthenelse{\isempty{#3}}{%
    \def\@lesson{Lecture #1}%
  }{%
    \def\@lesson{Lecture #1: #3}%
  }%
  \subsection*{\@lesson}
  \addcontentsline{toc}{section}{\@lesson}%
  \testdateparts{#2}
}

\newcommand{\exercise}[1]{%
  \def\@exercise{#1}%
  \vspace*{-1.5em}
  \subsection*{Exercise #1}
}
\newcommand{\subexercise}[1]{%
  \def\@subexercise{#1}
  \subsubsection*{Exercise \@exercise.#1}
}
\newcommand{\solution}{
  \subsection*{Solution \@exercise}
}
\newcommand{\subsolution}{
  \subsubsection*{Solution \@exercise.\@subexercise}
}

% fancy headers
\usepackage{fancyhdr}
\pagestyle{fancy}

\fancyhead[RO,LE]{\@lesson}
\fancyhead[RE,LO]{Hashem A. Damrah}
\fancyfoot[C]{\thepage}

\makeatother

% notes
\usepackage{todonotes}

% figure support
\usepackage{import}
\pdfminorversion=7
\usepackage{pdfpages}
\usepackage{transparent}
\newcommand{\incfig}[2][1]{%
  \def\svgwidth{#1\columnwidth}
  \import{./figures/}{#2.pdf_tex}
}

\usepackage{hyperref}
\hypersetup{hidelinks}
\usepackage{xifthen}
\usepackage{fontawesome}

% http://tex.stackexchange.com/questions/76273/multiple-pdfs-with-page-group-included-in-a-single-page-warning
\pdfsuppresswarningpagegroup=1
