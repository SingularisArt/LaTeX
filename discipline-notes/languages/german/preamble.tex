% basics
\usepackage[utf8]{inputenc} %
\usepackage[T1]{fontenc} %
\usepackage{braket} % Needed to use the bra ket notation
\usepackage{sidenotes} %
\usepackage{textcomp} %
\usepackage{url} %
\usepackage{hyperref} % Change color of links
\hypersetup{
	colorlinks,
	linkcolor={black},
	citecolor={black},
	urlcolor={blue!80!black}
}
\usepackage{colortbl} % Change color of the tables
\usepackage{graphicx}
\usepackage{lipsum} % Generate random text
\usepackage{float} % 
\usepackage{booktabs} %
\usepackage[shortlabels]{enumitem} %
\usepackage{emptypage} %
\usepackage{caption} %
\usepackage{subcaption} %
\usepackage{multicol} %
\usepackage[usenames,dvipsnames]{xcolor} %
\usepackage{amsmath, amsfonts, mathtools, amsthm, amssymb} %
\usepackage{mathrsfs} %
\usepackage{cancel} %
\usepackage{bm} %
\usepackage{systeme} %
\usepackage{stmaryrd} %
\usepackage{lmodern}
\usepackage{microtype}
\usepackage{upgreek}
\usepackage[misc]{ifsym}
\usepackage{anyfontsize}


\DeclareMathOperator{\sgn}{sgn}
\DeclareMathOperator{\GL}{GL}
\DeclareMathOperator{\im}{Im}
\DeclareMathOperator{\Ker}{Ker}
\DeclareMathOperator{\Hom}{Hom}
\DeclareMathOperator{\Tr}{Tr}
\DeclareMathOperator{\Hol}{Hol}
\DeclareMathOperator{\Aut}{Aut}
\DeclareMathOperator{\Fit}{Fitt}
\DeclareMathOperator{\coker}{coker}
\DeclareMathOperator{\Ext}{Ext}
\DeclareMathOperator{\Tor}{Tor}
\DeclareMathOperator{\Der}{Der}
\DeclareMathOperator{\PDer}{PDer}
\DeclareMathOperator{\rad}{rad}

\let\svlim\lim\def\lim{\svlim\limits}
\let\implies\Rightarrow
\let\impliedby\Leftarrow
\let\iff\Leftrightarrow
\let\epsilon\varepsilon
\newcommand\contra{\scalebox{1.1}{$\lightning$}}

% tables
\setlength{\tabcolsep}{5pt}
\renewcommand{\arraystretch}{1.5}

% horizontal rule
\newcommand\hr{
	\noindent\rule[0.5ex]{\linewidth}{0.5pt}
}

% hide parts
\newcommand\hide[1]{}

% si unitx
\usepackage{siunitx}
\sisetup{locale = FR}
\renewcommand\vec[1]{\mathbf{#1}}
\newcommand\mat[1]{\mathbf{#1}}

% fancy headers
\usepackage{fancyhdr}
\pagestyle{fancy}

\fancyhead[LE,RO]{Hashem A. Damrah}
\fancyhead[RO,LE]{}
\fancyhead[RE,LO]{}
\fancyfoot[LE,RO]{\thepage}
\fancyfoot[C]{\leftmark}

% notes
\usepackage{todonotes}
\usepackage{tcolorbox}

% figure support
\usepackage{import}
\pdfminorversion=7
\usepackage{pdfpages}
\usepackage{transparent}
\newcommand{\incfig}[2][1]{
	\def\svgwidth{#1\columnwidth}
	\import{./figures/}{#2.pdf_tex}
}

\pdfsuppresswarningpagegroup=1

% references
\usepackage{hyperref}
\hypersetup{hidelinks}
\usepackage{xifthen}
\usepackage{fontawesome}

\newcommand\urlref[2]{%
	\href{#1}{\raisebox{0.15ex}{\scriptsize \faLink}\:\textup{\textbf{#2}}}%
}
\newcommand\pdfref[3]{%
	\href{paper-manager://open-paper?id=#1&page=#2}{%
		\textup{[\textbf{\ifthenelse{\isempty{#3}}{here}{#3}}]}}%
}
\newcommand\absolutefileref[2]{%
	\href{run:#1}{\raisebox{0.15ex}{\scriptsize \faFile}\:\textup{\textbf{#2}}}%
}

% tikz
\usepackage{tikz}
\usepackage{tikz-cd}
\usepackage{tkz-euclide}
\usetikzlibrary{shapes.geometric}
\usetikzlibrary{intersections, angles, quotes, calc, positioning}
\usetikzlibrary{tikzmark}
\usetikzlibrary{arrows.meta}
\usetikzlibrary{
	calc,
	patterns,
	positioning
}

\usepackage{pgfplots}
\pgfplotsset{compat=1.13}

% pgfplots
\pgfplotsset{
	compat=1.16,
	samples=200,
	clip=false,
	my axis style/.style={
			axis x line=middle,
			axis y line=middle,
			legend pos=outer north east,
			axis line style={
					->,
				},
			legend style={
					font=\footnotesize
				},
			label style={
					font=\footnotesize
				},
			tick label style={
					font=\footnotesize
				},
			xlabel style={
					at={
							(ticklabel* cs:1)
						},
					anchor=west,
					font=\footnotesize,
				},
			ylabel style={
					at={
							(ticklabel* cs:1)
						},
					anchor=west,
					font=\footnotesize,
				},
			xlabel=$x$,
			ylabel=$y$
		},
}

% theorems
\usepackage{thmtools}
\usepackage[framemethod=TikZ]{mdframed}
\mdfsetup{skipabove=1em,skipbelow=0em}

\theoremstyle{definition}

\declaretheoremstyle[
	headfont=\bfseries\sffamily\color{ForestGreen!70!black}, bodyfont=\normalfont,
	mdframed={
			linewidth=2pt,
			rightline=false, topline=false, bottomline=false,
			linecolor=ForestGreen, backgroundcolor=ForestGreen!5,
		}
]{thmgreenbox}

\declaretheoremstyle[
	headfont=\bfseries\sffamily\color{NavyBlue!70!black}, bodyfont=\normalfont,
	mdframed={
			linewidth=2pt,
			rightline=false, topline=false, bottomline=false,
			linecolor=NavyBlue, backgroundcolor=NavyBlue!5,
		}
]{thmbluebox}

\declaretheoremstyle[
	headfont=\bfseries\sffamily\color{NavyBlue!70!black}, bodyfont=\normalfont,
	mdframed={
			linewidth=2pt,
			rightline=false, topline=false, bottomline=false,
			linecolor=NavyBlue
		}
]{thmblueline}

\declaretheoremstyle[
	headfont=\bfseries\sffamily\color{RawSienna!70!black}, bodyfont=\normalfont,
	mdframed={
			linewidth=2pt,
			rightline=false, topline=false, bottomline=false,
			linecolor=RawSienna, backgroundcolor=RawSienna!5,
		}
]{thmredbox}

\declaretheoremstyle[
	headfont=\bfseries\sffamily\color{RawSienna!70!black}, bodyfont=\normalfont,
	mdframed={
			linewidth=2pt,
			rightline=false, leftline=false, topline=true, bottomline=true,
			linecolor=RawSienna,
		}
]{thmredlinesupdown}

\declaretheoremstyle[
	headfont=\bfseries\sffamily\color{thmredRawSienna!70!black}, bodyfont=\normalfont,
	mdframed={
			linewidth=2pt,
			rightline=false, leftline=false, topline=false, bottomline=false,
		}
]{thmredname}

\declaretheoremstyle[
	headfont=\bfseries\sffamily\color{RawSienna!70!black}, bodyfont=\normalfont,
	numbered=no,
	mdframed={
			linewidth=2pt,
			rightline=false, topline=false, bottomline=false,
			linecolor=RawSienna, backgroundcolor=RawSienna!1,
		},
	qed=\qedsymbol
]{thmproofbox}

\declaretheoremstyle[
	headfont=\bfseries\sffamily\color{NavyBlue!70!black}, bodyfont=\normalfont,
	numbered=no,
	mdframed={
			linewidth=2pt,
			rightline=false, topline=false, bottomline=false,
			linecolor=NavyBlue, backgroundcolor=NavyBlue!1,
		},
]{thmexplanationbox}

\captionsetup{justification=centering,margin=0.5cm}
\captionsetup{labelfont={color=black}}

\declaretheorem[style=thmgreenbox, name=Definition]{definition}

\declaretheorem[style=thmredbox, name=Proposition]{prop}
\declaretheorem[style=thmredbox, name=Theorem]{theorem}
\declaretheorem[style=thmredbox, name=Lemma]{lemma}
\declaretheorem[style=thmredbox, numbered=no, name=Corollary]{corollary}
\declaretheorem[style=thmredbox, numbered=no, name=Identity]{identity}
\declaretheorem[style=thmredlinesupdown, numbered=no, name=Exercise]{exc}
\newcommand{\solution}{\noindent\textbf{\sffamily{\color{RawSienna!70!black}Solution.}}}

\declaretheorem[style=thmbluebox, numbered=no, name=Example]{example}
\declaretheorem[style=thmblueline, numbered=no, name=Remark]{remark}

\declaretheorem[style=thmproofbox, name=Proof]{replacementproof}
\renewenvironment{proof}[1][\proofname]{\vspace{-10pt}\begin{replacementproof}}{\end{replacementproof}}

\declaretheorem[style=thmexplanationbox, name=Proof]{tmpexplanation}
\newenvironment{explanation}[1][]{\vspace{-10pt}\begin{tmpexplanation}}{\end{tmpexplanation}}

\newtheorem*{notation}{Notation}
\newtheorem*{note}{Note}
\newtheorem*{previouslyseen}{As previously seen}
\newtheorem*{problem}{Problem}
\newtheorem*{observe}{Observe}
\newtheorem*{property}{Property}
\newtheorem*{intuition}{Intuition}

% this will contain the current date in yyyy-mm-dd format
\def\formatteddate{}
\newcommand\fileref[2]{
	\IfFileExists{./\formatteddate/#1}{
		\absolutefileref{./\formatteddate/#1}{#2}
	}{
		\textcolor{gray}{\absolutefileref{./\formatteddate/#1}{#2}}
	}
}
\newcommand{\xournal}{\fileref{note.xoj}{Handwritten notes}}%

% styles
\def\thickhrulefill{\leavevmode \leaders \hrule height 1ex \hfill \kern \z@}
\def\@makechapterhead#1{%
	\vspace*{10\p@}%
	{\parindent \z@ \centering \reset@font
		\thickhrulefill\quad
		\scshape \@chapapp{} \thechapter
		\quad \thickhrulefill
		\par\nobreak
		\vspace*{10\p@}%
		\interlinepenalty\@M
		\hrule
		\vspace*{10\p@}%
		\Huge \bfseries #1\par\nobreak
		\par
		\vspace*{10\p@}%
		\hrule
		\vskip 20\p@
	}}
\def\@makeschapterhead#1{%
	\vspace*{10\p@}%
	{\parindent \z@ \centering \reset@font
		\thickhrulefill
		\par\nobreak
		\vspace*{10\p@}
		\interlinepenalty\@M
		\hrule
		\vspace*{10\p@}
		\Huge \bfseries #1\par\nobreak
		\par
		\vspace*{10\p@}
		\hrule
		\vskip 20\p@
	}}
\makeatother
