\chapter{You Already Know a Little Latin}
\label{chap:you_already_know_a_little_latin}

\section{The Latin Alphabet}
\label{sec:the_latin_alphabet}

One feature that makes Latin easy to learn is its alphabet. They have the same
alphabet as the English alphabet, but with a few modifications:

\begin{itemize}
	\item Latin never uses the letter \textit{W}.
	\item Few Latin words use \textit{K}, they use \textit{C} instead.
	\item Latin used \textit{I} and \textit{V} as both consonants and vowels
	      until much later, when someone thought to bend the \textit{I} into a
	      \textit{J} and round the \textit{V} into a \textit{U}.
\end{itemize}

% section the_latin_alphabet (end)

\section{Pronunciation}
\label{sub_sec:pronunciation}

\subsection{Classical Pronunciation}
\label{sub_sub_sec:classical_pronunciation}

The big advantage for you is Latin doesn't have any silent letters.

\subsubsection{Vowel Sounds}
\label{sub_sub_sec:vowel_sounds}

\begin{table}[H]
	\centering

	\begin{tabular}{|l|l|l|l|}
		\hline
		\textbf{Long Vowel} & \textbf{Pronunciation}                       & \textbf{Short Vowel} & \textbf{Pronunciation}                 \\
		\hline
		a                   & ah (f\textbf{\underline{a}}ther)             & a                    & uh (ide\textbf{\underline{a}})         \\
		e                   & ay (m\textbf{\underline{a}}te)               & e                    & eh (b\textbf{\underline{e}}d)          \\
		i                   & ee (s\textbf{\underline{ee}}d)               & i                    & ih (p\textbf{\underline{i}}t)          \\
		o                   & o (n\textbf{\underline{o}}te)                & o                    & oh (p\textbf{\underline{o}}t)          \\
		u                   & oo (m\textbf{\underline{oo}}n)               & u                    & u (p\textbf{\underline{u}}t)           \\
		y                   & uw/umlaut (German \textbf{\underline{u}}ber) & y                    & oh (German H\textbf{\underline{u}}tte) \\
		\hline
	\end{tabular}

	\caption{Vowel Soundings}
	\label{tab:vowel_soundings}
\end{table}

% subsubsection vowel_sounds (end)

\subsubsection{Diphthong Sounds}
\label{sub_sub_sec:diphthong_sounds}

\begin{definition}[Diphthongs]
	\label{def:diphthongs}

	The combinations of vowel sounds.
\end{definition}

Diphthongs are extremely common, I've merged them in Table
\ref{tab:diphthong_sounds_in_classical_latin}.

\begin{table}[H]
	\centering

	\begin{tabular}{|l|l|l|}
		\hline
		\textbf{Diphthong} & \textbf{Pronunciation} & \textbf{As in English}                             \\
		\hline
		ae                 & igh                    & f\textbf{\underline{igh}}t                         \\
		au                 & ow                     & h\textbf{\underline{how}}                          \\
		ei                 & ey                     & th\textbf{\underline{ey}}                          \\
		eu                 & eyoo                   & th\textbf{\underline{ey}} t\textbf{\underline{oo}} \\
		oe                 & oi                     & t\textbf{\underline{oi}}l                          \\
		ui                 & uey                    & chew                                               \\
		\hline
	\end{tabular}

	\caption{Diphthong Sounds in Classical Latin}
	\label{tab:diphthong_sounds_in_classical_latin}
\end{table}

% subsubsection diphthong_sounds (end)

\subsubsection{Consonant Sounds}
\label{sub_sub_sec:consonant_sounds}

Most Latin consonants have the same sound as English consonants, except the
following exceptions.

\begin{table}[H]
	\centering

	\begin{tabular}{|l|l|}
		\hline
		\textbf{Laitn Consonant} & \textbf{Pronunciation}                                  \\
		\hline
		c                        & k (\textbf{\underline{c}}an; never as in "cereal")      \\
		g                        & g (\textbf{\underline{g}}ood; never as in "genuine")    \\
		j                        & y (\textbf{\underline{y}}outh)                          \\
		r                        & r (always trilled)                                      \\
		s                        & s (\textbf{\underline{s}}oft; never as in "fans")       \\
		v                        & w (\textbf{\underline{w}}oman)                          \\
		x                        & ks (wa\textbf{\underline{x}}; never as in "xenophobic") \\
		z                        & dz (a\textbf{\underline{dz}}e)                          \\
		bs                       & ps (la\textbf{\underline{ps}}e)                         \\
		bt                       & pt (exce\textbf{\underline{pt}})                        \\
		ch                       & kh (\textbf{\underline{ch}}aos; never as in "cheer")    \\
		gn                       & ngn (ha\textbf{\underline{ngn}}ail)                     \\
		ph                       & p-h (to\textbf{\underline{p-h}}eavy)                    \\
		th                       & t (\textbf{\underline{t}}ourist)                        \\
		ti                       & ti (pa\textbf{\underline{ti}}o; never as in "nation")   \\
		\hline
	\end{tabular}

	\caption{Consonant Sounds in Classical Latin}
	\label{tab:consonant_sounds_in_classical_latin}
\end{table}

% subsection consonant_sounds (end)

% subsection classical_pronunciation (end)

\subsection{Ecclesiastical Pronunciation}
\label{sub_sec:ecclesiastical_pronunciation}

Later Latin pronunciation is similar in the vowel pronunciation, but the
diphthongs and consonants are pronounced differently.

\begin{table}[H]
	\centering

	\begin{tabular}{|l|l|l|}
		\hline
		\textbf{Diphthong} & \textbf{Pronunciation} & \textbf{As in English}                             \\
		\hline
		ae                 & ay                     & m\textbf{\underline{a}}te                          \\
		au                 & ow                     & h\textbf{\underline{ow}}                           \\
		ei                 & ey                     & th\textbf{\underline{ey}}                          \\
		eu                 & eyoo                   & th\textbf{\underline{ey}} t\textbf{\underline{oo}} \\
		oe                 & ay                     & m\textbf{\underline{a}}te                          \\
		ui                 & uey                    & ch\textbf{\underline{ewy}}                         \\
		\hline
	\end{tabular}

	\caption{Diphthong Sounds in Ecclesiastical Latin}
	\label{tab:diphthong_sounds_in_ecclesiastical_latin}
\end{table}

% subsection ecclesiastical_pronunciation (end)

% section pronunciation (end)

% chapter you_already_know_a_little_latin (end)

\newpage
