\nte[Section 1.8]{Oct 17 2023 Tue (14:00:38)}{Intro to Linear Transformations}

Another way to view $A\x = \b$ is that this $\x$ vector is transformed
into the $\b$ vector by multiplying it with matrix $A$.

\begin{example}
  \label{exm:transform_x_to_b_by_multiplying_by_A}

  \[%
    \begin{bNiceMatrix}[columns-width=auto]
      4 & -3 & 1 & 3 \\
      2 & 0 & 5 & 1 \\
    \end{bNiceMatrix}
    \begin{bNiceMatrix}[columns-width=auto]
      1 \\
      4 \\
      -1 \\
      3 \\
    \end{bNiceMatrix}
  .\]%
\end{example}

We can view this matrix-vector production as a type of function on vectors.

\begin{definition}[Transformation]
  \label{def:transformation}

  A \textbf{transformation} $T : \R^n \to \R^m$, under
  $T(\x) = A\x$ is a rule that assigns each vector $\x \in \R^n$ to
  another vector in $\R^m$. The set $\R^n$ is called the
  \textbf{domain} and $\R^m$ is called the \textbf{codomain}.

  \begin{figure}[H]
    \centering

    \begin{tikzpicture}
      \draw[color=black,very thick](-4,0) circle (1.5);
      \draw[no marks] (-4,-2) node{domain: $\R^n$};

      \draw[color=black,very thick](4,0) circle (2);
      \draw[no marks] (4,-2.5) node{codomain: $\R^m$};

      \filldraw[color=linecolor1!60, fill=linecolor1!5,very thick](3.5,0) circle (1);
      \draw[no marks,color=linecolor1] (3.5,-1.3) node{range};

      \draw[dashed,very thick](-4,1.5)--++(7.5,-0.5);
      \draw[dashed,very thick](-4,-1.5)--++(7.5,0.5);
      \draw[->,very thick](-3.25,0.75)node[left]{$\x$}--++(6.25,-0.25)node[right,color=linecolor1]{$T(\x)$};
      \draw[->,very thick](-3.25,-0.75)node[left]{$\y$}--++(6.25,0.25)node[right,color=linecolor1]{$T(\y)$};
      \draw[->,very thick](-3.25,0)node[left]{$\zero$}--++(6.25,-0.5);
    \end{tikzpicture}

    \caption{}
    \label{fig:transformation}
  \end{figure}
\end{definition}

$\forall \x \in \R^n$, $T(\x) \in \R^m$ is called the
\textbf{image} of $\x$ under $T$. You can think of the image as the $y$ value
from the function $f(x)$.

\begin{note}
  \label{nte:range_may_not_be_the_same_size_as_codomain}

  The range of a transformation may not be the same as the codomain. In general,
  the range is typically smaller than the codomain.
\end{note}

\section{Matrix Transformations}
\label{sec:matrix_transformations}

The rest of this section focuses on mappings associated with matrix
multiplication. $\forall \x \in \R^n$, $T(\x)$ is computed as $A\x$,
where $A$ is an $m \times n$ matrix. We denote this with $\x \mapsto A\x$ or $\x
\mapsto T(\x)$ or $T : \R^n \to \R^m$, where $n$ is the
number of columns of $A$ and $m$ is the number of rows of $A$. The range for a
matrix transformation is just the set of all linear combinations of the columns
of $A$.

\begin{question}
  \label{qst:matrix_transformation}

  Consider the following matrix and vectors
  \[%
    A =
    \begin{bNiceMatrix}[columns-width=auto]
      1 & -3 \\
      3 & 5 \\
      -1 & 7 \\
    \end{bNiceMatrix},\quad
    \u =
    \begin{bNiceMatrix}[columns-width=auto]
      2 \\
      -1 \\
    \end{bNiceMatrix},\quad
    \b =
    \begin{bNiceMatrix}[columns-width=auto]
      3 \\
      2 \\
      -5 \\
    \end{bNiceMatrix},\aand
    \c =
    \begin{bNiceMatrix}[columns-width=auto]
      3 \\
      2 \\
      5 \\
    \end{bNiceMatrix}
  .\]%
  Define the transformation $T: \R^2 \to \R^3$ as $T(\x) =
  A\x$. Note that $\x$ is a vector in the domain $\R^2$.

  \begin{enumerate}
    \label{enum:matrix_transformation_qst}

    \item Find the image of $\u$ under $T$.

    \item Find a vector $\x \in \R^2$ whose image under $T$ is $\b$.

    \item Is there more than one $\x$ whose image under $T$ is $\b$?

    \item Determine if $\c$ is in the range of $T$.
  \end{enumerate}
\end{question}

\begin{solution}
  \label{sol:matrix_transformation}

  \begin{enumerate}
    \label{enum:matrix_transformation_sol} $ $

    \item The question is asking what is $T(\u)$? We can compute this by
      \[%
        T(\u) = A\u =
        \begin{bNiceMatrix}[columns-width=auto]
          1 & -3 \\
          3 & 5 \\
          -1 & 7 \\
        \end{bNiceMatrix}
        \begin{bNiceMatrix}[columns-width=auto]
          2 \\
          -1 \\
        \end{bNiceMatrix} =
        2
        \begin{bNiceMatrix}[columns-width=auto]
          1 \\
          3 \\
          -1 \\
        \end{bNiceMatrix} -
        1\begin{bNiceMatrix}[columns-width=auto]
          -3 \\
          5 \\
          7 \\
        \end{bNiceMatrix} =
        \begin{bNiceMatrix}[columns-width=auto]
          5 \\
          1 \\
          9 \\
        \end{bNiceMatrix}
      .\]%

    \item The question is asking to find $\x$ such that $T(\x) = \b$ or $A\x =
      \b$. To solve, we need to convert this into an augmented matrix and solve
      using the Row Reduction Algorithm
      \[%
        \begin{bNiceArray}{c|c}[columns-width=auto]
          A & \b \\
        \end{bNiceArray} =
        \begin{bNiceArray}{cc|c}[columns-width=auto]
          1 & -3 & 3 \\
          3 & 5 & 2 \\
          -1 & 7 & -5 \\
        \end{bNiceArray}
        \rref
        \begin{bNiceArray}{cc|c}[columns-width=auto]
          \circled{1} & 0 & \sfrac{3}{2} \\
          0 & \circled{1} & -\sfrac{1}{2} \\
          0 & 0 & 0 \\
        \end{bNiceArray}
        \longleftrightarrow
        \x =
        \begin{bNiceMatrix}[columns-width=auto]
          \sfrac{3}{2} \\
          -\sfrac{1}{2} \\
        \end{bNiceMatrix}
      .\]%

    \item No, because the solution to $A\x = \b$ was unique because there aren't
      any free variables.

    \item The question is asking if $\exists \x \in \R^2$ such that
      $T(\x) = \c$ (i.e., $\c$ is the image of $\x$ under $T$). To solve, we
      need to convert this into an augmented matrix and solve using the Row
      Reduction Algorithm
      \[%
        \begin{bNiceArray}{c|c}[columns-width=auto]
          A & \c \\
        \end{bNiceArray} =
        \begin{bNiceArray}{cc|c}[columns-width=auto]
          \circled{1} & -3 & 3 \\
          3 & 5 & 2 \\
          -1 & 7 & 5 \\
        \end{bNiceArray}
        \echelon
        \begin{bNiceArray}{cc|c}[columns-width=auto]
          \circled{1} & -3 & 3 \\
          0 & \circled{1} & 2 \\
          0 & 0 & -35 \\
        \end{bNiceArray}
      .\]%
      The row $[\,0~0~-35\,]$ tells us that $A\x = \c$ is inconsistent. Thus,
      $\x$ does not exist, and $\c$ is not in the range of $T$. \qedhere
  \end{enumerate}
\end{solution}

% section matrix_transformations (end)

\section{Linear Transformations}
\label{sec:linear_transformations}

\begin{definition}[Linear Transformation]
  \label{def:linear_transformation}

  Let $T : \R^n \to \R^m$. The transformation $T$ is \textbf{linear} if it
  satisfies the following properties:
  \begin{enumerate}
    \label{enum:linear_transformation}

    \item $\forall \u, \v \in \R^m$, $T(\u + \v) = T(\u) + T(\v)$.
      \imp{``Preserving vector addition''}.

    \item $\forall c, \u \in \R^n$, $T(c\u) = cT(\u)$. \imp{``Preserving scalar
      multiplication''}.
  \end{enumerate}
\end{definition}

\begin{purpleframe}
  \label{prpl:linear_transformation}

  If $T$ is a linear transformation, then $T(\vec{0}) = \vec{0}$ and we can also
  combine both properties \circled{1} and \circled{2} into the following
  equation
  \[%
    T(c\u + d\v) = cT(\u) + dT(\v)
  ,\]%
  $\forall c, d, \u, \v \in \R^n$, where $\R^n$ is the
  domain of $T$. In general, we can write
  \[%
    T(c_1\v_1 + c_2\v_2 + \ldots + c_p\v_p) = \sum_{i=1}^p c_i(T(\v_i))
  .\]%
  Combing both properties is sometimes called the \textbf{superposition
  principle}.
\end{purpleframe}

\begin{note}
  \label{nte:linear_transformation}

  Matrix transformations $T(\x) = A\x$ are linear because
  \[%
    T(c\u + d\v) = A(c\u + d\v) = A(c\u) + A(d\v) = cA\u + dA\v = cT(\u) + dT(\v)
  ,\]%
  however, not all linear transformations start in this form.
\end{note}

\begin{question}
  \label{qst:linear_transformation_1}

  Given a scalar $r$, let $T : \R^2 \to \R^2$, under $T(\x) = r\x$. When $0 \le
  r \le 1$, the transformation $T$ is called a \textbf{contraction}. When $r >
  1$, then $T$ is called a \textbf{dilation}. Let $r = 3$ and show that $T$ is a
  linear transformation.
\end{question}

\begin{solution}
  \label{sol:linear_transformation_1}

  Let $c, d, \u, \v \in \R^2$. We need to show that $T(c\u + d\v) = cT(\u) +
  dT(\v)$. So, we get
  \[%
    T(c\u + d\v) = 3(c\u + d\v) = 3c\u + 3d\v = c \cdot \underbrace{3\u}_{T(\u)} + d \cdot \underbrace{3\v}_{T(\v)} = cT(\u) + dT(\v)
  .\qedhere\]%
\end{solution}

\begin{question}
  \label{qst:linear_transformation_2}

    Let $T : \R^2 \to \R^3$ be a transformation defined by
    \[%
			T\left(\begin{bNiceMatrix}[columns-width=auto]
				x \\
				y \\
			\end{bNiceMatrix}\right) =
			\begin{bNiceMatrix}[columns-width=auto]
				-5x-y \\
				-3x-3 \\
				x-3y \\
			\end{bNiceMatrix}
    .\]%
    Determine if $T$ is linear. Note that $T$ can also be written as
    $T(x,y)=(-5x-y,-3x-3,x-3y)$.
\end{question}

\begin{solution}
  \label{sol:linear_transformation_2}

  We want to show $T(c\u + d\v) = cT(\u) + dT(\v)$. Let $\u = (2, 1)$ and $\v =
  (0, 3)$ (pick nontrivial values). Let $c = 1$ and $d = -2$. Now, let's compute
  $T(c\u + d\v)$
  \[%
    T(c\u + d\v) = T(1\u - 2\v)
    = T\left(\begin{bNiceMatrix}[columns-width=auto]
        2 \\
        1 \\
      \end{bNiceMatrix} -
      2\begin{bNiceMatrix}[columns-width=auto]
        0 \\
        3 \\
    \end{bNiceMatrix}\right)
    = T\left(\begin{bNiceMatrix}[columns-width=auto]
        2 \\
        -5 \\
    \end{bNiceMatrix}\right)
    = \begin{bNiceMatrix}[columns-width=auto]
      -5(2) - (-5) \\
      -3(2) - 3 \\
      (2) - 3(-5) \\
    \end{bNiceMatrix}
    = \begin{bNiceMatrix}[columns-width=auto]
      -5 \\
      -9 \\
      17 \\
    \end{bNiceMatrix}
  .\]%
  Now, let's compute $cT(\u) + dT(\v)$
  \[%
    cT(\u) + dT(\v) = 1T\left(\begin{bNiceMatrix}[columns-width=auto]
        2 \\
        1 \\
    \end{bNiceMatrix}\right) -
    2T\left(\begin{bNiceMatrix}[columns-width=auto]
        6 \\
        3 \\
    \end{bNiceMatrix}\right) =
    \begin{bNiceMatrix}[columns-width=auto]
      -11 \\
      9 \\
      -1 \\
    \end{bNiceMatrix} -
    2\begin{bNiceMatrix}[columns-width=auto]
      -3 \\
      -3 \\
      -9 \\
    \end{bNiceMatrix} =
    \begin{bNiceMatrix}[columns-width=auto]
      -5 \\
      -3 \\
      17 \\
    \end{bNiceMatrix}
  .\]%
  Since $T(c\u + d\v) \ne cT(\u) + dT(\v)$, $T$ is not linear.
\end{solution}

\begin{question}
  \label{qst:linear_transformation_3}

  Consider the following vectors
  \[%
    \u = \begin{bNiceMatrix}[columns-width=auto]
      -4 \\
      5 \\
    \end{bNiceMatrix},\quad
    \x = \begin{bNiceMatrix}[columns-width=auto]
      40 \\
      -24 \\
    \end{bNiceMatrix},\quad
    \v = \begin{bNiceMatrix}[columns-width=auto]
      -5 \\
      -2 \\
    \end{bNiceMatrix}\aand
    \y = \begin{bNiceMatrix}[columns-width=auto]
      17 \\
      36 \\
    \end{bNiceMatrix}
  .\]%
  Let $T : \R^2 \to \R^2$ be a linear transformation such that $T(\u) = \x$ and
  $T(\v) = \y$. Use the fact that $T$ is linear to determine the following
  vectors.

  \begin{multicols}{3}
    \begin{enumerate}
      \label{enum:linear_transformation_3_qst}

      \item $T\left(\sfrac{1}{2}\u\right)$

      \item $T(\u + \v)$.

      \item $T(2\u-3\v)$.
    \end{enumerate}
  \end{multicols}
\end{question}

\begin{solution}
  \label{sol:linear_transformation_3} $ $

  \begin{enumerate}
    \label{enum:linear_transformation_3_sol}

    \item We can factor out the half and solve for the rest
      \[%
        T\left(\frac{1}{2}\u\right) = \frac{1}{2}T(\u) = \frac{1}{2}\x = \frac{1}{2}
        \begin{bNiceMatrix}[columns-width=auto]
          40 \\
          -24 \\
        \end{bNiceMatrix} =
        \begin{bNiceMatrix}[columns-width=auto]
          20 \\
          -12 \\
        \end{bNiceMatrix}
      .\]%

    \item We can use the fact that we can break up the addition since it's a
      linear transformation
      \[%
        T(\u + \v) = T(\u) + T(\v) = \x + \y =
        \begin{bNiceMatrix}[columns-width=auto]
          40 \\
          -24 \\
        \end{bNiceMatrix} +
        \begin{bNiceMatrix}[columns-width=auto]
          17 \\
          36 \\
        \end{bNiceMatrix} =
        \begin{bNiceMatrix}[columns-width=auto]
          57 \\
          12 \\
        \end{bNiceMatrix}
      .\]%

    \item Since this is a linear transformation, we can use the fact that
      \[%
        T(a\u + b\v) = aT(\u) + bT(\v)
      ,\]%
      to get the following solution
      \[%
        T(2\u - 3\v) = 2T(\u) - 3T(\v) = 2\x - 3\y =
        2\begin{bNiceMatrix}[columns-width=auto]
          40 \\
          -24 \\
        \end{bNiceMatrix} -
        3\begin{bNiceMatrix}[columns-width=auto]
          17 \\
          36 \\
        \end{bNiceMatrix} =
        \begin{bNiceMatrix}[columns-width=auto]
          29 \\
          -156 \\
        \end{bNiceMatrix}
      .\qedhere\]%
  \end{enumerate}
\end{solution}

\begin{question}
  \label{qst:linear_transformation_4}

  Consider the following vectors
  \[%
    \e_1 =
    \begin{bNiceMatrix}[columns-width=auto]
      1 \\
      0 \\
    \end{bNiceMatrix},\quad
    \e_2 =
    \begin{bNiceMatrix}[columns-width=auto]
      0 \\
      1 \\
    \end{bNiceMatrix},\quad
    \y_1 =
    \begin{bNiceMatrix}[columns-width=auto]
      8 \\
      9 \\
    \end{bNiceMatrix},\quad
    \y_2 =
    \begin{bNiceMatrix}[columns-width=auto]
      0 \\
      -1 \\
    \end{bNiceMatrix},\quad
    \v =
    \begin{bNiceMatrix}[columns-width=auto]
      -5 \\
      -4 \\
    \end{bNiceMatrix},\aand
    \x =
    \begin{bNiceMatrix}[columns-width=auto]
      a \\
      b \\
    \end{bNiceMatrix}
  .\]%

  Let $T : \R^2 \to \R^2$ be a linear transformation such that $T(\e_1) = \y_1$
  and $T(\e_2) = \y_2$. Find the images of $\v$ and $\x$ under $T$.
\end{question}

\begin{solution}
  \label{sol:linear_transformation_4}

  First, we need to rewrite $\v$ and $\x$ in terms of $\c_1$ and $\c_2$
  \begin{alignat*}{6}
    \v &=
    \begin{bNiceMatrix}[columns-width=auto]
      -5 \\
      -4 \\
    \end{bNiceMatrix}
    &&= \begin{bNiceMatrix}[columns-width=auto]
      -5 \\
      0 \\
    \end{bNiceMatrix} +
    \begin{bNiceMatrix}[columns-width=auto]
      0 \\
      -4 \\
    \end{bNiceMatrix}
    &&= -5\begin{bNiceMatrix}[columns-width=auto,last-row]
      1 \\
      0 \\
      \imp{\e_1} \\
    \end{bNiceMatrix} -
    4\begin{bNiceMatrix}[columns-width=auto,last-row]
      0 \\
      1 \\
      \imp{\e_2} \\
    \end{bNiceMatrix}
    &&= - 5\e_1 - 4\e_2 \\
    \x &= \begin{bNiceMatrix}[columns-width=auto]
      a \\
      b \\
    \end{bNiceMatrix}
    &&= \begin{bNiceMatrix}[columns-width=auto]
      a \\
      0 \\
    \end{bNiceMatrix} +
    \begin{bNiceMatrix}[columns-width=auto]
      0 \\
      b \\
    \end{bNiceMatrix}
    &&= a
    \begin{bNiceMatrix}[columns-width=auto]
      1 \\
      0 \\
    \end{bNiceMatrix} +
    b\begin{bNiceMatrix}[columns-width=auto]
      0 \\
      1 \\
    \end{bNiceMatrix}
    &&= a\e_1 + b\e_2
  .\end{alignat*}

  Thus, since $T$ is linear, we get
  \begin{alignat*}{5}
    T(\v) &= T(-5\e_1 -4\e_2) &&= -5 \cdot \underbrace{T(\e_1)}_{\y_1} - 4 \cdot \underbrace{T(\e_1)}_{\y_2} &&= -5\y_1 - 4\y_2 &&= -5
    \begin{bNiceMatrix}[columns-width=auto]
      8 \\
      9 \\
    \end{bNiceMatrix} -
    4\begin{bNiceMatrix}[columns-width=auto]
      0 \\
      -1 \\
    \end{bNiceMatrix} &&=
    \begin{bNiceMatrix}[columns-width=auto]
      -40 \\
      -41 \\
    \end{bNiceMatrix} \\
    T(\x) &= T(a\e_1 + b\e_2) &&= a \cdot T(\c_1) + b \cdot T(\e_2) &&= a\y_1 + b\y_2 &&= a
    \begin{bNiceMatrix}[columns-width=auto]
      8 \\
      9 \\
    \end{bNiceMatrix} +
    b\begin{bNiceMatrix}[columns-width=auto]
      0 \\
      -1 \\
    \end{bNiceMatrix} &&=
    \begin{bNiceMatrix}[columns-width=auto]
      8a \\
      9a - b \\
    \end{bNiceMatrix}
  .\qedhere\end{alignat*}
\end{solution}

% section linear_transformations (end)

\newpage
