\nte[Sections 1.4 and 1.6]{Oct 05 2023 Thu (14:02:33)}{Matrix Equations}

If $A$ is an $m \times n$ matrix with columns $\a_1, \a_2, \dots, \a_n \in \R^m$
and $\x \in \R^m$, then the matrix-vector multiplication $A\x$ is defined as:
\[%
  A\x=
  \begin{bNiceMatrix}[columns-width=auto,last-row]
    \a_1 & \a_2 & \cdots & \a_n \\
         & \imp{A} & & \\
  \end{bNiceMatrix}
  \begin{bNiceMatrix}[columns-width=auto,last-row]
    x_1 \\
    x_2 \\
    \vdots \\
    x_n \\
    \imp{\x} \\
  \end{bNiceMatrix} = \sum_{i=1}^n x_i \a_i
.\]%

\begin{note}
  \label{nte:defined_matrix_vector_multiplication}

  The matrix-vector multiplication $A\x$ is defined only when the number of
  columns of $A$ is equal to the number of entries in $\x$.
\end{note}

\begin{question}
  \label{qst:matrix_equations_1}

  Let
  \[%
    A = \begin{bNiceMatrix}[columns-width=auto]
      1 & 2 & -1 \\
      0 & -5 & 3 \\
    \end{bNiceMatrix} \aand
    x = \begin{bNiceMatrix}[columns-width=auto]
      4 \\
      3 \\
      7 \\
    \end{bNiceMatrix}
  .\]%

  Find $A\x$.
\end{question}

\begin{solution}
  \label{sol:matrix_equations_1}

  \begin{align*}
    A\x &= \begin{bNiceMatrix}[columns-width=auto]
      1 & 2 & -1 \\
      0 & -5 & 3 \\
    \end{bNiceMatrix}
    \begin{bNiceMatrix}[columns-width=auto]
      4 \\
      3 \\
      7 \\
    \end{bNiceMatrix} =
    4\begin{bNiceMatrix}[columns-width=auto]
      1 \\
      0 \\
    \end{bNiceMatrix} +
    3\begin{bNiceMatrix}[columns-width=auto]
      2 \\
      -5 \\
    \end{bNiceMatrix} +
    7\begin{bNiceMatrix}[columns-width=auto]
      -1 \\
      3 \\
    \end{bNiceMatrix} =
    \begin{bNiceMatrix}[columns-width=auto]
      4 \\
      0 \\
    \end{bNiceMatrix} +
    \begin{bNiceMatrix}[columns-width=auto]
      6 \\
      -15 \\
    \end{bNiceMatrix} +
    \begin{bNiceMatrix}[columns-width=auto]
      -7 \\
      21 \\
    \end{bNiceMatrix} =
    \begin{bNiceMatrix}[columns-width=auto]
      3 \\
      6 \\
    \end{bNiceMatrix}
  .\qedhere\end{align*}
\end{solution}

\begin{question}
  \label{qst:matrix_equations_2}

  For $\v_1, \v_2, \v_3 \in \R^m$, write the linear combination $3\v_1 - 5\v_2 +
  7\v_3$
\end{question}

\begin{solution}
  \label{sol:matrix_equations_2}

  Place $\v_1, \v_2, \v_3$ into the columns of a matrix $A$ and place the
  weights $3$, $-5$, and $7$ into a vector $\x$
  \[%
    3\v_1 - 5\v_2 + 7\v_3 =
    \begin{bNiceMatrix}[columns-width=auto]
      \v_1 & \v_2 & \v_3 \\
    \end{bNiceMatrix}
    \begin{bNiceMatrix}[columns-width=auto]
      3 \\
      -5 \\
      7 \\
    \end{bNiceMatrix} = A\x
  .\qedhere\]%
\end{solution}

\begin{definition}[Identity Matrix]
  \label{def:identity_matrix}

  The \textbf{identity matrix} $I_n$ is an $n \times n$ matrix with all $1$'s in
  its diagonal positions and $0$'s everywhere else. In general, $I_n$ takes the
  shape of
  \[%
    \begin{bNiceMatrix}[columns-width=auto]
      1 & 0 & \dots & 0 \\
      0 & 1 & \dots & 0 \\
      \vdots & \vdots & \ddots & \vdots \\
      0 & 0 & \dots & 1 \\
    \end{bNiceMatrix}
  .\]%
\end{definition}

\begin{note}
  \label{nte:identity_matrix}

  We know that $\forall \x \in \R^n$, $I_n\x = x$.
\end{note}

\begin{theorem}
  \label{thm:properties_of_matrix_vector_multiplication}

  If $A$ is an $m \times n$ matrix, $\u, \v \in \R^n$, and $c$ is a scaler, then
  \begin{enumerate}
    \label{enum:properties_of_matrix_vector_multiplication}

    \item $A\left(\u + \v\right) = A\u + A\v$.
    \item $A\left(c\u\right) = c\left(A\u\right)$.
  \end{enumerate}
\end{theorem}

\begin{proof}
  \label{prf:properties_of_matrix_vector_multiplication}

  Take $n$ to be the number of columns in $A$ and $\u, \v \in \R^n$. Let $u_i$
  and $v_i$ be the $i$th entries in $\u$ and $\v$, respectively. To prove
  statement $1$, compute $A(\u + \v)$ as a linear combination of the columns of
  $A$ using the entries in $\u + \v$ as weights, giving us
  \begin{align*}
    A(\u + \v) &=
    \begin{bNiceMatrix}[columns-width=auto]
      \a_1 & \dots & \a_n \\
    \end{bNiceMatrix}
    \begin{bNiceMatrix}[columns-width=auto]
      u_1 + v_1 \\
      \vdots \\
      u_n + v_n \\
    \end{bNiceMatrix} \\
    &= \sum_{i=1}^{n} (u_i + v_i)\a_i \\
    &= \sum_{i=1}^{n} u_i\a_i + \sum_{i=1}^{n} v_i\a_i \\
    &= A\u + A\v
  .\end{align*}
  To prove statement, compute $A(c\u)$ as linear combination of the columns of
  $A$ using the entries in $c\u$ as weights, giving us
  \begin{align*}
    A(c\u) &=
    \begin{bNiceMatrix}[columns-width=auto]
      \a_1 & \dots & \a_n \\
    \end{bNiceMatrix}
    \begin{bNiceMatrix}[columns-width=auto]
      cu_1 \\
      \vdots \\
      cu_n \\
    \end{bNiceMatrix} \\
    &= \sum_{i=1}^{n} (cu_i)\a_i \\
    &= \sum_{i=1}^{n} c(u_i\a_i) \\
    &= c\left(\sum_{i=1}^{n} u_i\a_i\right) \\
    &= c(A\u)
  .\qedhere\end{align*}
\end{proof}

\begin{theorem}
  \label{thm:augmented_matrix}

  If $A$ is an $m \times n$ matrix, with columns $\a_1, \a_2, \dots, \a_n$ and
  if $\b \in \R^m$, then the matrix equation
  \[%
    A\x = \b
  ,\]%
  has the same solution set as the vector equation
  \[%
    \sum_{i=1}^{n} x_i\a_i = \b
  ,\]%
  which, in turn, has the same solution set as the system of linear equations
  whose augmented matrix is
  \[%
    \begin{bNiceArray}{cccc|c}
      \a_1 & \a_2 & \cdots & \a_n & \b \\
    \end{bNiceArray}
  .\]%
\end{theorem}

A recurring problem for this course is the solving of the \textbf{matrix
equation} $A\x = \b$, where matrix $A$ and $\b$ are given. If $A =
[\,\a_1~\a_2~\cdots~\a_n\,]$ and $\x = (x_1, x_2, \dots, x_n)$, then we know
that
\[%
  A\x = \sum_{i=1}^n x_i\a_i
,\]%
which must be equivalent to the vector equation
\[%
  \sum_{i=1}^n x_i\a_i = \b
.\]%

\begin{example}
  \label{exm:augmented_matrix_equals_vector_equation_equals_matrix_equation}

  Here's an example of a linear system with its three equivalent representations
  \[%
    \sysdelim..\systeme{
      \-8x_1 + x_2 = 2,
      4x_1 + 2x_2 = \-5
    } \longleftrightarrow
    \begin{bNiceArray}{cc|c}[columns-width=auto]
      -8 & 1 & 2 \\
      4 & 2 & -5 \\
    \end{bNiceArray}
    \longleftrightarrow
    x_1\begin{bNiceMatrix}[columns-width=auto]
      -8 \\
      4 \\
    \end{bNiceMatrix}
    +
    x_2\begin{bNiceMatrix}[columns-width=auto]
      1 \\
      2 \\
    \end{bNiceMatrix} =
    \begin{bNiceMatrix}[columns-width=auto]
      2 \\
      -5 \\
    \end{bNiceMatrix}
    \longleftrightarrow
    \begin{bNiceMatrix}[columns-width=auto]
      -8 & 1 \\
      4 & 2 \\
    \end{bNiceMatrix}
    \begin{bNiceMatrix}[columns-width=auto]
      x_1 \\
      x_2 \\
    \end{bNiceMatrix} =
    \begin{bNiceMatrix}[columns-width=auto]
      2 \\
      -5 \\
    \end{bNiceMatrix}
  .\]%
\end{example}

\begin{purpleframe}
  \label{prpl:matrix_equation_solutions}

  The matrix equation $A\x = \b$ has a solution if and only if $\b$ is a linear
  combination of the columns of $A$, which, in turn only happens when the linear
  system whose augmented matrix is $[\,\a_1~\a_2~\cdots~\a_n~\vert~\b\,]$ is
  consistent.
\end{purpleframe}

\section{Vector in Span}
\label{sec:vector_in_span}

Let's try and answer the question

\begin{center}
  $\forall \b \in \R^n$, is $\b \in$ $\Sspan\left\{\v_1, \v_2, \dots,
  \v_n\right\}$?
\end{center}

\begin{definition}[Set of Vectors]
  \label{def:set_of_vectors}

  A \textbf{set of vectors} $\left\{\v_1, \v_2, \dots, \v_p\right\} \in \R^m$
  spans $\R^m$ if every vector in $\R^m$ is a linear combination of $\v_1, \v_2,
  \dots, \v_p$. If this set of vectors satisfies this, then we write
  $\Sspan\left\{\v_1, \v_2, \dots, \v_p\right\} = \R^m$.
\end{definition}

\begin{theorem}
  \label{thm:vector_in_span}

  Let $A$ be an $m \times n$ matrix. Then the following statements are logically
  equivalent. That is, if one statement is true, then all are true. Also, if one
  statement is false, then all are false.
  \begin{enumerate}
    \label{enum:vector_in_span}

    \item $\forall \b \in \R^m$, the equation $A\x = \b$ has a
      solution.

    \item $\forall \b \in \R^m$ is a linear combination of the columns
      of $A$.

    \item The columns of $A$ span $\R^m$.

    \item $A$ has a pivot position in every row.

      \begin{note}
        \label{nte:vector_in_span_4}

        $A$ is the coefficient matrix of a linear system, not the augmented
        matrix.
      \end{note}
  \end{enumerate}
\end{theorem}

\begin{proof}
  \label{prf:vector_in_span}

  Statements $1$, $2$, and $3$ are logically equivalent. So, it suffices to show
  (for an arbitrary matrix $A$) that $1$ and $d$ are either both true or both
  false. This will tie all four statements together.

  Let $U$ be an echelon form of $A$. Given $\b \in  \R^m$, we can row reduce the
  augmented matrix $[\,A~\vert~\b\,]$ to an augmented matrix $[\,U~\vert~\d\,]$,
  for some $\d \in \R^m$. If statement $4$ is true, then each row of $U$
  contains a pivot position and there can be no pivot in the augmented column.
  So $A\x = \b$ has a solution for any $\b$, and $1$ is true. If $4$ is false,
  the last row of $U$ is all zeros. Let $\d$ be any vector with a $1$ in its
  entry. Then, $[\,U~\vert~\d\,]$ represents an inconsistent system. Since row
  operations are reversible, $[\,U~\vert~\d\,]$ can be transformed into the form
  $[\,A~\vert~\b\,]$. The new system $A\x = \b$ is also inconsistent and $1$ is
  false.
\end{proof}

\begin{example}
  \label{exm:vector_in_span}

  A basic example would be $\Sspan\left\{\v_1, \v_2\right\} = \R^2$, $\forall
  \v_1, \v_2 \in \R^2$, where $\v_1 \ne \v_2$
  \begin{figure}[H]
    \centering
    \incfig{vector-span}
    \caption{Vector Span}
    \label{fig:vector_span}
  \end{figure}
\end{example}

\begin{note}
  \label{nte:vector_in_span}

  To check statement $4$ in \cref{thm:vector_in_span}, all you need to do is row
  reduce the matrix $A$ into echelon form form not all the way to reduced
  echelon form, since you only need to check if each row has a pivot.
\end{note}

% section vector_in_span (end)

\section{Applications}
\label{sec:applications}

\subsection{Equilibrium Prices}
\label{sub_sec:equilibrium_prices}

In an economy divided into sectors, the total dollar value of a sector’s output
output is called its \textbf{price}. A sector's \textbf{equilibrium price} is
the price (or output) of a sector that exactly balances its expenses.

\begin{question}
  \label{qst:equilibrium_prices}

  Consider an economy with three sectors: Fuel and Power (F), Manufacturing (M),
  and Services (S). The output data of each sector is described below.
  \begin{enumerate}
    \label{enum:equilibrium_prices_qst}

    \item Sector F sells $75\%$ of its output to Sector M, $10\%$ to Sector S,
      and retains the rest.

    \item  Sector M sells $10\%$ of its output to Sector F, $70\%$ to Sector S,
      and retains the rest.

    \item  Sector S sells $20\%$ of its output to Sector F, $50\%$ to Sector M,
      and retains the rest.
  \end{enumerate}

  The total value of the economy is $\$1,000,000,000$ (1 billion). Find the
  equilibrium price for each sector. \textit{You may use the row reducing
  function on your calculator.}
\end{question}

\begin{solution}
  \label{sol:equilibrium_prices}

  We need to first construct a table to show the information for each sector.

  \begin{figure}[H]
    \centering

    \begin{tabular}{ccc|c}
      F & M & S & Purchased By: \\
      \hline
      $0.15$ & $0.10$ & $0.20$ & F \\
      $0.75$ & $0.20$ & $0.50$ & M \\
      $0.10$ & $0.70$ & $0.30$ & S \\
    \end{tabular}

    \caption{Equilibrium Prices}
    \label{fig:equilibrium_prices}
  \end{figure}

  Let $P_F$, $P_M$, and $P_S$ be the equilibrium prices for each sector.
  \[%
    (\textrm{expenses of Sector}) = (\textrm{output of Sector})
  .\]%

  Solving for all sectors gives us the following
  \[%
    \sysdelim..\systeme{
      0.15P_F + 0.10P_M + 0.20P_S = P_F,
      0.75P_F + 0.20P_M + 0.50P_S = P_M,
      0.10P_F + 0.70P_M + 0.30P_S = P_S
    } \implies
    \sysdelim..\systeme*{
      \-0.85P_F + 0.10P_M + 0.20P_S = 0,
      0.75P_F \- 0.80P_M + 0.50P_S = 0,
      0.10P_F + 0.70P_M \- 0.70P_S = 0,
      P_F + P_M + P_S = 1
    }
  .\]%

  We can then convert that system of linear equations to an augmented matrix and
  solve using the Row Reduction Algorithm
  \[%
    \begin{bNiceArray}{ccc|c}[columns-width=auto]
      -0.85 & 0.10 & 0.20 & 0 \\
      -0.75 & -0.80 & 0.50 & 0 \\
      0.10 & 0.70 & -0.70 & 0 \\
      1 & 1 & 1 & 1 \\
    \end{bNiceArray}
    \rref
    \begin{bNiceArray}{ccc|c}[columns-width=auto]
      \circled{1} & 0 & 0 & 0.151 \\
      0 & \circled{1} & 0 & 0.414 \\
      0 & 0 & \circled{1} & 0.435 \\
      0 & 0 & 0 & 0 \\
    \end{bNiceArray} \rightarrow
    \sysdelim..\systeme*{
      P_F \approx 0.151~\textrm{billion},
      P_M \approx 0.414~\textrm{billion},
      P_S \approx 0.435~\textrm{billion}
   }
  .\qedhere\]%
\end{solution}

% subsection equilibrium_prices (end)

\subsection{Balancing Chemical Equations}
\label{sub_sec:balancing_chemical_equations}

We can vector equations to solve unbalanced chemical equations.

\begin{question}
  \label{qst:balancing_chemical_equations}

  Balance this chemical equation by solving the appropriate vector equation.
  The solution must be in integers and in lowest possible terms. \textit{You
  may use the row reducing function on your calculator.}

  \[%
    \textrm{C}_{3}\textrm{H}_{8} + \textrm{O}_{2} \rightarrow \textrm{CO}_{2} + \textrm{H}_{2}\textrm{O}
  .\]%
\end{question}

\begin{solution}
  \label{sol:balancing_chemical_equations}

  The first step is to always write out the each molecule as a vector
  \[%
    \textrm{C}_{3}\textrm{H}_{8}: \, \begin{bNiceMatrix}[columns-width=auto]
      3 \\
      8 \\
      0 \\
    \end{bNiceMatrix}\quad
    \textrm{O}_{2}: \, \begin{bNiceMatrix}[columns-width=auto]
      0 \\
      0 \\
      2 \\
    \end{bNiceMatrix}\quad
    \textrm{C}\textrm{O}_{2}: \, \begin{bNiceMatrix}[columns-width=auto]
      1 \\
      0 \\
      2 \\
    \end{bNiceMatrix}\quad
    \textrm{H}_{2}\textrm{O}: \, \begin{bNiceMatrix}[columns-width=auto]
      0 \\
      2 \\
      1 \\
    \end{bNiceMatrix}\quad
  .\]%

  We can then use those vectors to write out a vector equation in standard form
  \begin{align*}
    &x_1\begin{bNiceMatrix}[columns-width=auto]
      3 \\
      8 \\
      0 \\
    \end{bNiceMatrix} +
    x_2\begin{bNiceMatrix}[columns-width=auto]
      0 \\
      0 \\
      2 \\
    \end{bNiceMatrix}
    = x_3\begin{bNiceMatrix}[columns-width=auto]
      1 \\
      0 \\
      2 \\
    \end{bNiceMatrix} +
    x_4\begin{bNiceMatrix}[columns-width=auto]
      0 \\
      2 \\
      1 \\
    \end{bNiceMatrix} \\
    &x_1\begin{bNiceMatrix}[columns-width=auto]
      3 \\
      8 \\
      0 \\
    \end{bNiceMatrix} +
    x_2\begin{bNiceMatrix}[columns-width=auto]
      0 \\
      0 \\
      2 \\
    \end{bNiceMatrix} +
    x_3\begin{bNiceMatrix}[columns-width=auto]
      -1 \\
      0 \\
      -2 \\
    \end{bNiceMatrix} +
    x_4\begin{bNiceMatrix}[columns-width=auto]
      0 \\
      -2 \\
      -1 \\
    \end{bNiceMatrix} =
    \begin{bNiceMatrix}[columns-width=auto]
      0 \\
      0 \\
      0 \\
    \end{bNiceMatrix}
  .\end{align*}

  We then convert it to an augmented matrix and solve it using the Row Reduction
  Algorithm
  \[%
    \begin{bNiceArray}{cccc|c}[columns-width=auto]
      3 & 0 & -1 & 0 & 0 \\
      8 & 0 & 0 & -2 & 0 \\
      0 & 2 & -2 & -1 & 0 \\
    \end{bNiceArray}
    \rref
    \begin{bNiceArray}{cccc|c}[columns-width=auto]
      \circled{1} & 0 & 0 & -\sfrac{1}{4} & 0 \\
      0 & \circled{1} & 0 & -\sfrac{5}{4} & 0 \\
      0 & 0 & \circled{1} & -\sfrac{3}{4} & 0 \\
    \end{bNiceArray}
  .\]%

  This gives us the following general solution
  \[%
    \sysdelim..\systeme*{
      x_1 = \sfrac{1}{4}x_4,
      x_2 = \sfrac{5}{4}x_4,
      x_3 = \sfrac{3}{4}x_4,
      x_4 = x_4
    }
  .\]%

  We need to pick an $x_4$ such that all coefficients are integers that are
  coprime. That would be $x_4 = 4$. This gives us our final solution of
  \[%
    \sysdelim..\systeme*{
      x_1 = 1,
      x_2 = 5,
      x_3 = 3,
      x_4 = 4
    } \implies
    1\textrm{C}_{3}\textrm{H}_{8} + 5\textrm{O}_{2} \rightarrow 3\textrm{CO}_{2} + 4\textrm{H}_{2}\textrm{O}
  .\qedhere\]%
\end{solution}

% subsection balancing_chemical_equations (end)

% section applications (end)

\newpage
