\nte[Section 1.3]{Oct 03 2023 Tue (14:00:24)}{Vectors and Vector Equations}

\section{Vectors and Vector Equations}
\label{sec:vectors_and_vector_equations}

\begin{definition}[Vector]
  \label{def:vector}

  A \textbf{vector} is a matrix with one column. The set of all vectors with $n$
  real number entries (or rows) is denoted as $R^n$.
\end{definition}

\begin{example}
  \label{exm:vector}

  A vector in $\R^3$ is denoted as
  \[%
    \begin{bNiceMatrix}[columns-width=auto]
      3 \\
      -2 \\
      5 \\
    \end{bNiceMatrix}
  .\]%
\end{example}

\begin{notation}
  \label{ntn:ways_to_denote_a_vector}

  To denote a variable is a vector, you can either use a boldface letter such as
  $\mathbf{u}$, or a letter with an arrow on top, e.g. $\u$. To create vectors,
  you can either use order pairs $\u = (3, 1, 2)$, or use braket notation, $\u =
  \langle 3, 1, 2 \rangle$
\end{notation}

\begin{note}
  \label{nte:ways_to_denote_a_vector}

  Throughout these notes, I'll be using the arrow notation to along with the
  parentheses notation denote vectors.
\end{note}

\subsection{Vector Operations}
\label{sub_sec:vector_operations}

Vectors in $\R^2$ can be represented as ordered pairs $(x, y)$ and thus as a
point in the plane. Typically we also draw a arrow from the origin to the point.

\begin{multicols}{2}
  \noindent Scaling a vector $\u$ ($\v$) by $c$ will be \textbf{stretched} or
  \textbf{shorten} by a factor of $c$, such as
  \begin{align*}
    \frac{1}{2}\underset{\color{linecolor2}{\u}}{
      \begin{bNiceMatrix}[columns-width=auto]
        4 \\
        2 \\
      \end{bNiceMatrix}
    }
    = \begin{bNiceMatrix}[columns-width=auto]
      2 \\
      1 \\
    \end{bNiceMatrix}
    {\color{linecolor2}{\,=\,\frac{1}{2}\u}} \oor
    -2\underset{\color{linecolor1}{\v}}{
      \begin{bNiceMatrix}[columns-width=auto]
        -1 \\
        1 \\
      \end{bNiceMatrix}
    }
    = \begin{bNiceMatrix}[columns-width=auto]
      -2 \\
      2 \\
    \end{bNiceMatrix}
    {\textcolor{linecolor1}{\,=\,-2\v}}
  \end{align*}\columnbreak
  \begin{center}
    \begin{tikzpicture}
      \begin{axis}[
        axis equal image,
        xmin=-5.25, xmax=5.25,
        ymin=-3.25, ymax=3.25,
        ]

        \addplot[color=linecolor2,->,thick,mark=none] coordinates{(0,0)(4,2)} node[above]{$\u$};
        \addplot[color=linecolor2,->,thick,mark=none] coordinates{(0,0)(2,1)} node[above]{$\frac{1}{2}\u$};
        \addplot[color=linecolor1,->,thick,mark=none] coordinates{(0,0)(-1,1)} node[above]{$\v$};
        \addplot[color=linecolor1,->,thick,mark=none] coordinates{(0,0)(2,-2)} node[below]{$-2\v$};
      \end{axis}
    \end{tikzpicture}
  \end{center}
\end{multicols}
\begin{multicols}{2}
  \noindent The sum $\u + \v$ corresponds to the point obtained by drawing the
  vector $\u$, then drawing the vector $\v$ from the end of $\u$ as if it were
  the origin, for example:
  \begin{equation*}
    \underset{\color{linecolor2}{\u}}{
      \begin{bNiceMatrix}[columns-width=auto]
        2 \\
        4 \\
      \end{bNiceMatrix}
    } +
    \underset{\color{linecolor1}{\v}}{
      \begin{bNiceMatrix}[columns-width=auto]
        5 \\
        1 \\
      \end{bNiceMatrix}
    }
    = \underset{\u + \v}{
      \begin{bNiceMatrix}[columns-width=auto]
        7 \\
        5 \\
      \end{bNiceMatrix}
    }
  \end{equation*}\columnbreak
  \begin{center}
    \begin{tikzpicture}
      \begin{axis}[
        xmin=-1, xmax=10.25,
        ymin=-1, ymax=6.25,
        ]

        \addplot[color=linecolor2,->,thick,mark=none] coordinates{(0,0)(2,4)} node[above left]{$\u$};
        \addplot[color=linecolor1,->,thick,mark=none] coordinates{(2,4)(7,5)} node[above left]{$\v$};
        \addplot[color=black,->,thick,mark=none] coordinates{(0,0)(7,5)} node[above right]{$\u + \v$};
        \addplot[color=linecolor1!50,->,thick,mark=none] coordinates{(0,0)(5,1)} node[below right]{$\v$};
        \addplot[color=linecolor2!50,->,thick,mark=none] coordinates{(5,1)(7,5)} node[below right]{$\u$};
      \end{axis}
    \end{tikzpicture}
  \end{center}
\end{multicols}

\begin{purpleframe}[Algebraic Properties of $R^n$]
  \label{prpl:algebraic_properties_of_rn}

  $\forall \u, \v, \w \in \R^n$ and $\forall c, d \in \R$, where $c$ and $d$ are
  scalers, we get the following properties.

  \begin{multicols}{2}\noindent
    \begin{enumerate}
      \label{enum:algebraic_properties_of_rn}

      \item $\u + \v = \v + \u$.
      \item $(\u + \v) + \w = \u + (\v + \w)$.
      \item $\u + \zero = \u$.
      \item $\u + (-\u) = \u - \u = \zero$.

      \item $c(\u + \v) = c\u + c\v$.
      \item $(c + d)\u = c\u + d\u$.
      \item $c(d\u) = (cd)\u$.
      \item $1\u = \u$
    \end{enumerate}
  \end{multicols}
\end{purpleframe}

\begin{note}
  \label{nte:the_zero_vector}

  The $\zero$ vector is $(0, \dots, 0) \in \R^n$.
\end{note}

\begin{note}
  \label{nte:vector_equals}

  Two vectors are equal if and only if their corresponding entries are equal.
\end{note}

% subsection vector_operations (end)

\subsection{Linear Combinations}
\label{sub_sec:linear_combinations}

\begin{definition}[Linear Combination]
  \label{def:linear_combination}

  Given vectors $\v_1, \v_2 \dots, \v_p \in \R$ and scalers $c_1, c_2, \dots,
  c_p \in \R$, the vector $\y$ is defined as
  \[%
    \y = \sum_{i=1}^p c_i\v_i
  ,\]%
  is called the \textbf{linear combination} of $\v_1 \v_2, \dots, \v_p$ with
  weights $c_1, c_2, \dots, c_p$.
\end{definition}

\begin{question}
  \label{qst:linear_combination}

  Consider the following vectors
  \[%
    \a_1 = \begin{bNiceMatrix}[columns-width=auto]
      1 \\
      -2 \\
      -5 \\
    \end{bNiceMatrix},\quad
    \a_2 = \begin{bNiceMatrix}[columns-width=auto]
      2 \\
      5 \\
      6 \\
    \end{bNiceMatrix},\aand
    \b = \begin{bNiceMatrix}[columns-width=auto]
      7 \\
      4 \\
      -3 \\
    \end{bNiceMatrix}
  .\]%

  Use the Row Reduction Algorithm to determine if $\b$ is a linear combination
  of $\a_1$ and $\a_2$.
\end{question}

\begin{solution}
  \label{sol:linear_combination}

  Let $x_1, x_2 \in \R$ be the scalers. We need to find $x_1$ and $x_2$ such
  that $x_1\a_1 + x_2\a_2 = \b$. This can be rewritten as
  \[%
    x_1
    \begin{bNiceMatrix}[columns-width=auto]
      1 \\
      -2 \\
      -5 \\
    \end{bNiceMatrix} +
    x_2
    \begin{bNiceMatrix}[columns-width=auto]
      2 \\
      5 \\
      6 \\
    \end{bNiceMatrix}
    = \begin{bNiceMatrix}[columns-width=auto]
      7 \\
      4 \\
      -3 \\
    \end{bNiceMatrix}
    \longleftrightarrow
    \begin{bNiceMatrix}[columns-width=auto]
      x_1 \\
      -2x_1 \\
      -5x_1 \\
    \end{bNiceMatrix}
    +
    \begin{bNiceMatrix}[columns-width=auto]
      2x_2 \\
      5x_2 \\
      6x_2 \\
    \end{bNiceMatrix}
    = \begin{bNiceMatrix}[columns-width=auto]
      7 \\
      4 \\
      -3 \\
    \end{bNiceMatrix}
    \longleftrightarrow
    \begin{bNiceMatrix}[columns-width=auto]
      x_1 + 2x_2 \\
      -2x_1 + 5x_2 \\
      -5x_1 + 6x_2 \\
    \end{bNiceMatrix}
    = \begin{bNiceMatrix}[columns-width=auto]
      7 \\
      4 \\
      -3 \\
    \end{bNiceMatrix}
  ,\]%
  which can be written as the following system of equations
  \[%
    \sysdelim..\systeme{
      x_1 + 2x_2 = 7,
      -2x_1 + 5x_2 = 4,
      -5x_1 + 6x_2 = -3
    } \longleftrightarrow
    \begin{bNiceArray}{cc|c}[columns-width=auto]
      1 & 2 & 7 \\
      -2 & 5 & 4 \\
      -5 & 6 & -3 \\
    \end{bNiceArray}
  .\]%

  Use the Row Reduction Algorithm to solve the system of equations.
  \begin{align*}
    \sysdelim..\systeme{
      2R_1 + R_2 \rightarrow R_2,
      5R_1 + R_3 \rightarrow R_3
    } &\longleftrightarrow
    \begin{bNiceArray}{cc|c}[columns-width=auto]
      \circled{1} & 2 & 7 \\
      0 & \circled{9} & 18 \\
      0 & 16 & 32 \\
    \end{bNiceArray} \\
    \sysdelim..\systeme{
      \frac{1}{2}R_2 \rightarrow R_2
    } &\longleftrightarrow
    \begin{bNiceArray}{cc|c}[columns-width=auto]
      \circled{1} & 2 & 7 \\
      0 & \circled{1} & 2 \\
      0 & 16 & 32 \\
    \end{bNiceArray} \\
    \sysdelim..\systeme{
      -16R_2 + R_3 \rightarrow R_3
    } &\longleftrightarrow
    \begin{bNiceArray}{cc|c}[columns-width=auto]
      \circled{1} & 2 & 7 \\
      0 & \circled{1} & 2 \\
      0 & 0 & 0 \\
    \end{bNiceArray} \\
    \sysdelim..\systeme{
      -2R_2 + R_1 \rightarrow R_1
    } &\longleftrightarrow
    \begin{bNiceArray}{cc|c}[columns-width=auto]
      \circled{1} & 0 & 3 \\
      0 & \circled{1} & 2 \\
      0 & 0 & 0 \\
    \end{bNiceArray}
  .\end{align*}

  That means $x_1 = 3$ and $x_2 = 2$. Thus $\b$ is a linear combination of
  $\a_1$ and $\a_2$, which makes $\b$ a linear combination of
  \[%
    3\a_1 + 2\a_2 = \b
  .\qedhere\]%
\end{solution}

\begin{purpleframe}
  \label{prpl:linear_combinations}

  The vector equation $x_1\a_1 + x_2\a_2 + \dots + x_n\a_n = \b$ has the same
  solution set as the linear system whose augmented matrix is
  \[%
    \begin{bNiceMatrix}[columns-width=auto]
      \color{gray}{|} & \color{gray}{|} &  & \color{gray}{|} & \color{gray}{|} \\
      \a_1 & \a_2 & \cdots & \a_n & \b \\
      \color{gray}{|} & \color{gray}{|} &  & \color{gray}{|} & \color{gray}{|}
    \end{bNiceMatrix} \quad\text{or just}\quad [\,\a_1~\a_2~\cdots~\a_n~\b\,]
  .\]%
\end{purpleframe}

\begin{example}
  \label{exm:linear_combinations}

  The augmented matrix of the vectors in \cref{qst:linear_combination} is
  \[%
    \begin{bNiceArray}{cc|c}[columns-width=auto]
      1 & 2 & 7 \\
      -2 & 5 & 4 \\
      -5 & 6 & -3 \\
    \end{bNiceArray}
  .\]%
\end{example}

\begin{definition}[Span]
  \label{def:span}

  Let $\v_1, \v_2, \dots, \v_p \in R^n$. The \textbf{span} of the vectors $\v_1,
  \v_2, \dots, \v_p$, denoted as $\Sspan\{\v_1$, $\v_2$, $\dots$, $\v_p\}$ is
  the set of all linear combinations of $\v_1, \v_2, \dots, \v_p$. That is,
  $\Sspan\{\v_1$, $\v_2$, $\dots$, $\v_p\}$ is the collection of all the vectors
  $\b$ that can be written as
  \[%
    \b = \sum_{i=1}^p x_i\v_i
  ,\]%
  for some scalers $x_1, x_2, \dots, x_p$.
\end{definition}

\begin{question}
  \label{qst:solve_for_h_to_be_linear_combination}

  Consider the following vectors
  \[%
    \a_1 = \begin{bNiceMatrix}[columns-width=auto]
      3 \\
      -4 \\
      1 \\
    \end{bNiceMatrix}, \qquad
    \a_2 = \begin{bNiceMatrix}[columns-width=auto]
      6 \\
      1 \\
      4 \\
    \end{bNiceMatrix}, \qquad
    \b = \begin{bNiceMatrix}[columns-width=auto]
      6 \\
      -26 \\
      h \\
    \end{bNiceMatrix}
  .\]%

  For what value(s) of $h$ is $\b$ a linear combination of $\a_1$ and $\a_2$?
  Justify your answer.
\end{question}

\begin{solution}
  \label{sol:solve_for_h_to_be_linear_combination}

  The equivalent augmented matrix of $x_1\a_1 + x_2\a_2 = \b$ is
  \[%
    \begin{bNiceArray}{cc|c}[columns-width=auto]
      \a_1 & \a_2 & \b \\
    \end{bNiceArray} =
    \begin{bNiceArray}{cc|c}[columns-width=auto]
      3 & 6 & 6 \\
      -4 & 1 & -26 \\
      1 & 4 & h \\
    \end{bNiceArray}
  .\]%

  We can then solve for $h$ like we've done in
  \cref{sec:criteria_for_consistent_linear_systems}, where we would find an $h$
  value such that the matrix is consistent.

  \begin{align*}
    \sysdelim..\systeme{
      \frac{1}{3}R_1 \rightarrow R_1
    } &\longleftrightarrow
    \begin{bNiceArray}{cc|c}[columns-width=auto]
      \circled{1} & 2 & 2 \\
      -4 & 1 & -26 \\
      1 & 4 & h \\
    \end{bNiceArray} \\
    \sysdelim..\systeme{
      4R_1 + R_2 \rightarrow R_2,
      \-1R_1 + R_3 \rightarrow R_3
    } &\longleftrightarrow
    \begin{bNiceArray}{cc|c}[columns-width=auto]
      \circled{1} & 2 & 2 \\
      0 & \circled{9} & 18 \\
      0 & 2 & h - 2 \\
    \end{bNiceArray} \\
    \sysdelim..\systeme{
      \frac{1}{9}R_2 \rightarrow R_2
    } &\longleftrightarrow
    \begin{bNiceArray}{cc|c}[columns-width=auto]
      \circled{1} & 2 & 2 \\
      0 & \circled{1} & 2 \\
      0 & 2 & h - 2 \\
    \end{bNiceArray} \\
    \sysdelim..\systeme{
      -2R_2 + R_3 \rightarrow R_3
    } &\longleftrightarrow
    \begin{bNiceArray}{cc|c}[columns-width=auto]
      \circled{1} & 2 & 2 \\
      0 & \circled{1} & 2 \\
      0 & 0 & \boxed{h + 2} \\
    \end{bNiceArray}
  .\end{align*}

  In order for this to be consistent, we want to avoid $[\,0~0~\vert~b\,]$ ,
  where $b \ne 0$.

  Thus, $h + 2 = 0 \implies h = -2$. So, $h = -2$.
\end{solution}

% subsection linear_combinations (end)

% section vectors_and_vector_equations (end)

\newpage
