\nte[Sections 2.1 and 2.2]{Oct 24 2023 Tue (14:03:14)}{Matrix Operations and Inverses}

\section{Matrix Arithmetic}
\label{sec:matrix_arithmetic}

Let $A$ be a matrix with $m$ rows and $n$ columns. Recall that we say that $A$
is an $m \times n$ matrix. Here, $m \times n$ is called the \textbf{size} of the
matrix, \underline{not} the dimension. The scaler entry in the $i$th row and
the $j$th column of $A$ is denoted $a_{ij}$ and is called the $(i, j)$-entry of
$A$. For example, consider the matrix below
\[%
  A = \begin{bNiceMatrix}[columns-width=auto]
    3 & 2 & 1 \\
    4 & 0 & 9 \\
    -2 & \imp{3} & 2 \\
  \end{bNiceMatrix}
.\]%
The $\imp{3}$ in row $3$ and column $2$ is denoted as $3 = a_{32}$ and
it's called the $(3,2)$-entry of $A$.

Diagonal entries of a matrix have the form $a_{11}, a_{22}, a_{33}, \dots,
a_{nn}$. These entries are called the \textbf{main diagonal} of a matrix. In the
matrix $A$ above, the main diagonal is $a_{11} = 3$, $a_{22} = 0$, and $a_{33} =
2$.

\begin{definition}[Square Matrix]
  \label{def:square_matrix}

  A matrix with the same number of rows and columns, denoted as $n \times n$, is
  called a \textbf{square matrix}. A \textbf{diagonal matrix} is a square matrix
  whose non-main diagonal entries are all zero.
\end{definition}

Like vectors, you can add or subtract matrices entry-by-entry. You may only add
matrices of the same size, otherwise the sum is undefined.

\begin{question}
  \label{qst:matrix_operation}

  Consider the following matrices
  \[%
    A = \begin{bNiceMatrix}[columns-width=auto]
      4 & 0 & 5 \\
      -1 & 3 & 2 \\
    \end{bNiceMatrix},\quad
    B = \begin{bNiceMatrix}[columns-width=auto]
      1 & 1 & 1 \\
      3 & 5 & 7 \\
    \end{bNiceMatrix},\aand
    C = \begin{bNiceMatrix}[columns-width=auto]
      5 & -3 \\
      1 & 2 \\
      9 & 2 \\
    \end{bNiceMatrix}
  .\]%
  Compute the following matrix sums if it is defined. If not, explain why.
  \begin{multicols}{3}
    \begin{enumerate}
      \label{enum:matrix_operation_qst}

      \item $A + B$.

      \item $B - 2A$.

      \item $3A + C$.
    \end{enumerate}
  \end{multicols}
\end{question}

\begin{solution}
  \label{sol:matrix_operation}

  \begin{enumerate}
    \label{enum:matrix_operation_sol} $ $

    \item $\begin{bNiceMatrix}[columns-width=auto]
        4 & 0 & 5 \\
        -1 & 3 & 2 \\
      \end{bNiceMatrix} +
      \begin{bNiceMatrix}[columns-width=auto]
        1 & 1 & 1 \\
        3 & 5 & 7 \\
      \end{bNiceMatrix} =
      \begin{bNiceMatrix}[columns-width=auto]
        5 & 1 & 6 \\
        2 & 8 & 9 \\
      \end{bNiceMatrix}$

    \item $\begin{bNiceMatrix}[columns-width=auto]
        1 & 1 & 1 \\
        3 & 5 & 7 \\
      \end{bNiceMatrix} -
      2\begin{bNiceMatrix}[columns-width=auto]
        4 & 0 & 5 \\
        -1 & 3 & 2 \\
      \end{bNiceMatrix} =
      \begin{bNiceMatrix}[columns-width=auto]
        1 & 1 & 1 \\
        3 & 5 & 7 \\
      \end{bNiceMatrix} +
      \begin{bNiceMatrix}[columns-width=auto]
        -8 & 0 & -10 \\
        2 & -6 & -4 \\
      \end{bNiceMatrix} =
      \begin{bNiceMatrix}[columns-width=auto]
        -7 & 1 & -9 \\
        5 & -1 & 3 \\
      \end{bNiceMatrix}$

    \item $3A + C$ is undefined due to the different matrix sizes. \qedhere
  \end{enumerate}
\end{solution}

\begin{theorem}
  \label{thm:matrix_arithmetic}

  Let $A$, $B$, and $C$ be matrices of the same size, and let $r$ and $s$ be
  scalars. Also, let $0$ denote the matrix of all zeros. Then, we get the
  following properties
  \begin{multicols}{3}
    \begin{enumerate}
      \label{enum:matrix_arithmetic}

      \item $A + B = B + A$
      \item $(A + B) + C = A + (B + C)$
      \item $A + 0 = A$
      \item $r(A + B) = rA + rB$
      \item $(r + s)A = rA + sA$
      \item $r(sA) = (rs)A$
    \end{enumerate}
  \end{multicols}
\end{theorem}

% section matrix_arithmetic (end)

\section{Matrix Multiplication}
\label{sec:matrix_multiplication}

Let $A$ be an $m \times n$ matrix and let $B$ be an $n \times p$ matrix. Then,
let $\x \in \R^p$. To denote how to multiply $A$ and $B$, we will first look at
the matrix-vector multiplication $B\x$. Let the columns of $B$ be $\b_1$,
$\b_2$, $\dots$, $\b_p$ and let the entries of $\x$ be $x_1$, $x_2$, $\dots$,
$x_p$. First, consider
\[%
  B\x = \begin{bNiceMatrix}[columns-width=auto]
    \b_1 & \b_2 & \cdots & \b_p
  \end{bNiceMatrix}
  \begin{bNiceMatrix}[columns-width=auto]
    x_1 \\
    x_2 \\
    \vdots \\
    x_p \\
  \end{bNiceMatrix} = x_1\b_1 + x_2\b_2 + \cdots + x_p\b_p
.\]%
Now, we will multiply this result by $A$ to get
\begin{align*}
  A(B\x) &= A(x_1\b_1 + x_2\b_2 + \cdots + x_p\b_p) \\
         &= A(x_1\b_1) + A(x_2\b_2) + \cdots + A(x_p\b_p) \\
         &= x_1A\b_1 + x_2A\b_2 + \cdots + x_pA\b_p \\
         &= [\,A\b_1~A\b_2~\cdots~A\b_p\,] \x
.\end{align*}
The matrix $[\,A\b_1~A\b_2~\cdots~A\b_p\,] \x$ is the definition of $AB$.

\begin{definition}[Column Matrix Multiplication]
  \label{def:column_matrix_multiplication}

  If $A$ is an $m \times n$ matrix and $B$ is an $n \times p$ matrix with the
  columns $\b_1$, $\b_2$, $\dots$, $\b_p$, then $AB$ is an $m \times p$ matrix
  of the form
  \[%
    AB = [\,A\b_1~A\b_2~\cdots~A\b_p\,]
  .\]%
\end{definition}

\begin{definition}[Entry-Wise Matrix Multiplication]
  \label{def:entry_wise_matrix_multiplication}

  The \textbf{dot product} between two vectors $\x = (x_1, x_2, \cdots, x_n)$
  and $\y = (y_1, y_2, \cdots, y_n)$ is defined as
  \[%
    \x \cdot \y = x_1y_1 + x_2y_2 + \cdots + x_ny_n
  .\]%
  In general, to find the $ij$th entry of $AB$, we compute
  \[%
    (AB)_{ij} = row_i(A) \cdot col_j(B)
  .\]%
\end{definition}

\begin{purpleframe}[Undefined Matrix Multiplication]
  \label{prpl:undefined_matrix_multiplication}

  The number of columns of $A$ must be equal to the number of rows of $B$.
  Otherwise, $AB$ is undefined.
\end{purpleframe}

\begin{question}
  \label{qst:column_matrix_multiplication}

  Given the matrices $A =
  \begin{bNiceMatrix}[columns-width=auto]
    2 & 3 \\
    1 & -5 \\
  \end{bNiceMatrix}$
  and $B = \begin{bNiceMatrix}[columns-width=auto]
    4 & 3 & 6 \\
    1 & -2 & 3 \\
  \end{bNiceMatrix}$, compute $AB$ using the column matrix multiplication method
  and the entry-wise matrix multiplication method.
\end{question}

\begin{solution}
  \label{sol:column_matrix_multiplication}

  $AB$ is defined since $A$ has $2$ columns and $B$ has $2$ rows. The resulting
  matrix will be a $2 \times 3$ matrix. The matrix $B$ is made up of the
  following columns
  \[%
    \b_1 = \begin{bNiceMatrix}[columns-width=auto]
      4 \\
      1 \\
    \end{bNiceMatrix},\quad
    \b_2 = \begin{bNiceMatrix}[columns-width=auto]
      3 \\
      -2 \\
    \end{bNiceMatrix},\aand
    \b_3 = \begin{bNiceMatrix}[columns-width=auto]
      6 \\
      3 \\
    \end{bNiceMatrix}
  .\]%
  Then, to compute $AB$, we multiply $A$ by each of the columns of $B$. In doing
  so, we get
  \begin{alignat*}{5}
    A\b_1 &= \begin{bNiceMatrix}[columns-width=auto]
      2 & 3 \\
      1 & -5 \\
    \end{bNiceMatrix}
    \begin{bNiceMatrix}[columns-width=auto]
      4 \\
      1 \\
    \end{bNiceMatrix} &&=
    4\begin{bNiceMatrix}[columns-width=auto]
      2 \\
      1 \\
    \end{bNiceMatrix} +
    1\begin{bNiceMatrix}[columns-width=auto]
      3 \\
      -5 \\
    \end{bNiceMatrix} &&=
    \begin{bNiceMatrix}[columns-width=auto]
      11 \\
      -1 \\
    \end{bNiceMatrix} \\
    A\b_2 &= \begin{bNiceMatrix}[columns-width=auto]
      2 & 3 \\
      1 & -5 \\
    \end{bNiceMatrix}
    \begin{bNiceMatrix}[columns-width=auto]
      3 \\
      -2 \\
    \end{bNiceMatrix} &&=
    3\begin{bNiceMatrix}[columns-width=auto]
      2 \\
      1 \\
    \end{bNiceMatrix} -
    2\begin{bNiceMatrix}[columns-width=auto]
      3 \\
      -5 \\
    \end{bNiceMatrix} &&=
    \begin{bNiceMatrix}[columns-width=auto]
      0 \\
      13 \\
    \end{bNiceMatrix} \\
    A\b_3 &= \begin{bNiceMatrix}[columns-width=auto]
      2 & 3 \\
      1 & -5 \\
    \end{bNiceMatrix}
    \begin{bNiceMatrix}[columns-width=auto]
      6 \\
      3 \\
    \end{bNiceMatrix} &&=
    6\begin{bNiceMatrix}[columns-width=auto]
      2 \\
      1 \\
    \end{bNiceMatrix} +
    3\begin{bNiceMatrix}[columns-width=auto]
      3 \\
      -5 \\
    \end{bNiceMatrix} &&=
    \begin{bNiceMatrix}[columns-width=auto]
      21 \\
      -9 \\
    \end{bNiceMatrix}
  .\end{alignat*}
  Therefore, we get
  \[%
    AB = \begin{bNiceMatrix}[columns-width=auto]
      A\b_1 & A\b_2 & A\b_3 \\
    \end{bNiceMatrix} =
    \begin{bNiceMatrix}[columns-width=auto]
      11 & 0 & 21 \\
      -1 & 13 & -9 \\
    \end{bNiceMatrix}
  .\]%

  To perform $AB$ using the entry-wise method, we get
  \begin{align*}
    \begin{bNiceMatrix}[columns-width=auto]
      2 & 3 \\
      1 & -5 \\
    \end{bNiceMatrix}
    \begin{bNiceMatrix}[columns-width=auto]
      4 & 3 & 6 \\
      1 & -2 & 3 \\
    \end{bNiceMatrix} &=
    \begin{bNiceMatrix}[columns-width=auto]
      (2)(4) + (3)(1) & (2)(3) + (3)(2) & (2)(3) + (3)(3) \\
      (1)(4) + (-5)(1) & (1)(3) + (-5)(-2) & (1)(6) + (-5)(3) \\
    \end{bNiceMatrix} \\
    &= \begin{bNiceMatrix}[columns-width=auto]
      11 & 0 & 21 \\
      -1 & 13 & -9 \\
    \end{bNiceMatrix}
  .\qedhere\end{align*}
\end{solution}

\begin{theorem}
  \label{thm:matrix_multiplication_properties}

  Let $A$ be an $m \times n$ matrix and let $B$ and $C$ be matrices where the
  following sums and products are defined. Let $r$ be any scalar.
  \begin{multicols}{3}
    \begin{enumerate}
      \label{enum:matrix_multiplication_properties}

      \item $A(BC) = (AB)C$.
      \item $A(B + C) = AB + AC$.
      \item $(B + C)A = BA + CA$.
      \item $r(AB) = (rA)B=A(rB)$.
      \item $AI_n = I_mA=A$.
    \end{enumerate}
  \end{multicols}
\end{theorem}

\begin{note}
  \label{note:matrix_multiplication}

  Here are some important facts about matrix multiplication.
  \begin{enumerate}
    \label{enum:nte_matrix_multiplication}

    \item In general, $AB \ne BA$.

    \item If $AB = AC$, it is \textbf{not} true in general that $B = C$.

    \item If $AB = 0$, it is \textbf{not} true in general that $A = 0$ or $B =
      0$. Both matrices can be nonzero.
  \end{enumerate}
\end{note}

Taking the power of a matrix looks like
\[%
  A^k = \underbrace{AAA \cdots AAA}_{k~\textrm{times}}
.\]%

\begin{definition}[Tranpose of a Matrix]
  \label{def:tranpose_of_a_matrix}

  The \textbf{transpose} of a matrix switches rows into columns.
\end{definition}

\begin{example}
  \label{exm:tranpose_of_a_matrix}

  Let $A = \begin{bNiceMatrix}[r,columns-width=auto]
    -5 & 2 \\
    1 & -3 \\
    0 & 4 \\
  \end{bNiceMatrix}$
  and
  $\x = \begin{bNiceMatrix}[r,columns-width=auto]
    3 \\
    0 \\
    -2 \\
  \end{bNiceMatrix}$. Then the transpose of each is
  \[%
    A\T = \begin{bNiceMatrix}[columns-width=auto]
			-5 & 1 & 0 \\
			2 & -3 & 4 \\
		\end{bNiceMatrix}\aand
		\x\T = \begin{bNiceMatrix}[columns-width=auto]
			3 & 0 & -2 \\
		\end{bNiceMatrix}
  .\]%
\end{example}

\begin{theorem}
  \label{thm:properties_of_transpose}

  Let $A$ and $B$ be matrices of appropriate size for the following sums and
  products. Let $r$ be any scalar.
  \begin{enumerate}
    \label{enum:properties_of_transpose}

    \item $(A\T)\T = A$.

    \item $(A + B)\T = A\T + B\T$. \imp{In general, $(A_1 + A_2 + \cdots
      + A_k)\T = A_1\T + A_2\T + \cdots + A_k\T$.}

    \item $(rA)\T = rA\T$.

    \item $(AB)\T = B\T A\T$. \imp{In general, $(A_1A_2 \cdots A_k)\T =
      A_k\T A_{k-1}\T \cdots A_2\T A_1\T$.}
  \end{enumerate}
\end{theorem}

\begin{note}
  \label{nte:ab_transpose_not_equal_to_a_transpose_b_transpose}

  In general, $(AB)\T$ is \textbf{not} equal to $A\T B\T$. Also, sometimes $A\T
  B\T$ is not defined even if $(AB)\T$ exists.
\end{note}

% section matrix_multiplication (end)

\section{The Inverse of a Matrix}
\label{sec:the_inverse_of_a_matrix}

We can find the matrix inverse of a \textbf{square matrix} $A$, denoted as
$A\I$, such that $A\I A = I$

When $A\I$ exists, we say that $A$ is \textbf{non-singular} or
\textbf{invertible}. If $A\I$ does not exist, then we say that $A$ is
\textbf{singular} or \textbf{not invertible}.

\begin{definition}[Inverse and Determinant]
  \label{def:inverse_and_determinant}

  Let $A = \begin{bNiceMatrix}[columns-width=auto]
    a & b \\
    c & d \\
  \end{bNiceMatrix}$ and let $ad - bc \ne 0$. Then, $A$ is invertible and
  \[%
    A\I = \frac{1}{ad - bc}
    \begin{bNiceMatrix}[columns-width=auto]
      d & -b \\
      -c & a \\
    \end{bNiceMatrix}
  .\]%
  If $ad - bc = 0$, then $A$ is not invertible. The quantity $ad - bc$ is called
  the \textbf{determinant} of $A$ and is denoted as $\det(A)$.
\end{definition}

\begin{note}
  \label{nte:determinate_of_a_2_by_2_matrix}

  You can only use $ad - bc$ to find the determinate of a $2 \times 2$ matrix.
\end{note}

\begin{question}
  \label{qst:inverse_of_a_matrix}

  Given the matrix $A =
  \begin{bNiceMatrix}[columns-width=auto]
    3 & 4 \\
    5 & 6 \\
  \end{bNiceMatrix}$, compute the inverse $A\I$ if it exists.
\end{question}

\begin{solution}
  \label{sol:inverse_of_a_matrix}

  Firstly, $\det(A) = (3)(6) - (4)(5) = -2 \ne 0$. Thus, $A\I$ exists.
  \[%
    A\I = \frac{1}{-2}
    \begin{bNiceMatrix}[columns-width=auto]
      3 & -4 \\
      -5 & 6 \\
    \end{bNiceMatrix} =
    \begin{bNiceMatrix}[columns-width=auto]
      -3 & 2 \\
      \sfrac{5}{2} & -3 \\
    \end{bNiceMatrix}
  .\qedhere\]%
\end{solution}

\begin{purpleframe}
  \label{prpl:alternative_way_of_solving_a_matrix_equation}

  If $A\x = \b$ for some $\b$, then multiplying both sides by $A\I$ gives
  \[%
    A\I A\x = A\I\b \implies I\x = A\I\b \implies \x = A\I\b
  ,\]%
  which shows an alternative way of solving $A\x = \b$.
\end{purpleframe}

\begin{theorem}
  \label{thm:inverse_of_a_matrix}

  If $A$ is an invertible square matrix, then $\forall \b \in \R^n$, the matrix
  equation $A\x = \b$ has a unique solution of the form $\x = A\I\b$.
\end{theorem}

\begin{question}
  \label{qst:solving_x_using_inverse}

  Let's again consider $A =
  \begin{bNiceMatrix}[r,columns-width=auto]
    3 & 4 \\
    5 & 6 \\
  \end{bNiceMatrix}$ with the vector $\b =
  \begin{bNiceMatrix}[r,columns-width=auto]
    -4 \\
    2 \\
  \end{bNiceMatrix}$. We solve $A\x = \b$ for $\x$.
\end{question}

\begin{solution}
  \label{sol:solving_x_using_inverse}

  We already know that $A\I$ exists. Thus, by \cref{thm:inverse_of_a_matrix} $\x
  = A\I\b$, which gives us
  \[%
    \x = \begin{bNiceMatrix}[columns-width=auto]
      -3 & 2 \\
      \sfrac{5}{2} & -3 \\
    \end{bNiceMatrix}
    \begin{bNiceMatrix}[columns-width=auto]
      -4 \\
      2 \\
    \end{bNiceMatrix} =
    -4\begin{bNiceMatrix}[columns-width=auto]
      -3 \\
      \sfrac{5}{2} \\
    \end{bNiceMatrix} +
    2\begin{bNiceMatrix}[columns-width=auto]
      2 \\
      -\sfrac{3}{2} \\
    \end{bNiceMatrix} =
    \begin{bNiceMatrix}[columns-width=auto]
      16 \\
      3 \\
    \end{bNiceMatrix}
  .\qedhere\]%
\end{solution}

\begin{note}
  \label{nte:not_most_efficient_method}

  Solving a matrix equation $A\x = \b$ with $\x = A\I\b$ is \textbf{not} the
  most efficient method to solving this linear system. For very large matrices
  (due to large datasets), the act of finding $A\I$ can be very computationally
  expensive. Instead, most computer software \textbf{row reduce} the augmented
  matrix $[\,A~\b\,]$. Row reduction is the \textbf{most efficient} method of
  solving $A\x = \b$.
\end{note}

\begin{theorem}
  \label{thm:properties_of_inverses}

  Let $A$ and $B$ be invertible, square matrices of appropriate size such that
  the following products are defined.
  \begin{enumerate}
    \label{enum:properties_of_inverses}

    \item $(A\I)\I = A$.

    \item $(AB)\I = B\I A\I$. \imp{In general, $(A_1A_2 \cdots A_k)\I =
      A_k\I A_{k-1}\I \cdots A_2\I A_1\I$.}

    \item $(A^T)\I = (A\I)^T$.
  \end{enumerate}
\end{theorem}

% section the_inverse_of_a_matrix (end)

\section{Inverses of Larger Matrices}
\label{sec:inverses_of_larger_matrices}

\begin{theorem}[Invertible Matrix Theorem]
  \label{thm:invertible_matrix_theorem}

  An $n \times n$ matrix $A$ is invertible if and only if $A$ is row equivalent
  to the identity matrix $I_n$. That is, $A$ can be row reduced into $I_n$. In
  fact, the sequence of row operations to row reduce $A$ to $I_n$ are the same
  row operations to convert $I_n$ to $A\I$.
\end{theorem}

\begin{purpleframe}
  \label{prpl:method_for_finding_a_inverse}

  To find $A\I$ for any $n \times n$ matrix $A$, row reduce $[\,A~I_n\,]$ into
  $[\,I_n~A\I\,]$. However, if $A$ cannot be row reduced to $I_n$, then $A\I$
  does not exist.
\end{purpleframe}

\begin{question}
  \label{qst:finding_a_inverse_with_row_reduction}

  Given the matrix $A =
  \begin{bNiceMatrix}[r,columns-width=auto]
    0 & 1 & 2 \\
    1 & 0 & 3 \\
    4 & -3 & 8 \\
  \end{bNiceMatrix}$, compute its inverse $A\I$ using row reduction if it
  exists.
\end{question}

\begin{solution}
  \label{sol:finding_a_inverse_with_row_reduction}

  We row reduce $[\,A~I_3\,] \to [\,I_3~A\I\,]$ In doing so, we get
  \begin{align*}
    \sysdelim..\systeme*{
      R_1 \rightarrow R_2
    } &\longleftrightarrow
    \begin{bNiceArray}{ccc|ccc}[columns-width=auto]
      \circled{1} & 0 & 3 & 0 & 1 & 0 \\
      0 & 1 & 2 & 1 & 0 & 0 \\
      4 & -3 & 8 & 0 & 0 & 1 \\
    \end{bNiceArray} \\
    \sysdelim..\systeme*{
      \-4R_1 + R_3 \rightarrow R_3
    } &\longleftrightarrow
    \begin{bNiceArray}{ccc|ccc}[columns-width=auto]
      \circled{1} & 0 & 3 & 0 & 1 & 0 \\
      0 & \circled{1} & 2 & 1 & 0 & 0 \\
      0 & -3 & -4 & 0 & -4 & 1 \\
    \end{bNiceArray} \\
    \sysdelim..\systeme*{
      3R_2 + R_3 \rightarrow R_3
    } &\longleftrightarrow
    \begin{bNiceArray}{ccc|ccc}[columns-width=auto]
      \circled{1} & 0 & 3 & 0 & 1 & 0 \\
      0 & \circled{1} & 2 & 1 & 0 & 0 \\
      0 & 0 & \circled{1} & \sfrac{3}{2} & -2 & \sfrac{1}{2} \\
    \end{bNiceArray} \\
    \sysdelim..\systeme*{
      \-2R_3 + R_2 \rightarrow R_2,
      \-3R_3 + R_1 \rightarrow R_1
    } &\longleftrightarrow
    \begin{bNiceArray}{ccc|ccc}[columns-width=auto]
      \circled{1} & 0 & 0 & -\sfrac{9}{2} & 7 & -\sfrac{3}{2} \\
      0 & \circled{1} & 0 & -2 & 4 & -1 \\
      0 & 0 & \circled{1} & \sfrac{3}{2} & -2 & \sfrac{1}{2} \\
    \end{bNiceArray} \\
  .\end{align*}
  Therefore, $A\I$ is
  \[%
    \begin{bNiceMatrix}[columns-width=auto]
      -\sfrac{9}{2} & 7 & -\sfrac{3}{2} \\
      -2 & 4 & -1 \\
      \sfrac{3}{2} & -2 & \sfrac{1}{2} \\
    \end{bNiceMatrix}
  .\qedhere\]%
\end{solution}

% section inverses_of_larger_matrices (end)

\section{More Matrix Transformations}
\label{sec:more_matrix_transformations}

We can use the idea of matrix multiplication to combine or \textit{compose} two
transformations

\begin{question}
  \label{qst:more_matrix_transformations}

  Let $T : \R^2 \to \R^2$ be a linear transformation that first rotates
  vectors by $\sfrac{2}{3}\pi$ radians counterclockwise and then reflects the
  vectors through the $x_1$-axis. Find the standard matrix of $T$.
\end{question}

\begin{solution}
  \label{sol:more_matrix_transformations}

  We want to find $T(\x) = M\x$, where $M$ is the standard matrix. First, a
  rotation by $\theta = \sfrac{2}{3}\pi$ is given by
  \[%
    B =
    \begin{bNiceMatrix}[columns-width=auto]
      \cos(\theta) & -\sin(\theta) \\
      \sin(\theta) & \cos(\theta) \\
    \end{bNiceMatrix} =
    \begin{bNiceMatrix}[columns-width=auto]
      \cos\left(\sfrac{2}{3}\pi\right) & -\sin\left(\sfrac{2}{3}\pi\right) \\
      \sin\left(\sfrac{2}{3}\pi\right) & \cos\left(\sfrac{2}{3}\pi\right) \\
    \end{bNiceMatrix} =
    \begin{bNiceMatrix}[columns-width=auto]
      -\sfrac{1}{2} & -\sfrac{\sqrt{2}}{2} \\
      \sfrac{\sqrt{2}}{2} & -\sfrac{1}{2} \\
    \end{bNiceMatrix}
  .\]%
  Then, the matrix for a reflection about $x_1$-axis is given by
  \[%
    A =
    \begin{bNiceMatrix}[columns-width=auto]
      1 & 0 \\
      0 & 1 \\
    \end{bNiceMatrix}
  .\]%
  To rotate $\x$ first, we have $B\x$. Then, I take the result and reflect by
  $A(B\x) = (AB)\x = M\x \rightarrow M = AB$. Therefore,
  \[%
    M = AB =
    \begin{bNiceMatrix}[columns-width=auto]
      1 & 0 \\
      0 & 1 \\
    \end{bNiceMatrix}
    \begin{bNiceMatrix}[columns-width=auto]
      -\sfrac{1}{2} & -\sfrac{\sqrt{2}}{2} \\
      \sfrac{\sqrt{2}}{2} & -\sfrac{1}{2} \\
    \end{bNiceMatrix} =
    \begin{bNiceMatrix}[columns-width=auto]
      -\sfrac{1}{2} & \sfrac{\sqrt{3}}{2} \\
      \sfrac{\sqrt{3}}{2} & \sfrac{1}{2} \\
    \end{bNiceMatrix}
  .\qedhere\]%
\end{solution}

% section more_matrix_transformations (end)

\section{Inverse of a Transformation}
\label{sec:inverse_of_a_transformation}

Given some transformation $T : \R^n \to \R^m$, we can find some inverse
transformation $T\I : \R^m \to \R^n$.

\begin{theorem}
  \label{thm:inverse_of_a_transformation}

  Let $T : \R^n \to \R^m$ be a linear transformation and let $A$ be the
  standard matrix of $T$. Then, $T$ is invertible if and only if $A$ is
  invertible. Specifically, if $T(\x) = A\x$, then $T\I(\x) = A\I\x$.
\end{theorem}

\begin{question}
  \label{qst:inverse_of_a_transformation}

  Let $T : \R^n \to \R^m$ be a linear transformation given by $T(x,y) =
  (-17x + 6y, 11x + 2y)$. Find a formula for $T\I$, that is, find
  $T\I(x,y)=(\underline{\phantom{Ax+By}},\underline{\phantom{Ax+By}})$.
\end{question}

\begin{solution}
  \label{sol:inverse_of_a_transformation}

  First, rewrite $T$ as $T(\x) = A\x$. In doing so, we get
  \[%
    T\left(\begin{bNiceMatrix}[columns-width=auto]
      x \\
      y \\
    \end{bNiceMatrix}\right) =
    \begin{bNiceMatrix}[columns-width=auto]
      -17x + 6y \\
      4x + 2y \\
    \end{bNiceMatrix} =
    \begin{bNiceMatrix}[columns-width=auto]
      -17x \\
      4x \\
    \end{bNiceMatrix} +
    \begin{bNiceMatrix}[columns-width=auto]
      6y \\
      2y \\
    \end{bNiceMatrix} =
    x\begin{bNiceMatrix}[columns-width=auto]
      -17 \\
      4 \\
    \end{bNiceMatrix} +
    y\begin{bNiceMatrix}[columns-width=auto]
      6 \\
      2 \\
    \end{bNiceMatrix} =
    \underbrace{
      \begin{bNiceMatrix}[columns-width=auto]
        -17 & 6 \\
        4 & 2 \\
      \end{bNiceMatrix}
    }_{A}
    \underbrace{
      \begin{bNiceMatrix}[columns-width=auto]
        x \\
        y \\
      \end{bNiceMatrix}
    }_{\x}
  .\]%
  Then, we can find $A\I$. But first, we need to see if the inverse exists,
  i.e., $\det(A) \ne 0$. In doing so, we get
  \[%
    \det(A) =
    \begin{vNiceMatrix}[columns-width=auto]
      -17 & 6 \\
      4 & 2 \\
    \end{vNiceMatrix} =
    (-17)(2) - (6)(11) = -32 \ne 0
  .\]%
  Therefore, $A\I$ exists. So, by \cref{thm:inverse_of_a_matrix}, $A\I$ is given
  by
  \[%
    A\I = \frac{1}{-32}
    \begin{bNiceMatrix}[columns-width=auto]
      -17 & 6 \\
      4 & 2 \\
    \end{bNiceMatrix} =
    \begin{bNiceMatrix}[columns-width=auto]
      -\sfrac{1}{16} & \sfrac{3}{16} \\
      \sfrac{11}{32} & \sfrac{17}{32} \\
    \end{bNiceMatrix}
  .\]%
  So, we get
  \begin{align*}
    T\I(\x) = A\I\x \rightarrow T\I\left(\begin{bNiceMatrix}[columns-width=auto]
      x \\
      y \\
    \end{bNiceMatrix}\right) &=
    \begin{bNiceMatrix}[columns-width=auto]
      -\sfrac{1}{16} & \sfrac{3}{16} \\
      \sfrac{11}{32} & \sfrac{17}{32} \\
    \end{bNiceMatrix}
    \begin{bNiceMatrix}[columns-width=auto]
      x \\
      y \\
    \end{bNiceMatrix} =
    x\begin{bNiceMatrix}[columns-width=auto]
      -\sfrac{1}{16} \\
      \sfrac{11}{32} \\
    \end{bNiceMatrix} +
    y\begin{bNiceMatrix}[columns-width=auto]
      \sfrac{3}{16} \\
      \sfrac{17}{32} \\
    \end{bNiceMatrix} \\
    &= \begin{bNiceMatrix}[columns-width=auto]
      -\sfrac{1}{16}x + \sfrac{3}{16}y \\
      \sfrac{11}{32}x + \sfrac{17}{32}y \\
    \end{bNiceMatrix} \\
    &= T\I(x, y) = \left(-\frac{1}{16}x + \frac{3}{16}y, \frac{11}{32}x + \frac{17}{32}y\right)
  .\qedhere\end{align*}
\end{solution}

% section inverse_of_a_transformation (end)

\newpage
