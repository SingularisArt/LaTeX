\nte[Section 1.7]{Oct 12 2023 Thu (14:03:01)}{Linear Independence}

\begin{definition}[Linear Independent/Dependent]
  \label{def:linear_independent_dependent}

  A set of vectors $\{\v_1, \v_2, \ldots, \v_p\} \in \R^n$ is \textbf{linearly
  independent} if the vector equation
  \[%
    \sum_{i=1}^p x_i\v_i = z
  ,\]%
  has only the trivial solution $\x = z$.

  If the vector equation above has a nontrivial solution $\x = (c_1, c_2,
  \ldots, c_p)$, then the set of vectors is \textbf{linearly dependent} and we
  can write
  \[%
    \sum_{i=1}^p c_i\v_i = z
  ,\]%
  which is called a \textbf{linear dependence relation}.
\end{definition}

\begin{question}
  \label{qst:linear_independent_dependent}

  Consider the following vectors
  \[%
    \v_1 = \begin{bNiceMatrix}[columns-width=auto]
      1 \\
      2 \\
      3 \\
    \end{bNiceMatrix},\quad
    \v_2 = \begin{bNiceMatrix}[columns-width=auto]
      4 \\
      5 \\
      6 \\
    \end{bNiceMatrix},\aand
    \v_3 = \begin{bNiceMatrix}[columns-width=auto]
      2 \\
      1 \\
      0 \\
    \end{bNiceMatrix}
  .\]%

  \begin{enumerate}
    \label{enum:linear_independent_dependent_qst}

    \item Determine if the set $\{\v_1, \v_2, \v_3\}$ is linearly independent.

    \item If it is not linearly independent, find $a$ linearly dependent
      solution.
  \end{enumerate}
\end{question}

\begin{solution}
  \label{sol:linear_independent_dependent} $ $

  \begin{enumerate}
    \label{enum:linear_independent_dependent_sol}

    \item Recall a linear system has a nontrivial solution if and only if it
      has at least one free variable. The equivalent augmented matrix of
      $x_1\v_1 + x_2\v_2 + x_3\v_3 = z$ is
      \[%
        \begin{bNiceArray}{ccc|c}[columns-width=auto]
          \v_1 & \v_2 & \v_3 & z \\
        \end{bNiceArray} =
        \begin{bNiceArray}{ccc|c}[columns-width=auto]
          1 & 4 & 2 & 0 \\
          2 & 5 & 1 & 0 \\
          3 & 6 & 0 & 0 \\
        \end{bNiceArray}
        \echelon
        \begin{bNiceArray}{ccc|c}[columns-width=auto]
          \circled{1} & 4 & 2 & 0 \\
          0 & \circled{-3} & -3 & 0 \\
          0 & 6 & 0 & 0 \\
        \end{bNiceArray}
      .\]%

      $x_3$ is a free variable, so $\left\{\v_1, \v_2, \v_3\right\}$ has a
      nontrivial solution, which means it is not linearly independent, making it
      linearly dependent.

    \item To find a linearly dependent solution, we need to solve the augmented
      matrix using the Row Reduction Algorithm
      \[%
        \begin{bNiceArray}{ccc|c}[columns-width=auto]
          \circled{1} & 4 & 2 & 0 \\
          0 & \circled{-3} & -3 & 0 \\
          0 & 6 & 0 & 0 \\
        \end{bNiceArray}
        \rref
        \begin{bNiceArray}{ccc|c}[columns-width=auto]
          \circled{1} & 0 & -2 & 0 \\
          0 & \circled{1} & 1 & 0 \\
          0 & 6 & 0 & 0 \\
        \end{bNiceArray}
        \longleftrightarrow
        \sysdelim..\systeme*{
          x_1 - 2x_2 = 0,
          x_2 + x_3 = 0,
          x_3 = x_3
        }
        \longleftrightarrow
        \sysdelim..\systeme*{
          x_1 = 2x_2,
          x_2 = -x_3,
          x_3 = x_3
        }
      .\]%

      We then pick any nonzero value for $x_3$.

      \begin{note}
        \label{nte:dont_pick_trivial_number}

        You shouldn't pick any trivial values, such as $1$ or $-1$
      \end{note}

      Let $x_3 = 5$. Thus, $x_1 = 10$ and $x_2 = -5$.

      So, we get the following linear dependence relation
      \[%
        10\v_1 - 5\v_2 + 5\v_3 = z
      .\qedhere\]%
  \end{enumerate}
\end{solution}

\begin{purpleframe}
  \label{prpl:linear_dependent_matrix_equation}

  The columns of a matrix $A$ are linearly independent if and only if the matrix
  equation $A\x = z$ has only the trivial solution.
\end{purpleframe}

\begin{theorem}[Characterization of Linearly Dependent Sets]
  \label{thm:characterization_of_linearly_dependent_sets}

  An indexed set $S = \left\{\v_1, \dots, \v_p\right\}$ of two or more vectors
  is linearly dependent if and only if at least one of the vectors in $S$ is a
  linear combination of the others. In fact, if $S$ is linearly dependent and
  $\v_1 \ne \zero$, then some $\v_j$, with $j > 1$, is a linear combination of
  the preceding vectors, $\v_1, \dots, \v_{j-1}$.
\end{theorem}

\begin{theorem}
  \label{thm:linearly_dependent}

  A set of vectors $S = \left\{\v_1, \v_2, \ldots, \v_p\right\}$ is linearly
  dependent dependent if and only if at least one of the vectors in $S$ is a
  linear combination of the other vectors.
\end{theorem}

\section{Linear Independence by Inspection}
\label{sec:linear_independence_by_inspection}

Consider two vectors $\v_1, \v_2 \in
\R^2$, as shown below

\begin{figure}[H]
  \centering

  \begin{tikzpicture}
    \begin{axis}[
      xmin=-4.75, xmax=5.75,
      ymin=-0.75, ymax=5.75,
      ]

      \addplot+[->,thick,mark=none] coordinates{(0,0)(4,1)} node[above right]{$\v_2$};
      \addplot+[->,thick,mark=none] coordinates{(0,0)(-3,5)} node[below left]{$\v_1$};
    \end{axis}
  \end{tikzpicture}

  \caption{}
  \label{fig:linear_independence_by_inspection}
\end{figure}

Is there any way to add a 3rd vector $\v_3$ so that all three are linearly
independent? \incorrect{No! Recall that for any vectors $\v_1$ and $\v_2$ such
that $\v_1 \neq \v_2$ then $\Sspan\{\v_1, \v_2\} = \R^2$. In other words, every
vector in $\R^2$ is a linear combination of $\v_1$ and $\v_2$. So any third
vector $\v_3$ will just be a linear combination of $\v_1$ and $\v_2$.}

\begin{theorem}
  \label{thm:linear_independence_by_inspection}

  A set $\left\{\v_1, \v_2, \ldots, \v_p\right\} \in \R^n$ is
  linearly dependent if $p > n$.
\end{theorem}

\begin{proof}
  \label{prf:linear_independence_by_inspection}

  Let $A = [\,\v_1~\cdots~\v_p\,]$. Then $A$ is $n \times p$, and the equation
  $A\x = \zero$ corresponds to a system of $n$ equations in $p$ unknowns. If $p
  > n$, there are more variables than equations, so there must be a free
  variable. Hence, $A\x = \zero$ has a nontrivial solution, and the columns of
  $A$ are linearly dependent.
\end{proof}

\begin{note}
  \label{nte:linear_independence_by_inspection}

  If $p \leq n$ in the theorem above, then we don't know if the set is linearly
  dependent or independent. It could be either.
\end{note}

\begin{theorem}
  \label{thm:contains_zero_vector}

  If a set $\{\v_1, \v_2, \ldots, \v_p\} \in \R^n$ contains the
  zero vector, then the set is linearly dependent.
\end{theorem}

\begin{proof}
  \label{prf:contains_zero_vector}

  By renumbering the vectors, we may suppose $\v_1 = \zero$. Then, the equation
  $1\v_1 + 0\v_2 + \cdots + 0\v_p = \zero$ shows that $S$ is linear dependent.
\end{proof}

\begin{question}
  \label{qst:linear_independence_by_inspection}

  Determine by inspection if the following sets of vectors are linearly
  dependent or not.
  \begin{multicols}{3}
    \begin{enumerate}
      \label{enum:linear_independence_by_inspection_qst}

      \item
        $\begin{bNiceMatrix}[columns-width=auto]
          1 \\
          7 \\
          6 \\
        \end{bNiceMatrix},
        \begin{bNiceMatrix}[columns-width=auto]
          2 \\
          0 \\
          9 \\
        \end{bNiceMatrix},
        \begin{bNiceMatrix}[columns-width=auto]
          3 \\
          1 \\
          5 \\
        \end{bNiceMatrix},
        \begin{bNiceMatrix}[columns-width=auto]
          4 \\
          1 \\
          8 \\
        \end{bNiceMatrix}$
      \item
        $\begin{bNiceMatrix}[columns-width=auto]
          2 \\
          3 \\
          5 \\
        \end{bNiceMatrix},
        \begin{bNiceMatrix}[columns-width=auto]
          0 \\
          0 \\
          0 \\
        \end{bNiceMatrix},
        \begin{bNiceMatrix}[columns-width=auto]
          1 \\
          1 \\
          8 \\
        \end{bNiceMatrix}$
      \item
        $\begin{bNiceMatrix}[columns-width=auto]
          -2 \\
          4 \\
          6 \\
          10 \\
        \end{bNiceMatrix},
        \begin{bNiceMatrix}[columns-width=auto]
          3 \\
          -6 \\
          -9 \\
          -15 \\
        \end{bNiceMatrix}$
    \end{enumerate}
  \end{multicols}
\end{question}

\begin{solution}
  \label{sol:linear_independence_by_inspection} $ $

  \begin{enumerate}
    \label{enum:linear_independence_by_inspection_sol}

    \item These vectors are in $\R^3$, but there are $4$ vectors.
      Thus, by \cref{thm:linear_independence_by_inspection}, this set is
      linearly dependent.

    \item This set of vectors contains the zero vector. Thus, by
      \cref{thm:contains_zero_vector}, this set is linearly dependent.

    \item Notice that $\v_2 = -\sfrac{3}{2}\v_1$, which is a linear
      combination. Thus, by \cref{thm:linearly_dependent} is linearly dependent.
  \end{enumerate}
\end{solution}

\begin{question}
  \label{qst:for_what_values_of_h_is_the_set_linearly_independent}

  Consider the following vectors
  \begin{align*}
    \v_1 = \begin{bNiceMatrix}[columns-width=auto]
      1 \\
      -1 \\
      3 \\
    \end{bNiceMatrix},\quad
    \v_2 = \begin{bNiceMatrix}[columns-width=auto]
      -5 \\
      7 \\
      8 \\
    \end{bNiceMatrix},\aand
    \v_3 = \begin{bNiceMatrix}[columns-width=auto]
      1 \\
      1 \\
      h \\
    \end{bNiceMatrix}
  .\end{align*}
  \begin{enumerate}
    \label{enum:for_what_values_of_h_is_the_set_linearly_independent_qst}

    \item For what value(s) of $h$ is $\v_3$ in $\Sspan\{\v_1, \v_2\}$?
      Justify your answer.

    \item For what value(s) of $h$ is $\{\v_1, \v_2 ,\v_3\}$ linearly
      \textit{dependent}? Justify your answer.

    \item For what value(s) of $h$ is $\{\v_1, \v_2, \v_3\}$ linearly
      \textit{independent}? Justify your answer.
  \end{enumerate}
\end{question}

\begin{solution}
  \label{sol:for_what_values_of_h_is_the_set_linearly_independent}

  \begin{enumerate}
    \label{enum:for_what_values_of_h_is_the_set_linearly_independent_sol}

    \item Recall that if $\v_3 \in \Sspan\left\{\v_1, \v_2\right\}$, it means
      that $x_1\v_1 + x_2\v_2 = \v_3$ has a solution for $x_1$ and $x_2$. In
      other words, we need to find $h$, such that $[\,\v_1~\v_2~\v_3\,]$ is
      consistent.

      \begin{align*}
        \sysdelim..\systeme{
          R_1 + R_2 \rightarrow R_3,
          \-3R_1 + R_3 \rightarrow R_3
        } &\longleftrightarrow
        \begin{bNiceArray}{cc|c}[columns-width=auto]
          \circled{1} & -5 & 1 \\
          0 & \circled{2} & 2 \\
          0 & 23 & h - 3 \\
        \end{bNiceArray} \\
        \sysdelim..\systeme{
          \sfrac{1}{2}R_2 \rightarrow R_2
        } &\longleftrightarrow
        \begin{bNiceArray}{cc|c}[columns-width=auto]
          \circled{1} & -5 & 1 \\
          0 & \circled{1} & 1 \\
          0 & 23 & h - 3 \\
        \end{bNiceArray} \\
        \sysdelim..\systeme{
          \-23R_2 + R_3 \rightarrow R_3
        } &\longleftrightarrow
        \begin{bNiceArray}{cc|c}[columns-width=auto]
          \circled{1} & -5 & 1 \\
          0 & \circled{1} & 1 \\
          0 & 0 & \boxed{h - 26} \\
        \end{bNiceArray}
      .\end{align*}

      If $h - 26 \ne 0$, then the third row becomes $[\,0~0~b\,]$, with $b \ne
      0$, which would make this system inconsistent.

      So, $h - 26 = 0 \implies h = 26$.

    \item Recall that a set $\left\{\v_1, \v_2, \v_3\right\}$ is linearly
      dependent if $x_1\v_1 + x_2\v_2 + x_3\v_3 = z$ has a nontrivial
      solution, so it has at least one free variable. Thus, we find $h$ such
      that it will have one free variable
      \begin{align*}
        \sysdelim..\systeme{
          R_1 + R_2 \rightarrow R_3,
          \-3R_1 + R_3 \rightarrow R_3
        } &\longleftrightarrow
        \begin{bNiceArray}{ccc|c}[columns-width=auto]
          \circled{1} & -5 & 1 & 0 \\
          0 & \circled{2} & 2 & 0 \\
          0 & 23 & h - 3 & 0 \\
        \end{bNiceArray} \\
        \sysdelim..\systeme{
          \sfrac{1}{2}R_2 \rightarrow R_2
        } &\longleftrightarrow
        \begin{bNiceArray}{ccc|c}[columns-width=auto]
          \circled{1} & -5 & 1 & 0 \\
          0 & \circled{1} & 1 & 0 \\
          0 & 23 & h - 3 & 0 \\
        \end{bNiceArray} \\
        \sysdelim..\systeme{
          \-23R_2 + R_3 \rightarrow R_3
        } &\longleftrightarrow
        \begin{bNiceArray}{ccc|c}[columns-width=auto]
          \circled{1} & -5 & 1 & 0 \\
          0 & \circled{1} & 1 & 0 \\
          0 & 0 & \boxed{h - 26} & 0 \\
        \end{bNiceArray}
      .\end{align*}

      If $h - 26 = 0$, then the third column has no pivot, which makes $x_3$ a
      free variable.

      \noindent If $h - 26 \ne 0$, then the third column has a pivot, which
      makes $x_3$ a basic variable.

    \item We know that $h = 26$ makes it linearly dependent, so $h \ne 26$ will
      make it linearly independent. \qedhere
  \end{enumerate}
\end{solution}

\begin{note}
  Given a matrix equation $A\x = \b$ where $A = [\,\a_1~\a_2~\cdots~\a_n\,]$,
  row reducing the coefficient matrix $[\,\a_1~\a_2~\cdots~\a_n\,]$ has the
  same row operations as row reducing the augmented matrix
  $[\,\a_1~\a_2~\cdots~\a_n~\b\,]$.
\end{note}

% section linear_independence_by_inspection (end)

\newpage
