\nte[Sections 2.8, 2.9 and 4.3]{Nov 07 2023 Tue (14:00:43)}{Subspaces}

\section{What is a Subspace}
\label{sec:what_is_a_subspace}

\begin{definition}[Subspace]
  \label{def:subspace}

  A \textbf{subspace} of $\R^n$ is any set $H \in \R^n$ that has the
  following properties
  \begin{enumerate}
    \label{enum:subspace}

    \item $\zero \in H$.

    \item $\forall \u,\v \in H$, $\u + \v \in H$. (\imp{Closed under
      addition}).

    \item $\forall \u \in H$ and any scalar $k$, $k\u \in H$. (\imp{Closed under
      scalar multiplication}).
  \end{enumerate}
\end{definition}

\begin{question}
  \label{qst:subspace}

  Consider any two vectors $\v_1,\v_2 \in \R^n$ and let $H =
  \Sspan\{\v_1,\v_2\}$. Show that $H$ is a subspace of $\R^n$.
\end{question}

\begin{solution}
  \label{sol:subspace}

  First, $\Sspan\{\v_1,\v_2\}$ is the set of all linear combinations of $\v_1$
  and $\v_2$.

  \begin{enumerate}
    \label{enum:subspace_solution}

    \item Notice that $\zero = 0\v_1 + 0\v_2 \implies \zero \in H$.

    \item Let $\u,\w \in H$. Then, $\u = a\v_1 + b\v_1$ and $\w = c\v_1 +
      d\v_2$. We need to show that $\u + \w \in H$.
      \[%
        \u + \w = a\v_1 + b\v_2 + c\v_1 + d\v_2 = a\v_1 + c\v_2 + b\v_2 + d\v_2 = \underbrace{(a + c)}_{\textrm{scalar}}\v_1 + \underbrace{(b + d)}_{\textrm{scalar}}\v_2 \in H
      .\]%

    \item Show that $k\u \in H$.
      \[%
        k\u = k(a\v_1 + b\v_2) = ka\v_1 + kb\v_2 \in \Sspan\{\v_1,\v_2\} \implies k\u \in H
      .\qedhere\]%
  \end{enumerate}
\end{solution}

\begin{note}
  \label{nte:span_is_a_subspace}

  In fact, $\forall \v_1,\v_2,\cdots,\v_p \in \R^n$,
  $\Sspan\{\v_1,\v_2,\cdots,\v_p\}$ is a subspace of $\R^n$.

  Also, $\zero$ is a subspace of $\R^n$. This is sometimes called the \textbf{zero
  subspace}.
\end{note}

% section what_is_a_subspace (end)

\section{Column Space and Null Space}
\label{sec:column_space_and_null_space}

\begin{definition}[Column and Null Space]
  \label{def:column_and_null_space}

  Given some matrix $A$, the \textbf{column space} of $A$, denoted $\Col(A)$ is
  the set of all linear combinations of the columns of $A$.

  The \textbf{null space} of $A$, denoted $\Nul(A)$ is the set the solutions of
  the equation $A\x = \zero$.
\end{definition}

\begin{purpleframe}
  \label{prpl:column_and_null_space}

  If $A$ is an $m \times n$ matrix, then $\Col(A)$ is a subspace of $\R^m$ and
  $\Nul(A)$ is a subspace of $\R^n$.
\end{purpleframe}

\begin{question}
  \label{qst:column_and_null_space} $ $

  \begin{enumerate}
    \label{enum:column_and_null_space_qst}

    \item How many vectors are in $\{\v_1,\v_2,\v_3\}$?
    \item Consider the matrix $A = [\,\v_1~\v_2~\v_3\,]$. How many vectors does $\Col(A)$ have?
    \item Is $\p$ in $\Col(A)$? Justify your answer.
    \item Is $\p$ in $\Nul(A)$? Justify your answer.
  \end{enumerate}
\end{question}

\begin{solution}
  \label{sol:column_and_null_space} $ $

  \begin{enumerate}
    \label{enum:column_and_null_space_sol}

    \item $\{\v_1,\v_2,\v_3\}$ is a set of 3 vectors.

    \item $\Col(A)$ is the set of all linear combinations of $\v_1,\v_2,\v_3$.
      \[%
        \Col(A) = \Sspan\{\v_1,\v_2,\v_3\}
      .\]%
      This is an infinite set.

    \item In order for $\p$ to be in $\Col(A)$, $\p$ must be a linear
      combination of the columns of $A$. So, $\p = x_1\v_1 + x_2\v_2 + x_3\v_3$
      must have a solution. In doing so, we get
      \[%
        \begin{bNiceMatrix}[columns-width=auto]
          \v_1 & \v_2 & \v_3 & \p \\
        \end{bNiceMatrix} =
        \begin{bNiceMatrix}[columns-width=auto]
          -1 & 6 & 2 & -31 \\
          1 & -2 & -4 & 11 \\
          3 & 1 & 3 & -2 \\
        \end{bNiceMatrix} \echelon
        \begin{bNiceMatrix}[columns-width=auto]
          -1 & 6 & 2 & -31 \\
          0 & 4 & -2 & -20 \\
          0 & 0 & \sfrac{37}{2} & 0 \\
        \end{bNiceMatrix}
      .\]%
      Since there is no row of the form $[\,0~0~0~b\,]$ with $b \ne 0$, then the
      original equation is consistent, meaning $\p \in \Col(A)$.

    \item In order for $\p$ to be in $\Nul(A)$, $\p$ must be a solution to $A\x
      = \zero$. In doing so, we get
      \[%
        \begin{bNiceMatrix}[columns-width=auto]
          -1 & 6 & 2 \\
          1 & -2 & -4 \\
          3 & 1 & 3 \\
        \end{bNiceMatrix}
        \begin{bNiceMatrix}[columns-width=auto]
          -31 \\
          11 \\
          -2 \\
        \end{bNiceMatrix} =
        -31\begin{bNiceMatrix}[columns-width=auto]
          -1 \\
          1 \\
          3 \\
        \end{bNiceMatrix} +
        11\begin{bNiceMatrix}[columns-width=auto]
          6 \\
          -2 \\
          1 \\
        \end{bNiceMatrix} -
        2\begin{bNiceMatrix}[columns-width=auto]
          2 \\
          -4 \\
          3 \\
        \end{bNiceMatrix} =
        \begin{bNiceMatrix}[columns-width=auto]
          93 \\
          -45 \\
          -88 \\
        \end{bNiceMatrix} \ne \zero
      .\]%
      Thus, $\p \notin \Nul(A)$.
  \end{enumerate}
\end{solution}

\begin{definition}[Basis]
  \label{def:basis}

  A \textbf{basis} for a subspace $H$ of $\R^n$ is a finite subset of $S$ of $H$
  that is linearly independent and spans $H$. That is, all vectors in $H$ can be
  written as a linear combination of the vectors in $S$.
\end{definition}

\begin{example}
  \label{exm:basis}

  The columns of the identity matrix $I_n$ which are $\e_1,\e_2,\dots\e_n$ form
  a basis for $\R^n$.
\end{example}

\begin{purpleframe}
  \label{prpl:basis}

  Given a matrix $A$, to find a basis for $\Nul(A)$, solve the equation $A\x =
  \zero$ and write the solution $\x$ in parametric vector form. The vectors in this
  solution set form a basis for $\Nul(A)$.
\end{purpleframe}

\begin{question}
  \label{qst:basis_for_null_space}

  Find a basis for the null space of the matrix $A$ shown below.
  \[%
    A = \begin{bNiceMatrix}[columns-width=auto]
      -3 & 6 & -1 & 1 & -7 \\
      1 & -2 & 2 & 3 & -1 \\
      2 & -4 & 5 & 8 & -4 \\
    \end{bNiceMatrix}
  .\]%
\end{question}

\begin{solution}
  \label{sol:basis_for_null_space}

  We solve $A\x = \zero$ in parametric vector form
  \[%
    \begin{bNiceMatrix}[columns-width=auto]
      A & \zero \\
    \end{bNiceMatrix} =
    \begin{bNiceMatrix}[columns-width=auto]
      -3 & 6 & -1 & 1 & -7 & 0 \\
      1 & -2 & 2 & 3 & -1 & 0 \\
      2 & -4 & 5 & 8 & -4 & 0 \\
    \end{bNiceMatrix} \rref
    \begin{bNiceMatrix}[columns-width=auto,last-row]
      \circled{1} & -2 & 0 & -1 & 3 & 0 \\
      0 & 0 & \circled{1} & 2 & -2 & 0 \\
      0 & 0 & 0 & 0 & 0 & 0 \\
      \imp{x_1} & \imp{x_2} & \imp{x_3} & \imp{x_4} & \imp{x_r} & \\
    \end{bNiceMatrix}
  .\]%
  This gives us the following general solution
  \[%
    \sysdelim..\systeme*{
      x1 - 2x_2 - x_4 + 3x_5 = 0,
      x_2 = x_2,
      x_3 + 2x_4 - 2x_5 = 0,
      x_4 = x_4,
      x_5 = x_5
    } \generalsol
    \sysdelim..\systeme*{
      x_1 = 2x_2 + x_4 - 3x_5,
      x_2 = x_2,
      x_3 = -2x_4 + 2x_5,
      x_4 = x_4,
      x_5 = x_5
    }
  .\]%
  Using that, we can solve for $\x$. In doing so, we get
  \begin{align*}
    \x &= \begin{bNiceMatrix}[columns-width=auto]
      x_1 \\
      x_2 \\
      x_3 \\
      x_4 \\
      x_5 \\
    \end{bNiceMatrix} =
    \begin{bNiceMatrix}[columns-width=auto]
      2x_2 + x_4 - 3x_5 \\
      x_2 \\
      -2x_4 + 2x_5 \\
      x_4 \\
      x_5 \\
    \end{bNiceMatrix} =
    \begin{bNiceMatrix}[columns-width=auto]
      2x_2 \\
      x_2 \\
      0 \\
      0 \\
      0 \\
    \end{bNiceMatrix} +
    \begin{bNiceMatrix}[columns-width=auto]
      x_4 \\
      0 \\
      -2x_4 \\
      x_4 \\
      0 \\
    \end{bNiceMatrix} +
    \begin{bNiceMatrix}[columns-width=auto]
      -3x_5 \\
      0 \\
      2x_5 \\
      0 \\
      x_5 \\
    \end{bNiceMatrix} \\
    &= x_2\begin{bNiceMatrix}[columns-width=auto]
      2 \\
      1 \\
      0 \\
      0 \\
      0 \\
    \end{bNiceMatrix} +
    x_4\begin{bNiceMatrix}[columns-width=auto]
      1 \\
      0 \\
      -2 \\
      1 \\
      0 \\
    \end{bNiceMatrix} +
    x_5\begin{bNiceMatrix}[columns-width=auto]
      -3 \\
      0 \\
      2 \\
      0 \\
      1 \\
    \end{bNiceMatrix}
  .\end{align*}
  Thus, a basis for $\Nul(A)$ is
  \[%
    \left\{
      \begin{bNiceMatrix}[columns-width=auto,last-row]
        2 \\
        1 \\
        0 \\
        0 \\
        0 \\
        \imp{\v_1} \\
      \end{bNiceMatrix},
      \begin{bNiceMatrix}[columns-width=auto,last-row]
        1 \\
        0 \\
        -2 \\
        1 \\
        0 \\
        \imp{\v_2} \\
      \end{bNiceMatrix},
      \begin{bNiceMatrix}[columns-width=auto,last-row]
        -3 \\
        0 \\
        2 \\
        0 \\
        1 \\
        \imp{\v_3} \\
      \end{bNiceMatrix}
    \right\}
  .\qedhere\]%
\end{solution}

\begin{question}
  \label{qst:basis_for_column_space}

  Find a basis for the column space of the matrix $A$ shown below.
  \begin{equation*}
    A = \begin{bNiceMatrix}[columns-width=auto]
      1 & 3 & 3 & 2 & -9 \\
      -2 & -2 & 2 & -8 & 2 \\
      2 & 3 & 0 & 7 & 1 \\
      3 & 4 & -1 & 11 & 8 \\
    \end{bNiceMatrix}
  \end{equation*}
\end{question}

\begin{solution}
  \label{sol:basis_for_column_space}

  We row reduce $A$ into echelon form. In doing so, we get
  \begin{align*}
    \begin{bNiceMatrix}[columns-width=auto]
      1 & 3 & 3 & 2 & -9 \\
      -2 & -2 & 2 & -8 & 2 \\
      2 & 3 & 0 & 7 & 1 \\
      3 & 4 & -1 & 11 & 8 \\
    \end{bNiceMatrix} \echelon
    \begin{bNiceMatrix}[columns-width=auto,last-row]
      \circled{1} & 0 & -3 & 5 & 0 \\
      0 & \circled{1} & 2 & -1 & 0 \\
      0 & 0 & 0 & 0 & \circled{1} \\
      0 & 0 & 0 & 0 & 0 \\
      \imp{\a_1} & \imp{\a_2} & \imp{\a_3} & \imp{\a_4} & \imp{\a_5} \\
    \end{bNiceMatrix}
  .\end{align*}
  Thus, the basis for $\Col(A)$ is
  \[%
    \left\{
      \begin{bNiceMatrix}[columns-width=auto,last-row]
        1 \\
        0 \\
        0 \\
        0 \\
        \imp{\a_1} \\
      \end{bNiceMatrix},
      \begin{bNiceMatrix}[columns-width=auto,last-row]
        0 \\
        1 \\
        0 \\
        0 \\
        \imp{\a_2} \\
      \end{bNiceMatrix},
      \begin{bNiceMatrix}[columns-width=auto,last-row]
        0 \\
        0 \\
        1 \\
        0 \\
        \imp{\a_5} \\
      \end{bNiceMatrix}
    \right\}
  .\qedhere\]%
\end{solution}

% section column_space_and_null_space (end)

\section{Dimension and Rank}
\label{sec:dimension_and_rank}

\begin{definition}[Dimension and Rank]
  \label{def:dimension_and_rank}

  The \textbf{dimension} of a nonzero subspace $H$, denoted as $\dim(H)$, is the
  number of vectors in any basis of $H$. The dimension of the zero subspace is
  defined to be $0$. \noindent The \textbf{rank} of a matrix $A$, denoted as
  $\ran(A)$, is the dimension of the column space of $A$.
\end{definition}

\begin{theorem}[The Rank Theorem]
  \label{thm:the_rank_theorem}

  If a matrix $A$ has $n$ columns, then $\ran(A) + \dim(\Nul(A)) = n$.
\end{theorem}

\begin{question}
  \label{qst:dimension_and_rank}

  Again consider the matrix $A$ shown below.
  \[%
    A = \begin{bNiceMatrix}[r,columns-width=auto]
      -3 & 6 & -1 & 1 & -7 \\
      1 & -2 & 2 & 3 & -1 \\
      2 & -4 & 5 & 8 & -4 \\
    \end{bNiceMatrix}
  .\]%
  Find the dimension of the null space of $A$ (sometimes called the
  \textit{nullity} of $A$) and also the rank of $A$.
\end{question}

\begin{solution}
  \label{sol:dimenson_and_rank}

  Recall, we found a basis for $\Nul(A)$ in \cref{sol:basis_for_null_space},
  which was
  \[%
    \left\{
      \begin{bNiceMatrix}[columns-width=auto,last-row]
        2 \\
        1 \\
        0 \\
        0 \\
        0 \\
        \imp{\v_1} \\
      \end{bNiceMatrix},
      \begin{bNiceMatrix}[columns-width=auto,last-row]
        1 \\
        0 \\
        -2 \\
        1 \\
        0 \\
        \imp{\v_2} \\
      \end{bNiceMatrix},
      \begin{bNiceMatrix}[columns-width=auto,last-row]
        -3 \\
        0 \\
        2 \\
        0 \\
        1 \\
        \imp{\v_3} \\
      \end{bNiceMatrix}
    \right\} \rightarrow \dim(\Nul(A)) = 3
  .\]%
  Also, let's take a look at the pivot columns of $A$. We already row reduced
  $[\,A~\zero\,]$ in \cref{sol:basis_for_column_space}, which was
  \[%
    \begin{bNiceMatrix}[columns-width=auto,last-row]
      \circled{1} & 0 & -3 & 5 & 0 \\
      0 & \circled{1} & 2 & -1 & 0 \\
      0 & 0 & 0 & 0 & \circled{1} \\
      0 & 0 & 0 & 0 & 0 \\
      \imp{\v_1} & \imp{\v_2} & \imp{\v_3} & \imp{\v_4} & \imp{\v_5} \\
    \end{bNiceMatrix} \rightarrow \ran(A) = \dim(\Col(A)) = 2
  .\qedhere\]%
\end{solution}

\begin{theorem}
  \label{thm:nullity_plus_rank_is_n}

  If a matrix $A$ has $n$ columns, then $\dim(\Nul(A)) + \ran(A) = n$.
\end{theorem}

% section dimension_and_rank (end)

\section{Row Space}
\label{sec:row_space}

\begin{definition}[Row Space]
  \label{def:row_space}

  Given some matrix $A$, the \textbf{row space} of $A$, denoted as $\Row(A)$, is
  the set of all linear combinations of the rows of $A$.
\end{definition}

Similar to column space, some rows may be redundant. Hence, to remove the
redundant rows, you need to row reduce $A$ into echelon form and look for the
nonzero rows.

\begin{purpleframe}
  \label{prpl:row_space}

  If two matrices $A$ and $B$ are row equivalent, then $\Row(A) = \Row(B)$. If
  $B$ is the reduced echelon form of $A$, then the nonzero rows of $B$ form a
  basis for row $A$.
\end{purpleframe}

\begin{example}
  \label{exm:row_space}

  In \cref{sol:dimenson_and_rank}, a basis for $\Row(A)$ should be
  \[%
    \left\{
      \begin{bNiceMatrix}[columns-width=auto]
        1 \\
        -2 \\
        0 \\
        -1 \\
        3 \\
        0 \\
      \end{bNiceMatrix},
      \begin{bNiceMatrix}[columns-width=auto]
        0 \\
        0 \\
        1 \\
        2 \\
        -2 \\
        0 \\
      \end{bNiceMatrix}
    \right\}
  .\]%
\end{example}

\begin{note}
  \label{nte:row_space}

  When the columns of $A\T$ are the rows in the original matrix $A$. Hence,
  $\Col(A\T) = \Row(A)$. Similarly, the rows in $A\T$ are actually the columns
  in $A$. So, $\Row(A\T) = \Col(A)$.
\end{note}

% section row_space (end)

\newpage
