\nte[Sections 4.4 and 4.6]{Nov 21 2023 Tue (14:00:05)}{Systems and Bases}

\section{Coordinate Systems}
\label{sec:coordinate_systems}

\begin{theorem}[Unique Representation Theorem]
  \label{thm:unique_representation_theorem}

  Let $\B = \{\b_1, \b_2, \ldots, \b_n\}$ be a basis for a vector space $V$.
  Then $\forall \x \in V$ there is a unique set of scalars $c_1, c_2, \ldots,
  c_n$ such that
  \begin{equation}\label{eqt:thm_unique_representation_theorem}
    \x = c_1\b_1 + c_2\b_2 + \cdots + c_n\b_n
  .\end{equation}
\end{theorem}

\begin{proof}
  \label{prf:unique_representation_theorem}

  Since $\beta$ spans $V$, there exists scalars such that
  \cref{eqt:thm_unique_representation_theorem} holds. Suppose $\x$ also has the
  representation
  \[%
    \x = d_1\b_1 + \cdots d_n\b_n
  ,\]%
  for scalars $d_1, \dots, d_n$. Then, subtracting, we have
  \begin{equation}\label{eqt:prf_unique_representation_theorem_1}
    \zero = \x - \x = (c_1 - d_1)\b_1 + \cdots + (c_n - d_n)\b_n
  .\end{equation}
  Since $\beta$ is linearly independent, the weights in
  \cref{eqt:prf_unique_representation_theorem_1} must all be zero. That is, $c_j
  = d_j$ for $1 \le j \le n$.
\end{proof}

The uniqueness of the representation allows us to use the basis $\B$ as a
``coordinates system''.

\begin{definition}[Coordinates of a Vector]
  \label{def:coordinates_of_a_vector}

  Suppose $\B = \{\b_1, \b_2, \ldots, \b_n\}$ is a basis for a vector space $V$
  and $\x \in V$. Then the \textbf{coordinates of $x$ relative to the basis
  $\B$} (also called the \textit{$\B$-coordinate vector of $\x$}) is the vector
  \[%
    [\x]_\B =
    \begin{bNiceMatrix}[columns-width=auto]
      c_1 \\
      c_2 \\
      \vdots \\
      c_n \\
    \end{bNiceMatrix}
  .\]%
  where $c_1, c_2, \ldots, c_n$ are the unique scalars such that $\x = c_1\b_1
  + c_2\b_2 + \cdots + c_n\b_n$. Also, the transformation $\x \mapsto [\x]_\B$
  is called the \textbf{coordinate mapping determine by $\B$}.
\end{definition}

\begin{example}
  \label{exm:coordinates_of_a_vector}

  Consider the following vectors
  \[%
    \b_1 =
    \begin{bNiceMatrix}[columns-width=auto]
      1 \\
      0 \\
    \end{bNiceMatrix},\quad
    \b_2 =
    \begin{bNiceMatrix}[columns-width=auto]
      1 \\
      2 \\
    \end{bNiceMatrix},\quad
    \e_1 =
    \begin{bNiceMatrix}[columns-width=auto]
      1 \\
      0 \\
    \end{bNiceMatrix},\aand
    \e_2 =
    \begin{bNiceMatrix}[columns-width=auto]
      0 \\
      1 \\
    \end{bNiceMatrix}
  .\]%
  Let $\B = \{\b_1, \b_2\}$ and $\E = \{\e_1, \e_2\}$ be bases for the vector
  space $\R^2$. Now consider the vector
  \[%
    \x =
    \begin{bNiceMatrix}[columns-width=auto]
      1 \\
      6 \\
    \end{bNiceMatrix}
  .\]%
  Let's look at this vector from the perspectives of each basis. Notice that we
  can write $\x$ as
  \[%
    \x =
    \begin{bNiceMatrix}[columns-width=auto]
      1 \\
      6 \\
    \end{bNiceMatrix} =
    \begin{bNiceMatrix}[columns-width=auto]
      1 \\
      0 \\
    \end{bNiceMatrix} +
    \begin{bNiceMatrix}[columns-width=auto]
      0 \\
      6 \\
    \end{bNiceMatrix} =
    1\begin{bNiceMatrix}[columns-width=auto]
      1 \\
      0 \\
    \end{bNiceMatrix} + 6
    \begin{bNiceMatrix}[columns-width=auto]
      0 \\
      1 \\
    \end{bNiceMatrix} = 1\e_1 + 6\e_2
  .\]%
  Thus, the $\E$-coordinates of $\x$ must be $[\x]_\E = (1, 6)$. Now notice that
  we can write $\x$ as
  \[%
    \x =
    \begin{bNiceMatrix}[columns-width=auto]
      1 \\
      6 \\
    \end{bNiceMatrix} =
    -2\begin{bNiceMatrix}[columns-width=auto]
      1 \\
      0 \\
    \end{bNiceMatrix} + 3
    \begin{bNiceMatrix}[columns-width=auto]
      1 \\
      2 \\
    \end{bNiceMatrix} = -2\b_1 + 3\b_2
  .\]%
  Thus, the $\B$-coordinates of $\x$ must be $[\x]_\B = (-2, 3)$. Graphically,
  each coordinate vector $[\x]_\E$ and $[\x]_\B$ gives us ``directions'' for
  graphing $\x$ on a coordinates system whose grid lines are formed by linear
  combinations of $\E$ and $\B$, respectively.
  \begin{figure}[H]
    \centering

    \begin{tikzpicture}[scale=0.5]
      \draw[black!60] (-2,-2) grid (3,7);
      \filldraw[black] (1,6) circle (3pt) node[below right]{$\x$};
      \draw[ultra thick,linecolor1,<->] (1,0) node[below]{$\e_2$} -- (0,0) -- (0,1) node[left]{$\e_1$};
      \draw[thick,linecolor1,-] (-2,0) -- (3,0);
      \draw[thick,linecolor1,-] (0,-2) -- (0,7);
      \draw (0.5,-2.5) node[below]{$\R^2$ with a $\E$-grid};
    \end{tikzpicture}\qquad\qquad\qquad
    \begin{tikzpicture}[scale=0.25]
      \pgftransformcm{2.5}{0}{1}{2}{\pgfpoint{0}{0}}
      \draw[black!60] (-3,-2) grid (3,7);
      \filldraw[black] (-2,3) circle (3pt) node[above right]{$\x$};
      \draw[ultra thick,linecolor1,<->] (1,0) node[right]{$\b_2$} -- (0,0) -- (0,1) node[left]{$\b_1$};
      \draw[thick,linecolor1,-] (-3,0) -- (3,0);
      \draw[thick,linecolor1,-] (0,-2) -- (0,7);
      \draw (0,-2.5) node[below]{$\R^2$ with a $\B$-grid};
    \end{tikzpicture}

    \caption{}
    \label{fig:coordinates_of_a_vector}
  \end{figure}

  The reason we care about seeing vectors from different perspectives or
  ``grids'' is because some calculations or operations may be easier in
  different grids.
\end{example}

\begin{question}
  \label{qst:coordinates_of_a_vector}

  Consider the following vectors
  \[%
    \b_1 =
    \begin{bNiceMatrix}[columns-width=auto]
      1 \\
      -4 \\
      3 \\
    \end{bNiceMatrix},\quad
    \b_2 =
    \begin{bNiceMatrix}[columns-width=auto]
      5 \\
      2 \\
      -2 \\
    \end{bNiceMatrix},\quad
    \b_3 =
    \begin{bNiceMatrix}[columns-width=auto]
      4 \\
      -7 \\
      0 \\
    \end{bNiceMatrix},\quad
    \x =
    \begin{bNiceMatrix}[columns-width=auto]
      5 \\
      -7 \\
      10 \\
    \end{bNiceMatrix},\aand
    [\y]_\B =
    \begin{bNiceMatrix}[columns-width=auto]
      3 \\
      0 \\
      -1 \\
    \end{bNiceMatrix}
  .\]%

  \begin{enumerate}
    \label{enum:coordinates_of_a_vector_qst}

    \item Find the coordinate vector $[\x]_\B$ of $\x$ relative to the basis
      $\B = \{\b_1, \b_2, \b_3\}$.

    \item Find the vector $\y$ determined by the given coordinate vector
      $[\y]_\B$ for the basis $\B = \{\b_1, \b_2, \b_3\}$.
  \end{enumerate}
\end{question}

\begin{solution}
  \label{sol:coordinates_of_a_vector} $ $

  \begin{enumerate}
    \label{enum:coordinates_of_a_vector_sol}

    \item Recall, if $[\x]_\B = (c_1, c_2, c_3)$, then $\x = c_1\b_1 + c_2\b_2 +
      c_3\b_3$. So, you can solve this vector equation with
      \[%
        \begin{bNiceArray}{ccc|c}
          \b_1 & \b_2 & \b_3 & \x \\
        \end{bNiceArray} =
        \begin{bNiceArray}{ccc|c}
          1 & 5 & 4 & 5 \\
          -4 & 2 & -7 & -7 \\
          3 & -2 & 0 & 10 \\
        \end{bNiceArray} \rref
        \begin{bNiceArray}{ccc|c}[last-row]
          \circled{1} & 0 & 0 & 4 \\
          0 & \circled{1} & 0 & 1 \\
          0 & 0 & \circled{1} & -1 \\
          \imp{c_1} & \imp{c_2} & \imp{c_3} & \\
        \end{bNiceArray} \rightarrow [\x]_\B = (4, 1, -1)
      .\]%

    \item Recall, if $[\y]_\B = (3, 0, -1)$, then
      \[%
        \y = 3\b_1 + 0\b_2 - 1\b_3 =
        3\begin{bNiceMatrix}[columns-width=auto]
          1 \\
          -4 \\
          3 \\
        \end{bNiceMatrix} +
        0\begin{bNiceMatrix}[columns-width=auto]
          5 \\
          2 \\
          -2 \\
        \end{bNiceMatrix} -
        1\begin{bNiceMatrix}[columns-width=auto]
          4 \\
          -7 \\
          0 \\
        \end{bNiceMatrix} =
        \begin{bNiceMatrix}[columns-width=auto]
          -1 \\
          -5 \\
          9 \\
        \end{bNiceMatrix}
      .\qedhere\]%
  \end{enumerate}
\end{solution}

Recall that we can write $\x = c_1\b_1 + c_2\b_2 + c_3\b_3$ from part $1$ to
\[%
  \underbrace{
    \begin{bNiceMatrix}[columns-width=auto]
      1 & 5 & 4 \\
      -4 & 2 & -7 \\
      3 & -2 & 0 \\
    \end{bNiceMatrix}
  }_{\imp{P_\B}}
  \underbrace{
    \begin{bNiceMatrix}[columns-width=auto]
      c_1 \\
      c_2 \\
      c_3 \\
    \end{bNiceMatrix}
  }_{\imp{[\x]_\B}} =
  \underbrace{
    \begin{bNiceMatrix}[columns-width=auto]
      5 \\
      -7 \\
      10 \\
    \end{bNiceMatrix}
  }_{\imp{\x}}
.\]%
The matrix $P_\B$ is called the \textbf{change-of-coordinates matrix} from $\B$
to the standard basis $\E \in \R^n$. In general, if $\B = \{\b_1, \b_2, \dots,
\b_n\}$ is a basis for $\R^n$, then the change-of-coordinates matrix from $\B$
to the standard basis is given by
\[%
  P_\B =
  \begin{bNiceMatrix}[columns-width=auto]
    \b_1 & \b_2 & \cdots & \b_n \\
  \end{bNiceMatrix}
,\]%
and if $\x \in \R^n$, then $P_\B[\x]_\B = \x$.

This matrix allows us to view ``moving'' from a $\B$-grid of $\R^n$ to a
$\E$-grid of $\R^n$ as a transformation. Specifically, the transformation
$[\x]_\B \mapsto \x$ is given by the formula
\[%
  T([\x]_\B) = P_\B[\x]_\B
.\]%
However, remember that $x \mapsto [\x]_\B$ was the coordinate mapping determined
by $\B$. Because $P_\B$ is invertible, then we can obverse that
\[%
  \rcancel{P\I_\B} \rcancel{P_\B}[\x]_\B = P\I_\B\x \implies P\I_\B\x = [\x]_\B
.\]%
Hence, the transformation $[\x]_\B \mapsto x$ whose standard matrix is given by
$P_\B$, is a linear, one-to-one, and onto from $\R^n$ to $\R^n$. In addition,
$P_\B$ is invertible. The same applies to $\x \mapsto [\x]_\B$, denoted as
$[~]_\B$, and $P\I_\B$.

\begin{question}
  \label{qst:change_of_coordinates_matrix}

  Consider the following basis of $\R^3$ below.
  \[%
    \B = \left\{
      \underset{\imp{\b_1}}{
        \begin{bNiceMatrix}[columns-width=auto]
          1 \\
          0 \\
          3 \\
        \end{bNiceMatrix}
      },
      \underset{\imp{\b_2}}{
        \begin{bNiceMatrix}[columns-width=auto]
          2 \\
          1 \\
          8 \\
        \end{bNiceMatrix}
      },
      \underset{\imp{\b_3}}{
        \begin{bNiceMatrix}[columns-width=auto]
          1 \\
          -1 \\
          2 \\
        \end{bNiceMatrix}
      }
    \right\}
  .\]%
  \begin{enumerate}
    \label{enum:change_of_coordinates_matrix_qst}

    \item Find the change-of-coordinates matrix from $\B$ to the standard basis
      $\E$ in $\R^3$.

    \item Use your answer in part $1$ to find the change-of-coordinates matrix
      from $\E$ to $\B$.
  \end{enumerate}
\end{question}

\begin{solution}
  \label{sol:change_of_coordinates_matrix} $ $

  \begin{enumerate}
    \label{enum:change_of_coordinates_matrix_sol}

    \item Since $\B = \{\b_1, \b_2, \b_3\}$, then the change-of-coordinates from
      $\B$ to $\E$ is given by
      \[%
        P_\B =
        \begin{bNiceMatrix}[columns-width=auto]
          \b_1 & \b_2 & \b_3 \\
        \end{bNiceMatrix} =
        \begin{bNiceMatrix}[columns-width=auto]
          1 & 2 & 1 \\
          0 & 1 & -1 \\
          3 & 8 & 2 \\
        \end{bNiceMatrix}
      .\]%

    \item Then, the change-of-coordinates from $\E$ to $\B$ is given by
      $P\I_\B$. We can find $P\I_\B$ by row reducing the augmented matrix
      \[%
        \begin{bNiceMatrix}[columns-width=auto]
          P_\B~I \\
        \end{bNiceMatrix} =
        \begin{bNiceArray}{ccc|ccc}
          1 & 2 & 1 & 1 & 0 & 0 \\
          0 & 1 & -1 & 0 & 1 & 0 \\
          3 & 8 & 2 & 0 & 0 & 1 \\
        \end{bNiceArray} \rref
        \begin{bNiceArray}{ccc|ccc}[last-row]
          \CodeBefore
          \tikz \draw[thick,main] ([xshift=1mm]1-|4) rectangle ([xshift=-2mm]last-|last);
          \Body
          1 & 0 & 0 & 10 & 4 & -3 ~ \\
          0 & 1 & 0 & -3 & -1 & 1 ~ \\
          0 & 0 & 1 & -3 & -2 & 1 ~ \\
          & & & & \imp{P\I_\B} & \\
        \end{bNiceArray}
      .\qedhere\]%
  \end{enumerate}
\end{solution}

% section coordinate_systems (end)

\section{Change of Basis}
\label{sec:change_of_basis}

\begin{definition}[Change of Coordinates Matrix]
  \label{def:change_of_coordinates_matrix}

  The matrix $\underset{\D \leftarrow \B}{P}$ is called the
  \textbf{change-of-coordinates matrix from} $\B$ to $\D$. That is, multiplying
  a vector by $\underset{\D \leftarrow \B}{P}$ converts from $\B$-coordinates to
  $\D$-coordinates.
\end{definition}

Let's look at transforming vectors between two non-standard basis $\B$ and $\D$.

\begin{question}
  \label{qst:transforming_vectors_between_two_non_standard_basis}

  Consider two bases $\B = \{\b_1, \b_2\}$ and $\D = \{\d_1, \d_2\}$ for a
  vector space $V$ such that $\b_1 = 4\d_1 + \d_2$ and $\b_2 = -6\d_1 + \d_2$.
  Suppose $\x$ is a vector in $V$ such that $\x = 3\b_1 + \b_2$. Find the
  $[\x]_\D$.
\end{question}

\begin{solution}
  \label{sol:transforming_vectors_between_two_non_standard_basis}

  First, $\x = 3\b_1 + \b_2$ tells us that $[\x]_\B = (3, 1)$. Recall, the
  transformation $[~]_\D$ is linear, giving us
  \begin{align*}
    [\x]_\D &= [3\b_1 + \b_2]_\D = 3[\b_1]_\D + 1[\b_2]_\D \rightarrow [\x]_\D = 3[\b_1]_D + 1[\b_2]_\D \\
    T(\x) &= T(3\b_1 + \b_2) = 3T(\b_1) + 1T(\b_2) \\
    \textrm{Matrix eq:}~&\phantom{=}
    \begin{bNiceMatrix}[columns-width=auto]
      [\b_1]_\D & [\b_2]_\D \\
    \end{bNiceMatrix}
    \begin{bNiceMatrix}[columns-width=auto]
      3 \\
      1 \\
    \end{bNiceMatrix} =
    [\x]_\D \quad \textrm{Notice:}~
    \begin{array}{l}
      \b_1 = 4\d_1 + \d_2 \rightarrow [\b_1]_\D = (4, 1) \\
      \b_2 = -6\d_1 + \d_2 \rightarrow [\b_2]_\D = (-6, 1)
    \end{array} \\
    &\phantom{=}\underset{\imp{\underset{\D \leftarrow \B}{P}}}{
      \begin{bNiceMatrix}[columns-width=auto]
        4 & -6 \\
        1 & 1 \\
      \end{bNiceMatrix}
    }
    \underset{\imp{[\x]_\B}}{
      \begin{bNiceMatrix}[columns-width=auto]
        3 \\
        1 \\
      \end{bNiceMatrix}
    } =
    3\begin{bNiceMatrix}[columns-width=auto]
      4 \\
      1 \\
    \end{bNiceMatrix} +
    1\begin{bNiceMatrix}[columns-width=auto]
      -6 \\
      1 \\
    \end{bNiceMatrix} =
    \underset{\imp{[\x]_\D}}{
      \begin{bNiceMatrix}[columns-width=auto]
        6 \\
        4 \\
      \end{bNiceMatrix}
    }
  .\qedhere\end{align*}
\end{solution}

\begin{theorem}
  \label{thm:unique_invertible_matrix_between_two_bases}

  Let $\B = \{\b_1, \b_2, \ldots, \b_n\}$ and $\D = \{\d_1, \d_2, \ldots,
  \d_n\}$ be bases of a vector space $V$. Then there is a unique $n \times n$
  invertible matrix $\underset{\D \leftarrow \B}{P}$ such that
  \[%
    [\x]_\D = \underset{\D \leftarrow \B}{P}[\x]_\B \qtq{where} \underset{\D \leftarrow \B}{P} = \left[~[\b_1]_\D~[\b_2]_\D~\cdots~[\b_n]_\D~\right]
  .\]%
\end{theorem}

Since $\underset{\D \leftarrow \B}{P}$ is the change-of-coordinates from $\B$ to
$\D$ and this matrix is invertible, then we have
\[%
  \underset{\D \leftarrow \B}{P}\I[\x]_\D = [\x]_\B
,\]%
which shows that $\underset{\D \leftarrow \B}{P}\I$ is the change-of-coordinates
matrix from $\D$ to $\B$, giving us
\[%
  \underset{\D \leftarrow \B}{P}\I = \underset{\B \leftarrow \D}{P}
.\]%

Given two bases $\B = \{\b_1, \b_2, \ldots, \b_n\}$ and $\D = \{\d_1, \d_2,
\ldots, \d_n\}$ of a vector space $V$, then we can use row reduction to get the
change-of-coordinates matrix from $\B$ to $\D$ as shown below.
\[%
  [\,\d_1~\cdots~\d_n~\vert~\b_1~\cdots~\b_n\,] \rightarrow \left[\,I~\underset{\D \leftarrow \B}{P}\,\right]
.\]%
Also, we can get the change-of-coordinates matrix from $\D$ to $\B$ with the row
reduction below.
\[%
  [\,\b_1~\cdots~\b_n~\vert~\d_1~\cdots~\d_n\,] \rightarrow \left[\,I~\underset{\B \leftarrow \D}{P}\,\right]
.\]%

\begin{question}
  \label{qst:change_of_basis}

  Consider the following vectors
  \[%
    \b_1 = \begin{bNiceMatrix}[columns-width=auto]
      -9 \\
      1 \\
    \end{bNiceMatrix},\quad
    \b_2 = \begin{bNiceMatrix}[columns-width=auto]
      -5 \\
      -1 \\
    \end{bNiceMatrix},\quad
    \d_1 = \begin{bNiceMatrix}[columns-width=auto]
      1 \\
      -4 \\
    \end{bNiceMatrix},\aand
    \d_2 = \begin{bNiceMatrix}[columns-width=auto]
      3 \\
      -5 \\
    \end{bNiceMatrix}
  .\]%

  Let $\B = \{\b_1, \b_2\}$ and $\D = \{\d_1, \d_2\}$ be bases for $\R^2$.
  \begin{enumerate}
    \label{enum:change_of_basis_qst}

    \item Find the change-of-coordinates matrix from $\B$ to $\D$.

    \item Use your answer in part $1$ to find the change-of-coordinates matrix
      from $\D$ to $\B$.
  \end{enumerate}
\end{question}

\begin{solution}
  \label{sol:change_of_basis} $ $

  \begin{enumerate}
    \label{enum:change_of_basis_sol}

    \item To find $\underset{D \leftarrow B}{P}$, we setup
      $[\,\d_1~\d_2~\vert~\b_1~\b_2\,]$ and row reduce, giving us
      \begin{align*}
        \begin{bNiceArray}{cc|cc}
          1 & 3 & -9 & -5 \\
          -4 & -5 & 1 & -1 \\
        \end{bNiceArray} \rref
        \begin{bNiceArray}{cc|cc}
          1 & 0 & 6 & 4 \\
          0 & 1 & -5 & -3 \\
        \end{bNiceArray} \rightarrow
        \underset{D \leftarrow B}{P} =
        \begin{bNiceMatrix}[columns-width=auto]
          6 & 4 \\
          -5 & -3 \\
        \end{bNiceMatrix}
      .\end{align*}

    \item To find $\underset{B \leftarrow D}{P}$, we find the determinant of
      $\underset{D \leftarrow B}{P}$ and divide by it, giving us
      \[%
        \underset{B \leftarrow D}{P} = \underset{D \leftarrow B}{P}\I = \frac{1}{(6)(-3) - (4)(-5)}
        \begin{bNiceMatrix}[columns-width=auto]
          -3 & -4 \\
          5 & 6 \\
        \end{bNiceMatrix} =
        \frac{1}{2}\begin{bNiceMatrix}[columns-width=auto]
          -3 & -4 \\
          5 & 6 \\
        \end{bNiceMatrix} =
        \begin{bNiceMatrix}[columns-width=auto]
          -\sfrac{3}{2} & -2 \\
          \sfrac{5}{2} & 3 \\
        \end{bNiceMatrix}
      .\qedhere\]%
  \end{enumerate}
\end{solution}

\begin{question}
  \label{qst:}

  Let $\X = \{\x_1, \x_2, \x_3\}$ and $\Y = \{\y_1, \y_2, \y_3\}$ be bases for a
  vector space $V$. Suppose that $\y_1 = 4\x_1 - \x_2$, $\y_2 = \x_1 + 2\x_2 -
  4\x_3$, and $\y_3 = -3\x_2 + 3\x_3$.
  \begin{enumerate}
    \label{enum:_qst}

    \item Find the coordinate vectors $[\y_1]_\X$, $[\y_2]_\X$, and $[\y_3]_\X$.
      Then use these coordinate vectors to write the change-of-coordinates
      matrix from $\Y$ to $\X$.

    \item Use your answer in part $1$ to find $[\v]_\X$ for $\v = 5\y_1 - 2\y_2
      + \y_3$. \textit{Use calculator.}

    \item Use your answer in part $1$ to find the change-in-coordinates matrix
      from $\X$ to $\Y$. \textit{Use calculator.}

    \item Suppose $[\w]_\X = (2,0,2)$. Use your answer in part $3$ to find
      $[\w]_\Y$. \textit{Use calculator.}
  \end{enumerate}
\end{question}

\begin{solution}
  \label{sol:} $ $

  \begin{enumerate}
    \label{enum:_sol}

    \item Notice
      \[%
        \begin{rcases*}
          \begin{aligned}
            \y_1 &= 4\x_1 - \x_2 &&\rightarrow [\y_1]_\X = \langle 4, 1, 0 \rangle \\
            \y_2 &= \x_1 + 2\x_2 - 4\x_3 &&\rightarrow [\y_2]_\X = \langle 1, 2, -4 \rangle \\
            \y_3 &= -3\x_2 + 3\x_3 &&\rightarrow [\y_3]_\X = \langle 0, -3, 3 \rangle \\
          \end{aligned}
        \end{rcases*}
        \underset{\X \leftarrow \Y}{P} =
        \begin{bNiceMatrix}[columns-width=auto]
          4 & 1 & 0 \\
          -1 & 2 & -4 \\
          0 & -3 & 3 \\
        \end{bNiceMatrix}
      .\]%

    \item We know that $\v = 5\y_1 - 2\y_2 + \y_3 \rightarrow [\v]_\Y = (5, -2,
      1)$. Thus,
      \[%
        [\v]_\X = \underset{\X \leftarrow \Y}{P} =
        \begin{bNiceMatrix}[columns-width=auto]
          4 & 1 & 0 \\
          -1 & 2 & -4 \\
          0 & -3 & 3 \\
        \end{bNiceMatrix}
        \begin{bNiceMatrix}[columns-width=auto]
          5 \\
          -2 \\
          1 \\
        \end{bNiceMatrix} =
        \begin{bNiceMatrix}[columns-width=auto]
          18 \\
          -12 \\
          11 \\
        \end{bNiceMatrix}
      .\]%

    \item $\begin{bNiceArray}{ccc|ccc}
        4 & 1 & 0 & 1 & 0 & 0 \\
        -1 & 2 & -4 & 0 & 1 & 0 \\
        0 & -3 & 3 & 0 & 0 & 1 \\
      \end{bNiceArray} \rref
      \begin{bNiceArray}{ccc|ccc}[last-row]
        \CodeBefore
        \tikz \draw[thick,main] ([xshift=1mm]1-|4) rectangle ([xshift=-2.2mm]last-|last);
        \Body
        \circled{1} & 0 & 0 & \sfrac{2}{7} & \sfrac{1}{7} & \sfrac{1}{7} ~ \\
        0 & \circled{1} & 0 & -\sfrac{1}{7} & -\sfrac{4}{7} & -\sfrac{4}{7} ~ \\
        0 & 0 & \circled{1} & \sfrac{4}{21} & -\sfrac{16}{21} & -\sfrac{3}{7} ~ \\
        & & & & \imp{\underset{\Y \leftarrow \X}{P}} & \\
      \end{bNiceArray}$.

    \item $[\w]_\Y = \underset{\Y \leftarrow \X}{P} [\w]_\X
      \begin{bNiceMatrix}[columns-width=auto]
        \sfrac{2}{7} & \sfrac{1}{7} & \sfrac{1}{7} \\
        -\sfrac{1}{7} & -\sfrac{4}{7} & -\sfrac{4}{7} \\
        \sfrac{4}{21} & -\sfrac{16}{21} & -\sfrac{3}{7}
      \end{bNiceMatrix}
      \begin{bNiceMatrix}[columns-width=auto]
        2 \\
        0 \\
        2 \\
      \end{bNiceMatrix} =
      \begin{bNiceMatrix}[columns-width=auto]
        \sfrac{6}{7} \\
        -\sfrac{10}{7} \\
        \sfrac{26}{21} \\
      \end{bNiceMatrix}$. \qedhere
  \end{enumerate}
\end{solution}

% section change_of_basis (end)

\newpage
