\nte[Section 1.5]{Oct 10 2023 Tue (14:00:26)}{Solution Sets and More Application}

\section{Homogeneous Linear Systems}
\label{sec:homogeneous_linear_systems}

\begin{definition}[Homogeneous Linear Systems]
  \label{def:homogeneous_linear_systems}

  $A$ linear system is called \textbf{homogeneous} if it can be written in the
  form $A\x = \zero$ where $A$ is the $m \times n$ coefficient matrix.
\end{definition}

This system always has at least one solution, which is $\x = \zero$, which is
called the \textbf{trivial solution}. The important question is whether the
linear system has a \textbf{nontrivial solution}.

\begin{purpleframe}
  \label{prpl:homogeneous_linear_systems}

  The homogeneous equation $A\x = \zero$ has a nontrivial solution if and only
  if the equation has at least one free variable. Hence, we need only row reduce
  the augmented matrix $[\,A~\vert~\zero\,]$, into echelon form and check if
  there are any non-pivot columns.
\end{purpleframe}

\begin{question}
  \label{qst:homogeneous_linear_systems}

  Determine if the following homogeneous system has a nontrivial solution. Then
  describe the solution set.
  \[%
    \sysdelim..\systeme{
      3x_1 + 5x_2 - 4x_3 = 0,
      -3x_1 - 2x_2 + 4x_3 = 0,
      6x_1 + x_2 - 8x_3 = 0
    }
  .\]%
\end{question}

\begin{solution}
  \label{sol:homogeneous_linear_systems}

  Let A be the matrix of coefficients of the system and row reduce the
  augmented matrix $[\,A~\vert~\zero\,]$ to echelon form
  \begin{align*}
    \sysdelim..\systeme{
      R_1 + R_2 \rightarrow R_2,
      -2R_2 + R_3 \rightarrow R_3
    } &\longleftrightarrow
    \begin{bNiceArray}{ccc|c}[columns-width=auto]
      \circled{-3} & 5 & -4 & 0 \\
      0 & \circled{-3} & 0 & 0 \\
      0 & -9 & 0 & 0 \\
    \end{bNiceArray} \\
    \sysdelim..\systeme{
      3R_2 + R_3 \rightarrow R_3
    } &\longleftrightarrow
    \begin{bNiceArray}{ccc|c}[columns-width=auto]
      \circled{-3} & 5 & -4 & 0 \\
      0 & \circled{-3} & 0 & 0 \\
      0 & 0 & 0 & 0 \\
    \end{bNiceArray}
  .\end{align*}

  Since $x_3$ is a free variable, $A\x = \zero$ has nontrivial solutions (one for
  each nonzero choice of $x_3$). To describe the solution set, we need to
  continue the row reduction of $[\,A~\vert~\zero\,]$ to reduced echelon form
  \[%
    \begin{bNiceArray}{ccc|c}[columns-width=auto]
      1 & 0 & -\sfrac{4}{3} & 0 \\
      0 & 1 & 0 & 0 \\
      0 & 0 & 0 & 0 \\
    \end{bNiceArray} \longleftrightarrow
    \sysdelim..\systeme*{
      x_1 - \sfrac{4}{3}x_3 = 0,
      x_2 = 0,
      0 = 0,
    } \longleftrightarrow
    \sysdelim..\systeme*{
      x_1 = \sfrac{4}{3}x_3,
      x_2 = 0,
      0 = 0,
    }
  .\]%

  As a vector, the general solution of $[\,A~\vert~\zero\,]$ has the form
  \begin{equation}\label{eqt:parametric_vector_form}
    \x = \begin{bNiceMatrix}[columns-width=auto]
      x_1 \\
      x_2 \\
      x_3 \\
    \end{bNiceMatrix}
    \begin{bNiceMatrix}[columns-width=auto]
      \sfrac{4}{3}x_3 \\
      0 \\
      x_3 \\
    \end{bNiceMatrix} =
    x_3\begin{bNiceMatrix}[columns-width=auto]
      \sfrac{4}{3} \\
      0 \\
      1 \\
    \end{bNiceMatrix} = x_3\v
  ,\end{equation}
  where
  \[%
    \v = \begin{bNiceMatrix}[columns-width=auto]
      \sfrac{4}{3} \\
      0 \\
      1 \\
    \end{bNiceMatrix}
  .\qedhere\]%
\end{solution}

\subsection{Parametric Vector Form}
\label{sub_sec:parametric_vector_form}

The solution found in \cref{eqt:parametric_vector_form} is called
\textbf{parametric vector form} of the solution where $x_3$ is the parameter.
\begin{example}
  \label{exm:parametric_vector_form}

  Here's another more complicated example of the solution in parametric vector
  form
  \[%
    \v = \begin{bNiceMatrix}[columns-width=auto]
      -4x_1 + 3x_2 \\
      x_3 \\
      -3x_3 - x_1 \\
    \end{bNiceMatrix}
    = x_1\begin{bNiceMatrix}[columns-width=auto]
      -4 \\
      0 \\
      -1 \\
    \end{bNiceMatrix}
    +
    x_2\begin{bNiceMatrix}[columns-width=auto]
      3 \\
      0 \\
      0 \\
    \end{bNiceMatrix}
    +
    x_3\begin{bNiceMatrix}[columns-width=auto]
      0 \\
      1 \\
      -3 \\
    \end{bNiceMatrix}
  .\]%
\end{example}

% subsection parametric_vector_form (end)

\begin{question}
  \label{qst:parametric_vector_form}

  Find the solution of $A\x = \zero$ in parametric vector form where
  \[%
    A = \begin{bNiceMatrix}[columns-width=auto]
      2 & -6 & 14 & 6 \\
      0 & -9 & 9 & 5 \\
    \end{bNiceMatrix}
  .\]%
\end{question}

\begin{solution}
  \label{sol:parametric_vector_form}

  First, convert everything to an augmented matrix
  \[%
    \begin{bNiceArray}{c|c}[columns-width=auto]
      A & \zero \\
    \end{bNiceArray} =
    \begin{bNiceArray}{cccc|c}[columns-width=auto]
      2 & -6 & 14 & 6 & 0 \\
      0 & -9 & 9 & 5 & 0 \\
    \end{bNiceArray}
    \rref
    \begin{bNiceArray}{cccc|c}[columns-width=auto]
      \circled{1} & 0 & 4 & \sfrac{4}{3} & 0 \\
      0 & \circled{1} & -1 & -\sfrac{5}{9} & 0 \\
    \end{bNiceArray}
    \longleftrightarrow
    \sysdelim..\systeme*{
      x_1 + 4x_3 + \sfrac{4}{3}x_4 = 0,
      x_2 - x_3 - \sfrac{5}{9}x_4 = 0,
      x_3 = x_3,
      x_4 = x_4
    }
  .\]%

  We can then convert it to the general solution, and from there, convert it to
  the parametric vector form
  \begin{align*}
    \sysdelim..\systeme*{
      x_1 = -4x_3 - \sfrac{4}{3}x_4,
      x_2 = x_3 + \sfrac{5}{9}x_4,
      x_3 = x_3,
      x_4 = x_4
    } \longleftrightarrow
    \x &= \begin{bNiceMatrix}[columns-width=auto]
      x_1 \\
      x_2 \\
      x_3 \\
      x_4 \\
    \end{bNiceMatrix} =
    \begin{bNiceMatrix}[columns-width=auto]
      -4x_3 - \sfrac{4}{3}x_4 \\
      x_3 + \sfrac{5}{9}x_4 \\
      x_3 \\
      x_4 \\
    \end{bNiceMatrix} =
    \begin{bNiceMatrix}[columns-width=auto]
      -4x_3 \\
      x_3 \\
      x_3 \\
      0 \\
    \end{bNiceMatrix} +
    \begin{bNiceMatrix}[columns-width=auto]
      -\sfrac{4}{3}x_4 \\
      \sfrac{5}{9}x_4 \\
      0 \\
      x_4 \\
    \end{bNiceMatrix} \\
    &= x_3\begin{bNiceMatrix}[columns-width=auto]
      -4 \\
      1 \\
      1 \\
      0 \\
    \end{bNiceMatrix} +
    x_4\begin{bNiceMatrix}[columns-width=auto]
      -\sfrac{4}{3} \\
      \sfrac{5}{9} \\
      0 \\
      1 \\
    \end{bNiceMatrix}
  .\qedhere\end{align*}
\end{solution}

Given vectors $\u$ and $\w$, the line $L$ that is parallel to $\u$ and passes
through $\w$ is given by
\[%
  \v(r) = r\u + \w
,\]%
for any parameter $r$. Also, the line $M$ that passes through both $\u$ and
$\w$ is given by
\[%
  \v(r) = r(\w - \u) + \u
,\]%
for any parameter $r$.

\begin{note}
  \label{nte:each_line_treated_as_a_vector_function}

  Each line is treated as a vector function $\v(r)$ where different inputs of
  $r$ will result in different vectors (or points) on the lines.
\end{note}

\begin{question}
  \label{qst:parallel_and_passes_through_line_with_vectors}

  Consider vectors $\u = (4, 3)$ and $\w = (-1, 4)$. Find the line $L$ that is
  parallel to $\u$ and passes through $\w$. Find the line $M$ that passes
  through both $\u$ and $\w$.
\end{question}

\begin{solution}
  \label{sol:parallel_and_passes_through_line_with_vectors}

  The line $L$ that is parallel to $\u$ and passes through $\w$ is given by
  \[%
    \v(r) = r\u + \w = r
    \begin{bNiceMatrix}[columns-width=auto]
      4 \\
      -3 \\
    \end{bNiceMatrix} +
    \begin{bNiceMatrix}[columns-width=auto]
      -1 \\
      4 \\
    \end{bNiceMatrix}
  ,\]%
  and the line $M$ that passes through both $\u$ and $\w$ is given by
  \[%
    \v(r)=r(\w-\u)+\u=
    r\left(
      \begin{bNiceMatrix}[columns-width=auto]
        4 \\
        -3 \\
      \end{bNiceMatrix} -
      \begin{bNiceMatrix}[columns-width=auto]
        -1 \\
        4 \\
      \end{bNiceMatrix}
    \right) +
    \begin{bNiceMatrix}[columns-width=auto]
      4 \\
      -3 \\
    \end{bNiceMatrix} = s
    \begin{bNiceMatrix}[columns-width=auto]
      5 \\
      -7 \\
    \end{bNiceMatrix} +
    \begin{bNiceMatrix}[columns-width=auto]
      4 \\
      -3 \\
    \end{bNiceMatrix}
  .\]%
  And we can also graph the vectors $\u$ and $\w$ along with the lines $L$ and
  $M$ as shown below.
  \begin{figure}[H]
    \centering

    \begin{tikzpicture}
      \begin{axis}[
        scale=0.85,
        xmin=-5.75, xmax=5.75,
        ymin=-5.75, ymax=5.75,
        ]

        \addplot coordinates{(0,0)(4,-3)} node[below left]{$\u$};
        \addplot coordinates{(0,0)(-1,4)} node[below left]{$\w$};
        \addplot+[latex-latex,domain=-3.25:5.75,mark=none]{(-3/4)*(x+1)+4} node[midway,above right]{$L$};
        \addplot+[latex-latex,domain=-2.25:5.75,mark=none]{(-7/5)*(x+1)+4} node[pos=0.70,above right]{$M$};
      \end{axis}
    \end{tikzpicture}

    \caption{}
    \label{fig:parallel_and_passes_through_line_with_vectors}
  \end{figure}
\end{solution}

% section homogeneous_linear_systems (end)

\section{Solutions of Nonhomogeneous Systems}
\label{sec:solutions_of_nonhomogeneous_systems}

When a nonhomogeneous linear system has many solutions, the general solution
can be written in parametric vector form as one vector plus an arbitrary linear
combination of vectors that satisfy the corresponding homogeneous system.

\begin{theorem}
  \label{thm:solution_in_parametric_vector_form}

  The solution for $A\x = \b$ in parametric vector form can be written as $\v_h
  + \p$ where $\p$ is a particular solution (i.e. $A\p = \b$) and $\v_h$ is any
  solution of $A\x = \zero$.
\end{theorem}

\Cref{thm:solution_in_parametric_vector_form} says that $A\x = \b$ has a
solution, then the solution set is obtained by translating the solution set of
$A\x = \b$, using any particular solution $\p$ of $A\x = \b$ for the
translation. Even when $n > 3$, our mental image of the solution set of a
consistent system $A\x = \b$ (with $\b \ne 0$) is either a single nonzero point
or a line or plane not passing through the origin.

\begin{note}
  \label{nte:solution_in_parametric_vector_form}

  \Cref{thm:solution_in_parametric_vector_form} applies only to an equation
  $A\x = \b$ that has at least one nonzero solution $\p$. When $A\x = \b$ has
  no solution, the solution set is empty.
\end{note}

% section solutions_of_nonhomogeneous_systems (end)

\section{More Applications}
\label{sec:more_applications}

\subsection{Candy}
\label{sub_sec:candy}

\begin{question}
  \label{qst:candy}

  A candy company sells three types of chocolate boxes as shown below that each
  contain a combination of three types of chocolate pieces (circle, triangle,
  and square chocolates).

  \begin{align*}
    \underset{\displaystyle\$43}{\begin{array}{cc}
      \square & \bigcirc \\
      \bigtriangleup & \bigtriangleup \\
    \end{array}}\qquad
    \underset{\displaystyle\$45}{\begin{array}{cc}
      \square & \bigcirc \\
      \bigcirc & \bigtriangleup \\
    \end{array}}\qquad
    \underset{\displaystyle\$42}{\begin{array}{cc}
      \square & \square \\
      \bigtriangleup & \bigtriangleup \\
    \end{array}}\qquad
    \underset{\displaystyle?}{\begin{array}{ccc}
      \square & \bigtriangleup & \bigcirc \\
      \square & \bigtriangleup & \bigcirc \\
    \end{array}}
  \end{align*}

  What should the price be for the six piece box?
\end{question}

\begin{solution}
  \label{sol:candy}

  Let $P_C$, $P_T$, and $P_S$ be the prices of the circle, triangle, and square
  pieces.

  \begin{align*}
    \begin{array}{l}
      \textrm{Box $1$:} \  1P_S + 1P_C + 2P_T = 43 \\
      \textrm{Box $2$:} \  1P_S + 2P_C + 1P_T = 45 \\
      \textrm{Box $3$:} \  2P_S + 0P_C + 2P_T = 42 \\
    \end{array}
    \longleftrightarrow
    \begin{bNiceArray}{ccc|c}[columns-width=auto]
      1 & 1 & 2 & 43 \\
      1 & 2 & 1 & 45 \\
      2 & 0 & 2 & 42 \\
    \end{bNiceArray}
    \rref
    \begin{bNiceArray}{ccc|c}[columns-width=auto]
      \circled{1} & 0 & 0 & 11 \\
      0 & \circled{1} & 0 & 12 \\
      0 & 0 & \circled{1} & 10 \\
    \end{bNiceArray}
  .\end{align*}

  This gives us the following answers
  \[%
    \sysdelim..\systeme*{
      P_S = \$11,
      P_C = \$12,
      P_T = \$10,
    } \implies
    2P_S + 2P_C + 2P_T = 2(11) + 2(12) + 2(10) = 66
  .\qedhere\]%
\end{solution}

% subsection candy (end)

\subsection{Dealing with Derivatives}
\label{sub_sec:dealing_with_derivatives}

\begin{question}
  \label{qst:dealing_with_derivatives}

  Consider the function $f(x) = ae^{-3x} + be^{-2x} + cxe^{-x}$, where $a$,
  $b$, and $c$ are constants.

  \begin{enumerate}
    \label{enum:dealing_with_derivatives_qst}

    \item Find the derivatives $f^{\prime}$ and $f^{\prime\prime}$.

    \item Suppose that we know the initial conditions $f(0) = -3$,
      $f^{\prime}(0) = 5$, and $f^{\prime\prime}(0) = 0$. Using your answers in
      part $1$, construct a linear system by finding a linear equation for each
      initial condition.

    \item Find the values of $a$, $b$, and $c$ by solving for the linear system
      found in part $2$. \textit{You may use the row reducing function on your
      calculator.}

    \item Using your answers in part $3$, find the value of $f(1)$.
  \end{enumerate}
\end{question}

\begin{solution}
  \label{sol:dealing_with_derivatives}

  \begin{enumerate}
    \label{enum:dealing_with_derivatives_sol}

    \item To find the first and second derivatives, we must use the product
      rule, which states
      \[%
        \odv{}{x}[f \cdot g] = f^{\prime} \cdot g + g^{\prime} \cdot f
      ,\]%
      and the chain rule, which states that
      \[%
        \odv{}{x}[f(g(x))] = f^{\prime}(g(x)) \cdot g^{\prime}(x)
      .\]%

      Using that, we can find the first derivative
      \begin{align*}
        \odv{}{x}[f(x)] &= \odv{}{x}\left[ae^{-3x} + be^{-2x} + cxe^{-x}\right] \\
                        &= \odv{}{x}[ae^{-3x}] + \odv{}{x}[be^{-2x}] + \odv{}{x}[cxe^{-x}] \\
                        &= -3ae^{-3x} - 2be^{-2x} + \odv{}{x}[cx] \cdot e^{-x} + \odv{}{x}[e^{-x}] \cdot cx \\
                        &= -3ae^{-3x} - 2be^{-2x} + ce^{-x} - cxe^{-x}
      .\end{align*}

      We can then do the same thing for the second derivative
      \begin{align*}
        \odv[order={2,2}]{}{x}[f(x)] &= \odv{}{x}\left[\odv{}{x}\left[ae^{-3x} + be^{-2x} + cxe^{-x}\right]\right] \\
                                     &= \odv{}{x}[-3ae^{-3x} - 2be^{-2x} + ce^{-x} - cxe^{-x}] \\
                                     &= \odv{}{x}[-3ae^{-3x}] + \odv{}{x}[-2be^{-2x}] + \odv{}{x}[ce^{-x}] - \odv{}{x}[cxe^{-x}] \\
                                     &= 9ae^{-3x} + 4be^{-2x} - ce^{-x} - \left(\odv{}{x}[cx] \cdot e^{-x} + \odv{}{x}[e^{-x}] \cdot cx\right) \\
                                     &= 9ae^{-3x} + 4be^{-2x} - ce^{-x} - (ce^{-x} - cxe^{-x}) \\
                                     &= 9ae^{-3x} + 4be^{-2x} - 2ce^{-x} + cxe^{-x})
      .\end{align*}

    \item We can then use our previous answer to construct our system of
      equations
      \begin{alignat*}{10}
        f(0) &= ae^{-3(0)} + be^{-2(0)} + c(0)e^{-0} &&= 3 &&\implies a + b &&= -3 \\
        f^{\prime}(0) &= -3ae^{-3(0)} - 2be^{-2(0)} + ce^{-(0)} - c(0)e^{-(0)} &&= 5 &&\implies -3a - 2b + c &&= 5 \\
        f^{\prime\prime}(0) &= 9ae^{-3(0)} + 4be^{-2(0)} - 2ce^{-(0)} + c(0)e^{-(0)} &&= 0 &&\implies 9a + 4b - 2c &&= 0
      .\end{alignat*}

    \item We can then convert our previous answer to a linear system and solve
      it using the Row Reduce Algorithm
      \[%
        \begin{bNiceArray}{ccc|c}[columns-width=auto]
          \circled{1} & 1 & 0 & -3 \\
          -3 & -2 & 1 & 5 \\
          9 & 4 & -2 & 0 \\
        \end{bNiceArray}
        \rref
        \begin{bNiceArray}{ccc|c}[columns-width=auto]
          \circled{1} & 0 & 0 & \sfrac{10}{3} \\
          0 & \circled{1} & 0 & -\sfrac{19}{3} \\
          0 & 0 & \circled{1} & \sfrac{7}{3} \\
        \end{bNiceArray}
        \implies
        \sysdelim..\systeme*{
          a = \sfrac{10}{3},
          b = -\sfrac{19}{3},
          c = \sfrac{7}{3}
        }
      .\]%

    \item Using all the information that we've collected, when can then find
      $f(1)$, which is
      \[%
        f(1) = \frac{10}{3}e^{-3(1)} - \frac{19}{3}e^{-2(1)} + \frac{7}{3}e^{-(1)} \approx 0.167
      .\qedhere\]%
  \end{enumerate}
\end{solution}

% subsection dealing_with_derivatives (end)

% section more_applications (end)

\newpage
