%%%%%%%%%%%%%%%%%%%%%%%%%%%%%%%%%%%%%%%%%%%%%%%%%%%%%%%%%%%%%%%%%%%%%%%%%%%%%%%%
%                                                                              %
%                              Required Packages                               %
%                                                                              %
%%%%%%%%%%%%%%%%%%%%%%%%%%%%%%%%%%%%%%%%%%%%%%%%%%%%%%%%%%%%%%%%%%%%%%%%%%%%%%%%

% Required for creating documents
\usepackage[utf8]{inputenc}
\usepackage[T1]{fontenc}

% Required math packages
\usepackage{amsmath}
\usepackage{amsfonts}
\usepackage{mathtools}
\usepackage{amsthm}
\usepackage{amssymb}

\usepackage{multicol} % for multiple columns
\usepackage[usenames,dvipsnames,pdftex]{xcolor} % Required for nicer colors
\usepackage{hyperref} % Required for hyperlinks
\usepackage{xparse} % Required for \NewDocumentCommand
\usepackage{graphicx} % Required for including images
\usepackage{enumitem} % Required for customizing lists
\usepackage{float} % Required for positioning figures and tables
\usepackage{array} % Required for customizing tables
\usepackage{systeme} % Required for \systeme
\usepackage{cancel} % Required for \cancel
\usepackage{derivative} % Required for \odv and \pdv
\usepackage{authoraftertitle} % Required for \MyTitle
\usepackage{geometry} % Required for adjusting page dimensions
\usepackage{minitoc} % Required for creating a mini table of contents
\usepackage[noabbrev]{cleveref} % Required for clever referencing
\usepackage{emptypage} % Required for removing page numbers on empty pages
\usepackage{empheq} % Required for boxing equations
\usepackage{nicematrix} % Required for better matrices
\usepackage{booktabs} % Required for better tables
\usepackage{cellspace} % Required for better spacing in tables
\usepackage{longtable} % Required for long tables
\usepackage{xfrac} % Required for extra fraction options

\usepackage{tasks} % Required for creating tasks
\usepackage[font=bf]{caption} % Required for customizing captions
\usepackage{siunitx} % Required for SI units
\usepackage{titletoc} % Required for customizing table of contents
\usepackage[framemethod=TikZ]{mdframed} % Required for creating boxes
\usepackage{braket} % Required for creating braket notation

% Required for drawing figures
\usepackage{tikz}
\usepackage{tkz-euclide}
\usepackage{tikz-cd}

% Required for creating plots
\usepackage{pgfplots}
\usepackage{pgfplotstable}

\usepackage{titling} % Required for customizing title page
\usepackage{ifthen} % Required for if-then-else statements
\usepackage{xifthen} % Required for if-then-else statements
\usepackage{fancyhdr} % Required for customizing headers and footers
\usepackage{import} % Required for importing pdf_tex files
\usepackage{titlesec} % Required for customizing sectioning commands

% Required for creating boxes
\usepackage{varwidth}
\usepackage{thmtools}
\usepackage{etoolbox}
\usepackage[most,many,breakable]{tcolorbox}


%%%%%%%%%%%%%%%%%%%%%%%%%%%%%%%%%%%%%%%%%%%%%%%%%%%%%%%%%%%%%%%%%%%%%%%%%%%%%%%%
%                                                                              %
%                                Basic Settings                                %
%                                                                              %
%%%%%%%%%%%%%%%%%%%%%%%%%%%%%%%%%%%%%%%%%%%%%%%%%%%%%%%%%%%%%%%%%%%%%%%%%%%%%%%%

% Define Colors
\makeatletter

\newcommand\newcolor[3]{
  \@ifclasswith\class{nocolor}{
    \definecolor{#1}{HTML}{#3}
  }{
    \definecolor{#1}{HTML}{#2}
  }
}

\makeatother

\newcolor{linecolor1}{3DC6F3}{000000}
\newcolor{linecolor2}{F034A3}{000000}
\newcolor{linecolor3}{F57215}{000000}
\newcolor{linecolor4}{733786}{000000}
\newcolor{linecolor5}{80CF5C}{000000}
\newcolor{correct}{00FF00}{00FF00}
\newcolor{incorrect}{FF0000}{FF0000}
\newcolor{tbl}{F035A3}{000000}
\newcolor{main}{F035A3}{000000}
\newcolor{solution}{00AEEF}{000000}
\newcolor{qedsolution}{76B4CF}{000000}
\newcolor{proof}{F035A3}{000000}
\newcolor{subsubsection}{B00D15}{000000}
\newcolor{example}{00A6E4}{000000}
\newcolor{examplebg}{F2FBF8}{FFFFFF}

\colorlet{definition}{main}
\colorlet{theorem}{main}
\colorlet{remark}{main}
\colorlet{qedproof}{proof}
\colorlet{question}{example}
\colorlet{questionfg}{examplebg}

% Geometry
\geometry{
  tmargin=2cm,
  rmargin=1in,
  lmargin=1in,
  margin=0.85in,
  bmargin=2cm,
  footskip=1cm,
}

% Tasks
\settasks{label=\bfseries\arabic*.),label-width=2em}

% Tables
\newcolumntype{C}{>{\Centering\arraybackslash}X}
\setlength{\tabcolsep}{5pt}
\renewcommand\arraystretch{1.5}
\renewcommand\thetable{\Roman{table}}
\captionsetup[figure]{font=small}
\captionsetup{justification=centering}
\setlength\cellspacetoplimit{6pt}
\setlength\cellspacebottomlimit{6pt}

% Symbols
\allowdisplaybreaks
\let\svlim\lim\def\lim{\svlim\limits}
\let\svsum\sum\def\sum{\svsum\limits}

% Lists
\setlist[itemize,1]{label=--}
\setlist[itemize,2]{label=\textbullet}
\setlist[enumerate,1]{label=\protect\circled{\arabic*}}

% SI Unitx
\sisetup{
  locale = US,
  per-mode = symbol,
  propagate-math-font = true,
  reset-math-version = false,
  exponent-mode = engineering,
  round-mode = figures,
  round-precision = 3,
  drop-zero-decimal,
}

\DeclareSIUnit{\millimeter}{mm}
\DeclareSIUnit{\centimeter}{cm}
\DeclareSIUnit{\decimeter}{dm}
\DeclareSIUnit{\inch}{in}
\DeclareSIUnit{\foot}{ft}
\DeclareSIUnit{\yard}{yd}
\DeclareSIUnit{\meter}{m}
\DeclareSIUnit{\kilometer}{km}
\DeclareSIUnit{\mile}{mi}
\DeclareSIUnit{\astronomicalunit}{au}
\DeclareSIUnit{\lightyear}{ly}
\DeclareSIUnit{\fahrenheit}{F}
\DeclareSIUnit{\celsius}{C}
\DeclareSIUnit{\millisecond}{ms}
\DeclareSIUnit{\second}{sec}
\DeclareSIUnit{\minute}{min}
\DeclareSIUnit{\hour}{hr}
\DeclareSIUnit{\day}{d}
\DeclareSIUnit{\week}{wk}
\DeclareSIUnit{\month}{mos}
\DeclareSIUnit{\year}{yr}
\DeclareSIUnit{\milligram}{mg}
\DeclareSIUnit{\gram}{g}
\DeclareSIUnit{\ounce}{oz}
\DeclareSIUnit{\pound}{lb}
\DeclareSIUnit{\kilogram}{kg}
\DeclareSIUnit{\ton}{t}
\DeclareSIUnit{\gallon}{gal}
\DeclareSIUnit{\liter}{L}
\DeclareSIUnit{\milliliter}{mL}

% Center Title Page
\renewcommand\maketitlehooka{\null\mbox{}\vfill}
\renewcommand\maketitlehookd{\vfill\null}

% Footnote Line
\renewcommand\footnoterule{\hrule\vspace{0.1cm}}

% Modify Links Color
\hypersetup{
  colorlinks,
  linkcolor=main!90,
  citecolor=black,
  urlcolor=black,
}

% TikZ
\usetikzlibrary{
  intersections,
  angles,
  quotes,
  calc,
  positioning,
  3d,
  arrows,
  arrows.meta,
  patterns,
}

\tikzset{>=stealth}
\tikzset{->-/.style={decoration={markings,mark=at position .5 with {\arrow{>}}},postaction={decorate}}}

% PGF Plots
\pgfplotsset{compat=1.18}

\usepgfplotslibrary{fillbetween}
\usepgfplotslibrary{patchplots}
\usepgfplotslibrary{external}
\usetikzlibrary{decorations.pathreplacing,calligraphy}
\tikzexternalenable

\pgfplotsset{style1/.style={color=linecolor1,mark=none,line width=1pt,solid}}
\pgfplotsset{style2/.style={color=linecolor2,mark=none,line width=1pt,solid}}
\pgfplotsset{style3/.style={color=linecolor3,mark=none,line width=1pt,solid}}
\pgfplotsset{style4/.style={color=linecolor4,mark=none,line width=1pt,solid}}
\pgfplotsset{style5/.style={color=linecolor5,mark=none,line width=1pt,solid}}
\pgfplotsset{asymptote/.style={color=gray,mark=none,line width=1pt,<->,dashed}}
\pgfplotsset{soldot/.style={color=linecolor2,only marks,mark=*}}
\pgfplotsset{holdot/.style={color=linecolor2,fill=white,only marks,mark=*}}
\pgfplotsset{integration/.style={name path=curve,color=linecolor1,mark=none,line width=0.5pt,solid}}

\pgfplotscreateplotcyclelist{stylelist}{
  style1,
  style2,
  style3,
  style4,
  style5,
}

\def\axisdefaultwidth{175pt}
\def\axisdefaultheight{\axisdefaultwidth}
\pgfplotsset{
  every axis/.append style={
    axis x line=middle, x axis line style={name path=xaxis},
    axis y line=middle, y axis line style={name path=yaxis},
    ticks=none,
    axis line style={->},
    xlabel={$x$},
    ylabel={$y$},
    samples=1000,
    cycle list name=stylelist
  },
}


%%%%%%%%%%%%%%%%%%%%%%%%%%%%%%%%%%%%%%%%%%%%%%%%%%%%%%%%%%%%%%%%%%%%%%%%%%%%%%%%
%                                                                              %
%                           School Specific Commands                           %
%                                                                              %
%%%%%%%%%%%%%%%%%%%%%%%%%%%%%%%%%%%%%%%%%%%%%%%%%%%%%%%%%%%%%%%%%%%%%%%%%%%%%%%%

%%%%%%%%%%%%%%%%%%%%%%
%  Helpful Commands  %
%%%%%%%%%%%%%%%%%%%%%%

\makeatletter

\newcommand\resetcounters{
  \setcounter{section}{0}
  \setcounter{subsection}{0}
  \setcounter{subsubsection}{0}
  \setcounter{paragraph}{0}
  \setcounter{subparagraph}{0}
}

\newcommand*\cleartoleftpage{%
  \clearpage
  \thispagestyle{empty}
  \ifodd\value{page}\else\hbox{}\newpage\fi
}

%%%%%%%%%%%%%%%%%%%%%%%%%%%%%
%  Lecture/Chapter Command  %
%%%%%%%%%%%%%%%%%%%%%%%%%%%%%

\def\@notenum{}
\newcommand\includenote[1]{
  \ifnum #1<10
    \def\@notenum{0#1}
  \else
    \def\@notenum{#1}
  \fi

  \setcounter{chapter}{#1}
  \resetcounters
  \IfFileExists{\noteloc-\@notenum.tex}{\input{\noteloc-\@notenum.tex}}{}
}

\newcommand\includenotes[2]{
  \foreach \n in {#1,...,#2}{
    \includenote{\n}
  }
}

\newcommand{\localtoc}{
  \startcontents
  \printcontents{}{1}{\noindent{\color{main}\rule{\textwidth}{0.4pt}\par}\vspace*{-0.3cm}\subsection*{{\color{main}Lecture Note Overview}}}
  \noindent{{\color{main}\rule{\textwidth}{0.4pt}\par}}
}

\newcommand{\removetocentry}[1]{%
  \addtocontents{toc}{\protect\setcounter{tocdepth}{-1}}
  #1
  \addtocontents{toc}{\protect\setcounter{tocdepth}{\arabic{tocdepth}}}
}

\newcommand\customlabel[2]{%
  \protected@write \@auxout {}{\string \newlabel {#1}{{#2}{\thepage}{#2}{#1}{}} }%
  \hypertarget{#1}{}
}

\def\@note{}
\newcounter{nte}
\NewDocumentCommand\nte{O{} O{} m m}{%
  \cleartoleftpage
  \setcounter{nte}{\arabic{chapter}}
  \resetcounters
  \ifthenelse{\isempty{#3}}{%
    \def\@note{\lecorchap~\arabic{chapter}}
  }{%
    \def\@note{\lecorchap~\arabic{chapter}: #4}
  }%
  \def\@notetbl{\lecorchap~\arabic{chapter}}
  \phantomsection\addcontentsline{toc}{chapter}{\protect\numberline{\thechapter}#4}
  \customlabel{note_\@notenum}{\@notenum}
  {\fontsize{10}{12}\selectfont\sffamily\ifstrequal{#1}{}{}{\noindent#1}\ifstrequal{#3}{}{}{\hfill#3}}
  \vspace*{0.03cm}
  \hrule
  \vspace*{0.3cm}
  \noindent{\bfseries\sffamily\fontsize{20}{30}\selectfont\@note}

  \ifstrequal{#2}{false}{}{\localtoc}
  \setcounter{section}{0}
}

% Intro
\newcommand\createintro{
  \maketitle

  \pagenumbering{roman}
  \begin{center}
    \textbf{{\LARGE Introduction}}
  \end{center}

  \IfFileExists{./intro.tex}{Lectures notes from the course \MyTitle, given by professor Ian Nicolas at the \faculty~at \location~in the academic year \academicyear, in the \term. This course covers linear systems, vectors in n-space, vector space properties of n-space, and matrix algebra, including eigenspaces. Credit for the material in these notes are loosely based on professor Ian's notes and from the \href{https://www.pearson.com/en-us/subject-catalog/p/linear-algebra-and-its-applications/P200000006235}{Linear Algebra and Its Applications} textbook. The credit for the typesetting is my own.

\textit{Disclaimer:} This document will inevitably contain some mistakes--both simple typos and legitimate errors. Keep in mind that these are the notes of an undergraduate student in the process of learning the material himself, so take what you read with a grain of salt. If you find mistakes and feel like telling me, I will be grateful and happy to hear from you, even for the most trivial of errors. You can reach me by email, in English, Arabic, Hebrew, or Spanish at \href{mailto:singularisartt@gmail.com}{singularisartt@gmail.com}.
}{}

  \pagestyle{fancy}
  \renewcommand\headrulewidth{0pt}

  \fancyhead{}
  \fancyfoot[C]{%
    \textit{For more notes like this, visit \href{\linktootherpages}{\shortlinkname}}.%
  }%

  \begin{tcolorbox}[enhanced,colback=white,center upper,size=fbox, drop shadow southwest,sharp corners]
    \term: \academicyear, \\
    Last Update: \today, \\
    \faculty, \location.
  \end{tcolorbox}

  \newpage
  \tableofcontents

  \pagenumbering{arabic}
  \setcounter{page}{1}

  \renewcommand\headrulewidth{0.4pt}
  \fancyhead[R]{\@note}
  \fancyhead[L]{\@author}
  \fancyfoot[C]{\thepage}
}

%%%%%%%%%%%%%%%%%%%%%
%  Random Commands  %
%%%%%%%%%%%%%%%%%%%%%

\makeatletter

% Import Figures
\newcommand\incfig[2][1]{
  \def\svgwidth{#1\columnwidth}
  \import{\figloc-\@notenum}{#2.pdf_tex}
}

\makeatother

% Circle
\newcommand*\circled[1]{\tikz[baseline=(char.base)]{
  \node[shape=circle,draw,inner sep=1pt] (char) {#1};}
}

% Correct
\newcommand\correct[1]{\textcolor{correct}{#1}}
\newcommand\incorrect[1]{{\color{incorrect}#1}}
\newcommand\inctocor[2]{\incorrect{#1} \ensuremath{\to} \correct{#2}}

% Important
\newcommand\imp[1]{{\color{main}#1}}

% For diagonal strikeout in blue
\newcommand{\blu}[1]{\phantom{#1}}

% For diagonal strikeout in red
\newcommand{\rcancel}[1]{\renewcommand\CancelColor{\color{red}}\cancel{#1}}

% Helpful text in math
\newcommand\echelon{\underrightarrow{\textrm{ echelon form }}}
\newcommand\rref{\underrightarrow{\textrm{ rref }}}
\newcommand\pick[1]{\xrightarrow[#1]{\textrm{ pick }}}
\newcommand\generalsol{\xrightarrow[\textrm{solution}]{\textrm{ general }}}
\newcommand\ngeneralsol{\parbox{4em}{general \\ solution}\textrm{:}}
\renewcommand\and{\text{and}}
\newcommand\aand{\quad\text{and}\quad}
\newcommand\adj{\text{adj}}
\newcommand\qtq[1]{\quad\textrm{#1}\quad}
\newcommand\oor{\quad\text{or}\quad}
\newcommand\Col{\textrm{Col}}
\newcommand\Nul{\textrm{Null}}
\newcommand\Row{\textrm{Row}}
\newcommand\ran{\textrm{rank}}
\newcommand\dist{\text{dist}}
\newcommand\sz{\stackrel{\textrm{set}}{=}}
\newcommand\ce{\overset{\checkmark}{=}}
\newcommand\proj{\text{proj}}

% Vectors
\renewcommand\a{\vec{a}}
\renewcommand\b{\vec{b}}
\renewcommand\c{\vec{c}}
\renewcommand\d{\vec{d}}
\newcommand\e{\vec{e}}
\newcommand\f{\vec{f}}
\newcommand\p{\vec{p}}
\renewcommand\r{\vec{r}}
\renewcommand\u{\vec{u}}
\renewcommand\v{\vec{v}}
\newcommand\w{\vec{w}}
\newcommand\x{\vec{x}}
\newcommand\y{\vec{y}}
\newcommand\z{\vec{z}}
\newcommand\zero{\vec{0}}
\newcommand\T{^{\top}}
\newcommand\I{^{-1}}

% Hat vectors
\newcommand\ah{\hat{a}}
\newcommand\bh{\hat{b}}
\newcommand\ch{\hat{c}}
\renewcommand\dh{\hat{d}}
\newcommand\eh{\hat{e}}
\newcommand\ph{\hat{p}}
\newcommand\uh{\hat{u}}
\newcommand\vh{\hat{v}}
\newcommand\wh{\hat{w}}
\newcommand\xh{\hat{x}}
\newcommand\yh{\hat{y}}
\newcommand\zh{\hat{z}}

% Subspaces
\newcommand\B{\mathcal{B}}
\newcommand\CC{\mathbb{C}}
\newcommand\C{\mathcal{C}}
\newcommand\D{\mathcal{D}}
\newcommand\E{\mathcal{E}}
\newcommand\F{\mathcal{F}}
\newcommand\R{\mathbb{R}}
\newcommand\U{\mathcal{U}}
\newcommand\X{\mathcal{X}}
\newcommand\Y{\mathcal{Y}}
\newcommand\W{\mathbf{w}}

% Span
\newcommand\Sspan{\text{Span}}

% Edit Section/Subsection/Subsubsection
\makeatletter
\titleformat*{\section}{\sffamily\scshape\fontsize{14}{16}\bfseries}
\titleformat*{\subsection}{\color{subsubsection}\scshape\sffamily\fontsize{13}{15}\bfseries}
\def\@seccntformat#1{\llap{\csname the#1\endcsname\quad}}
\renewcommand\thesection{\arabic{section}}
\makeatother

% Edit equation number display
\makeatletter
\newcommand*\widefbox[1]{\color{main}\fbox{\hspace{2em}#1\hspace{2em}}}
\renewcommand\theequation{\arabic{equation}}
\renewcommand\tagform@[1]{%
  {\color{main}\boxed{\textbf{#1}}}
}
\def\@eqnnum{{\normalfont \normalcolor \theequation}}
\makeatother

% Equivalently
\newcommand\equivalently[1]{[ \textbf{Equivalently:} #1 ]}

% L'Hopital's Rule
\newcommand\lop{\stackrel{\textrm{H}}{=}}

% Check mark
\def\checkmark{\tikz\fill[scale=0.4](0,.35) -- (.25,0) -- (1,.7) -- (.25,.15) -- cycle;} 

% Vinculum
\newcommand\vinculum[1]{\frac{\hspace{#1cm}}{}}


%%%%%%%%%%%%%%%%%%%%%%%%%%%%%%%%%%%%%%%%%%%%%%%%%%%%%%%%%%%%%%%%%%%%%%%%%%%%%%%%
%                                                                              %
%                                 Environments                                 %
%                                                                              %
%%%%%%%%%%%%%%%%%%%%%%%%%%%%%%%%%%%%%%%%%%%%%%%%%%%%%%%%%%%%%%%%%%%%%%%%%%%%%%%%

\mdfsetup{skipabove=1em,skipbelow=0em}
\tcbuselibrary{theorems,skins,hooks}

\newcommand\qedsolution{{\color{qedsolution}\rule{4mm}{1.5mm}}}
\newcommand\qedproof{{\color{qedproof}\rule{4mm}{1.5mm}}}

\declaretheoremstyle[
  headfont=\sffamily\bfseries\color{definition},
  headformat=\fbox{\arabic{definition}}~\NAME\NOTE,
  bodyfont=\normalfont,
  headpunct=,
  mdframed={
    linewidth=0.5pt,
    linecolor=definition,
  },
]{thmdefinitionbox}

\declaretheoremstyle[
  headfont=\sffamily\bfseries\color{theorem},
  headformat=\fbox{\arabic{theorem}}~\NAME\NOTE,
  bodyfont=\normalfont,
  headpunct=,
  mdframed={
    linewidth=0.5pt,
    linecolor=theorem,
  },
]{thmtheorembox}

\declaretheoremstyle[
  headfont=\sffamily\bfseries\color{question}\colorbox{questionfg}{QUESTION \arabic{question}},
  headformat=\textbf{\NOTE},
  notefont=\bfseries,
  bodyfont=\normalfont,
  notefont={\color{black}\bfseries},
  notebraces={~ },
  headpunct=,
]{thmquestionbox}

\declaretheoremstyle[
  headfont=\sffamily\bfseries\color{example},
  headformat=\NAME\NOTE,
  bodyfont=\normalfont,
  headpunct=,
  mdframed={
    linewidth=0.5pt,
    linecolor=example,
    backgroundcolor=examplebg,
  },
]{thmexamplebox}

\declaretheoremstyle[
  headfont=\sffamily\color{solution},
  headformat=\NAME~\arabic{solution},
  notefont=\bfseries,
  headpunct=,
  qed=\qedsolution,
  spaceabove=\topsep,
  spacebelow=\topsep,
]{thmsolutionbox}

\declaretheoremstyle[
  headfont=\sffamily\bfseries\color{remark},
  bodyfont=\itshape,
  notebraces={~ },
  notefont=\bfseries,
  headpunct=,
  mdframed={
    linewidth=0.5pt, linecolor=remark,
    rightline=false, topline=false, bottomline=false,
  },
]{thmremarkbox}

\declaretheoremstyle[
  headfont=\sffamily\color{proof},
  headformat=\NAME,
  headindent=0mm,
  bodyfont=\normalfont,
  notefont=\bfseries,
  headpunct=,
  qed=\qedproof,
]{thmreplacementproofbox}

\declaretheoremstyle[
  headfont=\sffamily\color{main},
  headformat=\NOTE,
  headindent=0mm,
  bodyfont=\normalfont,
  notefont=\bfseries,
  notebraces={~ },
  headpunct=,
  mdframed={
    linewidth=0.5pt, linecolor=main,
    topline=true, bottomline=true, leftline=true, rightline=true,
  }
]{thmpurpleframebox}

\declaretheoremstyle[
  headfont=\sffamily\bfseries,
  headindent=24pt,
  bodyfont=\normalfont,
  notefont=\bfseries,
  notebraces={~},
  headpunct=:,
]{thmmaindefinitionbox}

\declaretheoremstyle[
  headfont=\sffamily,
  headindent=24pt,
  bodyfont=\itshape,
  notefont=\bfseries,
  notebraces={~ },
  headpunct=:,
]{thmmainplainbox}

\declaretheorem[style=thmdefinitionbox,       name=Definition]      {definition}
\declaretheorem[style=thmtheorembox,          name=Theorem]         {theorem}
\declaretheorem[style=thmexamplebox,          name=Example]         {example}
\declaretheorem[style=thmquestionbox,         name=]                {question}
\declaretheorem[style=thmsolutionbox,         name=SOLUTION]        {solution}
\declaretheorem[style=thmremarkbox,           name=Remark]          {remark}
\declaretheorem[style=thmreplacementproofbox, name=PROOF]           {replacementproof}
\declaretheorem[style=thmpurpleframebox,      name=]                {purpleframe}

\declaretheorem[numbered=no,          style=thmmaindefinitionbox,   name=Note]            {note}
\declaretheorem[numbered=no,          style=thmmaindefinitionbox,   name=Problem]         {problem}

\declaretheorem[numbered=no,          style=thmmainplainbox,        name=Lemma]           {lemma}
\declaretheorem[numbered=no,          style=thmmainplainbox,        name=Corollary]       {corollary}
\declaretheorem[numbered=no,          style=thmmainplainbox,        name=Proposition]     {proposition}
\declaretheorem[numbered=no,          style=thmmainplainbox,        name=Conjecture]      {conjecture}
\declaretheorem[numbered=no,          style=thmmainplainbox,        name=Explanation]     {explanation}
\declaretheorem[numbered=no,          style=thmmainplainbox,        name=Notation]        {notation}
\declaretheorem[numbered=no,          style=thmmainplainbox,        name=Recall]          {recall}
\declaretheorem[numbered=no,          style=thmmainplainbox,        name=Claim]           {claim}
\declaretheorem[numbered=no,          style=thmmainplainbox,        name=Case]            {case}
\declaretheorem[numbered=no,          style=thmmainplainbox,        name=Acknowledgment]  {acknowledgment}
\declaretheorem[numbered=no,          style=thmmainplainbox,        name=Conclusion]      {conclusion}

\renewenvironment{proof}[1][\proofname]{\begin{replacementproof}}{\end{replacementproof}}

\crefname{definition}{definition}{definitions}
\Crefname{definition}{Definition}{Definitions}

\crefname{theorem}{theorem}{theorems}
\Crefname{theorem}{Theorem}{Theorems}

\crefname{example}{example}{examples}
\Crefname{example}{Example}{Examples}

\crefname{question}{question}{questions}
\Crefname{question}{Question}{Questions}

\crefname{solution}{solution}{solutions}
\Crefname{solution}{Solution}{Solutions}

\crefname{remark}{remark}{remarks}
\Crefname{remark}{Remark}{Remarks}

\crefname{note}{note}{notes}
\Crefname{note}{Note}{Notes}

\crefname{problem}{problem}{problem}
\Crefname{problem}{Problem}{Problems}

\crefname{lemma}{lemma}{lemmas}
\Crefname{lemma}{Lemma}{Lemmas}

\crefname{corollary}{corollary}{corollaries}
\Crefname{corollary}{Corollary}{Corollaries}

\crefname{proposition}{proposition}{propositions}
\Crefname{proposition}{Proposition}{Propositions}

\crefname{conjecture}{conjecture}{conjectures}
\Crefname{conjecture}{Conjecture}{Conjectures}

\crefname{explanation}{explanation}{explanations}
\Crefname{explanation}{Explanation}{Explanations}

\crefname{notation}{notation}{notations}
\Crefname{notation}{Notation}{Notations}

\crefname{recall}{recall}{recalls}
\Crefname{recall}{Recall}{Recalls}

\crefname{claim}{claim}{claims}
\Crefname{claim}{Claim}{Claims}

\crefname{case}{case}{cases}
\Crefname{case}{Case}{Cases}

\crefname{acknowledgment}{acknowledgment}{acknowledgments}
\Crefname{acknowledgment}{Acknowledgment}{Acknowledgments}

\crefname{conclusion}{conclusion}{conclusions}
\Crefname{conclusion}{Conclusion}{Conclusions}


%%%%%%%%%%%%%%%%%%%%%%%%%%%%%%%%%%%%%%%%%%%%%%%%%%%%%%%%%%%%%%%%%%%%%%%
%                                                                     %
%                          Table of Contents                          %
%                                                                     %
%%%%%%%%%%%%%%%%%%%%%%%%%%%%%%%%%%%%%%%%%%%%%%%%%%%%%%%%%%%%%%%%%%%%%%%

\contentsmargin{0cm}
\makeatletter

\titlecontents{chapter}[3.7pc]
{%
  \addvspace{30pt}%
  \begin{tikzpicture}[remember picture, overlay]%
    \draw[fill=tbl,draw=tbl] (-7,-.1) rectangle (-0.7,.5);%
    \pgftext[left,x=-3.7cm,y=0.2cm]{\color{white}\Large\sc\bfseries \lecorchap~\thecontentslabel\\*\hspace*{.7em}\ \thecontentslabel};%
  \end{tikzpicture}%
  \color{tbl}\large\sc\bfseries%
}%
{}
{}
{\;\titlerule\;\large\sc\bfseries Page \thecontentspage
	\begin{tikzpicture}[remember picture, overlay]
		\draw[fill=tbl,draw=tbl] (2pt,0) rectangle (4,0.1pt);
	\end{tikzpicture}}%

\titlecontents{section}[3.7pc]
{\addvspace{2pt}}
{\contentslabel[\thecontentslabel]{2pc}}
{}
{\hfill\small \thecontentspage}
[]

\titlecontents{subsection}[5.75pc]
{\addvspace{2pt}}
{\contentslabel[\thecontentslabel]{2pc}}
{}
{\hfill\small \thecontentspage}
[]%

\renewcommand{\tableofcontents}{%
  \chapter*{%
    \vspace*{-20\p@}%
    \begin{tikzpicture}[remember picture, overlay]%
      \pgftext[right,x=15cm,y=0.2cm]{\color{tbl}\Huge\sc\bfseries \contentsname};%
      \draw[fill=tbl,draw=tbl] (13,-.75) rectangle (20,1);%
      \clip (13,-.75) rectangle (20,1);
      \pgftext[right,x=15cm,y=0.2cm]{\color{white}\Huge\sc\bfseries \contentsname};%
    \end{tikzpicture}}%
  \@starttoc{toc}}

\makeatother
